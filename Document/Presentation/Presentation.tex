\documentclass{beamer}
\usetheme{default}

% Miscellaneous
\usepackage{etoolbox}

% Layout

% Graphics
\usepackage[footnotesize]{caption} % For captions
%\usepackage{caption} % For subcaptions
\usepackage{subcaption} % For subcaptions
\usepackage{tikz} % For advanced graphics functions
\usepackage{pgfplots}

% TikZ
% Two-dimensional versions
\newrobustcmd{\drawplus}[3]{;\draw (#1+#3/2,#2) -- (#1+#3/2,#2+#3); \draw (#1,#2+#3/2) -- (#1+#3,#2+#3/2);}
\newrobustcmd{\squarepath}[1]{-- ++(#1,0) -- ++(0,#1) -- ++(-#1,0) -- cycle}
% Three-dimensional versions
\newrobustcmd{\drawthreedimplus}[4]{\def\cx{#1} \def\cy{#2} \def\cz{#3} \def\s{#4} \drawthreedimplushelper}
\newrobustcmd{\drawthreedimplushelper}[6]{;\draw (\cx+\s*#1/2,\cy+\s*#2/2,\cz+\s*#3/2) -- +(\s*#4,\s*#5,\s*#6); \draw (\cx+\s*#4/2,\cy+\s*#5/2,\cz+\s*#6/2) -- +(\s*#1,\s*#2,\s*#3);}
\newrobustcmd{\threedimsquarepath}[7]{-- ++(#1*#2,#1*#3,#1*#4) -- ++(#1*#5,#1*#6,#1*#7) -- ++(-#1*#2,-#1*#3,-#1*#4) -- cycle}

\begin{document}

\begin{frame}
\title{Wave Model and Watercraft Model for Simulation of Sea State} % The title of the report
\author{Kristofer Krus} % Your name
%\date{\today} % The date of ... (?)
\date{\mbox{August 23}, 2013} % The date of ... (?)
\titlepage
\end{frame}

\begin{frame}
\frametitle{Outline}
\tableofcontents
%\tableofcontents[pausesections] % Creates a new slide for each additional section
\end{frame}

\part{Introduction}


%\fbox{\setlength\fboxsep{1pt}
%\fbox{\setlength\fboxsep{2pt}
%\fbox{\setlength\fboxsep{3pt}
%\fbox{\setlength\fboxsep{4pt}
%\fbox{\setlength\fboxsep{5pt}
%\fbox{\setlength\fboxsep{6pt}
%\fbox{\setlength\fboxsep{7pt}
%\fbox{\setlength\fboxsep{8pt}
%hej!
%}
%}
%}
%}
%}
%}
%}
%}

\subsection{Motivation}

\begin{frame}[<+(1)->]
\frametitle{Motivation}

\uncover<+(1)->{Helikoptersimulator.}
%\begin{itemize}[<+(1)->]
%\item Helikoptersimulator
%item Wave dispersion
%end{itemize}

\uncover<+(1)->{\href{http://www.youtube.com/watch?v=bC2XIGMI2kM}{Lynx Helicopter Operating Limit Development}}

\end{frame}

\chapter{Requirements and difficulties}
\label{chap:requirementsanddifficulties}

\section{Wave dispersion and non-linearity}

One of the greatest challenges when simulating an ocean is the fact that ocean waves are subject to \idxs{wave}{dispersion}, also known as \indexify{frequency dispersion}\index{frequency dispersion!see{dispersion}}, which means that waves with different \wavelengths travel with different speeds. This means that the wave equation,
%
\begin{equation} \label{eq:wave_equation}
\frac{\partial^2 \eta}{\partial t^2} \,=\, c^2\nabla^2\eta,
\end{equation}
%
--- a \PDE\xspace --- which otherwise both has a simple definition and is simple to solve numerically, cannot be used, since it assumes that waves of all wavelengths travel with a single speed, $c$. Here, $\eta$ is the free surface elevation as a function of the the horizontal location, $\vec{r}$, and the time, $t$, and $\nabla$ is the \idxs{del}{operator}\index{$\nabla$|see{del operator}} which is commonly used in \idxs{vector}{algebra}.

In Airy wave theory, which treats the propagation of surface waves, the dispersion relation
%
\begin{equation} \label{eq:dispersion}
\omega^2(k) \,=\, \left(g\,+\,\frac{\gamma}{\rho}\,k^2\right)\,k\,\tanh(k\,h),
\end{equation}
%
is derived for water with no mean velocity \citep{Phillip1977}. Here, $\omega$ is the \idxs{angular}{frequency} of one \idxs{wave}{component}, $k$ is the \idx{wavenumber} of the component, $g$ is the \idxs{gravitational}{acceleration}, $\gamma$ is the \idxs{surface}{tension}, $\rho$ is the water density and $h$ is the \idxs{water}{depth}.

While this equation describes the propagation of a wave composed of a single wavelength very well, it cannot tell how a free surface elevation, $\eta$, consisting of multiple wavelengths will evolve. Only if the wave amplitude is very small (typically such that $|\nabla\eta| \ll 1$ and $|\eta| \ll h$, where $h$ is the water depth) can the surface be approximated as linear, and waves with different wavelengths can be individually described by \eqref{eq:dispersion} and superposed on top of each other to form the free surface elevation, without introducing too much error. If the \idxs{wave}{amplitude} on the other hand is not that small, strong non-linear phenomena are likely to take place, including for example \idxs{wave}{breaking}, which won't be caught in the simulation if the surface is linearized.

What is maybe even worse is that there is no simple way of turning this equation into a \PDE represented in the spatial domain (like \eqref{eq:wave_equation}). This increases the difficulty to describe the evolution of the free surface elevation significantly, even for a surface that has already been linearized. This is typically solved by transforming the free surface elevation in some way before processing it.

\section{Fluid--Structure Interaction}

As noted previously, ships have to be affected by waves, and ships also have to give rise to waves. Hence, there has to be a \idxs{two-way}{interaction} between water and ships. For most of the two-dimensional wave models, which just treat the surface as a \idxs{height}{map}, there is no natural way to make water and ships interact with each other.

The ships can quite easily be made to roll in a realistic way by just approximating the \idxs{pressure}{field} felt by the \idxs{ship}{hull} by looking at the free surface elevation, even though this method is slightly incorrect since it doesn't take into account the deviations the ship itself causes the pressure field. But to make ships give rise to waves as they are traveling on the water is more challenging.

One possibility is to use a separate, static height map for the wake, which moves after the ship as it is traveling. This wake will always look the same no matter where the ship is traveling and will not be affected by obstacles in the water. However, if the ship suddenly changes speed or course, so does the wake, which is a highly unnatural behavior for a wake.

A better approach may be to use a \idxs{response}{map} that tells the water how to respond when a ship is traveling on it. In that case, the wake will not be stored as a separate height map, but be merged into the same height map that is used to simulate the waves that affect the ship. The response would of course also depend on the speed of the ship so that the faster the ship goes, the higher the generated waves will be, and for a non-moving ship, there will be no waves generated.
\begin{frame}[<+(1)->]
\frametitle{Related work}

Two-dimensional methods

\begin{itemize}
\item item 1
\item item 2
\item item 3
\end{itemize}

Three-dimensional methods

\begin{itemize}
\item item 4
\item item 5
\item item 6
\end{itemize}

\end{frame}
\section{Metoder som fungerar}

\subsection{Fouriersyntes}

\newcounter{frequencydomainpauses}

\begin{frame}
\frametitle{Fouriersyntes}

\begin{columns}[c]

\column{.7\textwidth}

\begin{itemize}[<+(1)->]
\item Vattnet modelleras som en höjdkarta: $h(\vec{x})$
\item Simulering sker i frekvensdomänen: $H(\vec{k\,})$
    \setcounter{frequencydomainpauses}{\thebeamerpauses}
\item Uppdatering av vattenytan: $\displaystyle H_{n+1}(\vec{k\,}) = e^{i\Delta t\sqrt{gk}}*H_{n}(\vec{k\,})$
%$\displaystyle \widetilde{h}(\vec{k\,}) \,*=\, (\cos(\sqrt{gk}) + i*\sin(\sqrt{gk}))
\item Snabb invers fouriertransform för att få höjdkartan 
    \begin{itemize}[<+(1)->]
    \item Ignorera imaginär komponent
    \end{itemize}
\item $\text{FFT}^{-1} \Rightarrow O(n \log n)$
\end{itemize}

\column{.3\textwidth}

\uncover<\thefrequencydomainpauses->{
\begin{figure}
\centering
\includegraphics[width=\textwidth]{Images/Other/Fourier_transform}
\end{figure}%\vfil{1em}
\begin{figure}
\centering
\includegraphics[width=\textwidth]{Images/Other/spectrum_res_increment}
\end{figure}

}

\end{columns}

\end{frame}

\begin{frame}
\frametitle{Fouriersyntes, pseudokod}

\begin{itemize}[<+(1)->]
\item Uppdatering av vattenytan: $\displaystyle H_{n+1}(\vec{k\,}) = e^{i\Delta t\sqrt{gk}}*H_{n}(\vec{k\,})$
\item Pseudokod: \texttt{\\
for all k\_vec in H's domain \{         \\
\ \ \ \ float k = k\_vec.length();      \\
\ \ \ \ float a = dt*sqrt(g*k);         \\
\ \ \ \ H(k\_vec) *= cos(a) + i*sin(a); \\
\}                                      \\
h = Re(invFFT(H));                      \\
}
\end{itemize}


\end{frame}

\begin{frame}
\frametitle{Fouriersyntes, videos}

\begin{itemize}
\item \href{http://www.youtube.com/watch?v=sf6EVn2Zgk4}{MerCUDA, a real time GPU-based ocean simulator}
\item \href{http://www.youtube.com/watch?v=Lj_V5-bTvK0}{Offshore rescue}
\item \href{http://www.youtube.com/watch?v=3YW9WFwD-rI}{Real time stormy ocean 3D}
\item \href{http://www.youtube.com/watch?v=ocHyTiHEphg}{Ship Simulator Extremes Gameplay}
\end{itemize}

\end{frame}

\subsection{Skalrum}

\begin{frame}
\frametitle{Skalrum}

\begin{itemize}[<+(1)->]
\item Vattnet modelleras som en höjdkarta: $h(\vec{x})$
\item Olika våglängder sparas på olika griddar
\item Gridupplösningen avgörs av kortaste våglängden
    \begin{itemize}[<+(1)->]
    \item T.ex. 1--2 m, 2--4 m, 4--8 m, o.s.v.
    \end{itemize}
\item Vågekvationen löses separat på varje grid
\item Höjdkartan ges av summan av alla griddar
\item Geometrisk summa $\Rightarrow O(n)$
\end{itemize}

\end{frame}

\section{Implementerad metod}

\chapter{The Finite volume method}
\label{chap:ns_equations}

%TODO: Insert time values for more discretisized variables
%TODO: Insert error bounds for more discretized equations

%\newrobustcmd{\gammapath}{{\gamma[\vec{r}_1,\,\vec{r}_2]}}
%\newabbrev{\textgammapath}{\mbox{$\gammapath$}}

The \FVM is a way of solving a \PDE, or a set of \PDEs, where the room is \discretized into a large number of non-moving, adjacent volume elements which is commonly referred to as \cells. Different \properties are discretized into certain points. \idxse{scalar}{field}{Scalar fields} are usually discretized to the \idxsp{cell}{center}{s}, or sometimes to \idxp{node}{s} on the \idxsp{cell}{corner}{s}, which can be convenient since \interpolation of fields discretized to the cell centers tend to be more difficult \citep{Losasso2004}. In a \idxs{collocated}{grid}, all properties are stored at the same locations, so the \idxse{vector}{property}{vector properties} are discretized to the same locations as the \idxse{scalar}{property}{scalar properties}. On a \idxs{staggered}{grid} on the other hand, the \velocity (or the \momentum, depending on implementation) is discretized to the \idxsp{cell}{face}{s}. For \thisprojectwork, a staggered grid has been used, and throughout the report the discretization locations for the various properties will be called \idxsp{storage}{location}{s}.

\section{Fluid simulation}

The \FVM can handle \simulation of fluids in a realistic way. It is natural to use for the representation of fluids since the fluids, which are continuous medias (at least on the level they are simulated), are represented in a continuous ways. This can be contrasted to representing the fluids as \idxse{fluid}{particles}{particles}, which are discrete.

When using the \FVM, the simulation can be concentrated to the interesting parts of the flow by using \idxs{adaptive}{mesh refinement} \citep{Popinet2003,Losasso2004}, to speed up the simulation orders of magnitude without loosing orders of magnitude in numerical precision in the interesting parts of the flow. This doesn't really have any natural correspondence when using particles. It is possible to initialize the particles with different \idxsp{effective}{size}{s} in order to make some parts of flow have a lower \idxs{particle}{density} and hence require less \idxs{computational}{power} per unit volume; in that way \idxs{numerical}{precision} can be traded for speed. However, this introduces another problem --- as the flow evolves, \advection may cause large particles to end up at places in the flow where high numerical precision is desired and decrease the numerical precision to under the required level. Besides, if the particles have a high velocity relative to each other (a high \temperature), \diffusion will cause large particles and small particles to mix and the large particles will once again end up at places in the flow where high numerical precision is desired.

One naive attempt to solve this problem could be to dynamically resize the particles as they end up in parts of the flow with different requirements on the numerical precision. However, this will add or remove mass a those locations, so the simulation will not \idxse{conservation of}{mass}{conserve mass}. A succeeding naive attempt to solve this problem in turn could be to distribute mass that has been removed uniformly over all other particles by scaling them with a factor, but that would lead to a non-physical transportation of mass which would move the center of mass, and it would even cause the simulation to not \idxse{conservation of}{momentum}{conserve momentum}. A better remedy to this problem is to split too large particles into smaller particles and merge too small particles into larger particles as first done in \citeyear{Desbrun1999} \citep{Desbrun1999} and later improved in a number of works \citep[e.g.][]{Yan2009}. However, these techniques still require a fairly large number of particles and are hence not suitable for real time simulations of water surfaces on large bodies of water.

When applying the \FVM in \CFD, it simulates the \flow of a \fluid by dividing the fluid into a large number of non-moving, adjacent \cells and letting the fluid flow between the cells, through the \idxsp{cell}{face}{s}. The motion of the fluid is described by a set of \PDEs, usually the \idxs{Euler}{equations} or the \idxs{Navier--Stokes}{equations}.

The main difference between the Navier--Stokes equations and the Euler equations is that the Navier--Stokes equations takes \index{visousity|see{viscous force}}\idxsp{viscous}{force}{s} into account whe\-reas the Euler equations do not. The Euler equations are therefore a special case of the Navier--Stokes equations. Many textbooks also omit \idxsp{external}{force}{s} when writing about the Euler equations, although gravity which is such an external force usually is included when simulating \idxs{free surface}{flow} using the Euler equations.

In simulations where the viscous force plays a big role, the Euler equations are not sufficient and so the Navier--Stokes equations are usually used, otherwise the Euler equations usually works as well as the Navier--Stokes equations and are even preferred to the Navies--Stokes equations because of the simplification they imply for the model as well as the computations. In \thisprojectwork, it is the Euler equations that are solved.

\section{Divergence calculation}

In the \PDEs, in order to calculate the \divergence of a \idxs{vector}{field}, the \idxs{divergence}{theorem} is used and a \idxs{volume}{integral} of the divergence of the field is converted to a \idxs{surface}{integral} of the vector field itself. The divergence theorem states that

\begin{equation} \label{eq:divergence_theorem}
\iiint_V\nabla\cdot\vec{F}(\vec{r'})\,\opd V \,=\, \oiint_S(\vec{F}(\vec{r'})\cdot\normal)\,\opd S
\end{equation}

where $\vec{F}$ is a vector field, $V$ is a \idxs{control}{volume}, which in our case is the cell surrounding the point $\vec{r}$ in which the divergence is to be calculated, $S$ is the surface of $V$, with \idxs{normal}{vector} pointing outwards, $\opd V$ and $\opd S$ are \infinitesimal elements in $V$ and $S$ respectively, $\normal$ is the normal of $\opd S$ and $\vec{r'}$ is the position of $\opd V$ and $\opd S$ respectively. The divergence of $\vec{F}(\vec{r})$ is then \approximated as the \average divergence of $\vec{F}$ in $V$ and calculated as

\begin{equation} \label{eq:divergence_surface_integral}
\nabla\cdot\vec{F}(\vec{r}) \,=\, \frac{1}{V}\,\oiint_S(\vec{F}\cdot\normal)\,\opd S.
\end{equation}

In the \FVM, the surface of a cell consists of \idxsp{cell}{face}{s}, $S_i$, between the cell itself and \neighboring cells, so \eqref{eq:divergence_surface_integral} can be rewritten as

\begin{equation} \label{eq:divergence_cell_face_sum}
\nabla\cdot\vec{F}(\vec{r})\ =\ \frac{1}{V}\,\sum_{S_i} \oiint_{S_i}(\vec{F}\cdot\normal)\,\opd S\ =\ \frac{1}{V}\,\sum_{S_i} F_i\,S_i,
\end{equation}

where $i$ is an index, $S_i$ is the \area of the cell face to the $i$th neighbor cell and $F_i$ is the average \idxs{field}{flux} through $S_i$, defined as

\begin{equation} \label{eq:fi_integral}
F_i \,=\, \frac{1}{S_i}\oiint_{S_i}(\vec{F}\cdot\normal)\,\opd S.
\end{equation}

In the \FVM, cell faces are usually flat, which means that the normal vector $\normal$ is constant for a certain cell face $S_i$. \eqref{eq:fi_integral} can therefore be rewritten as

\begin{equation} \label{eq:fi_flat_cell_face}
F_i \,=\, \frac{1}{S_i}\,\normal_i\cdot\oiint_{S_i}\vec{F}\,\opd S,
\end{equation}

where $\normal_i$ is the normal of $S_i$. $F_i$, which is now just the $\normal_i$-component of the average value of the field on the cell face $S_i$, is on a staggered grid stored directly on $S_i$.

\section{Gradient calculation}

For \idxsp{orthogonal}{grid}{s}, the \gradient of a \idxs{scalar}{field} is calculated in a similar way, but in this case the \idxs{gradient}{theorem} is used. The gradient theorem states that

\begin{equation} \label{eq:gradient_theorem}
\phi(\vec{r}_2)-\phi(\vec{r}_1) \,=\, \int_\gammapath\nabla\phi(\vec{r'})\cdot \opd\vec{r'},
\end{equation}

where $\phi$ is a scalar field, \textgammapath is a path within $\phi$'s domain, connecting the vectors $\vec{r}_1$ and $\vec{r}_2$ and $\int_\gammapath$ denotes a \idxs{path}{integral} along \textgammapath. By dividing both sides of \eqref{eq:gradient_theorem} with \mbox{$\Delta r = |\vec{r}_2\,-\,\vec{r}_1|$}, we obtain

\begin{equation} \label{eq:gradient_theorem_divided}
\frac{\phi(\vec{r}_2)-\phi(\vec{r}_1)}{\Delta r} \,=\, \frac{\int_\gammapath\nabla\phi(\vec{r'})\cdot \opd\vec{r'}}{\Delta r} \,=\, \frac{\int_\gammapath\nabla\phi(\vec{r'})\cdot\frac{\Delta\vec{r}}{|\Delta\vec{r}|} \opd r'}{\Delta r}
\end{equation}

where $\Delta\vec{r} = \vec{r}_2 -  \vec{r}_1$ and

\begin{equation}
\nabla\phi(\vec{r})\cdot\frac{\Delta\vec{r}}{|\Delta\vec{r}|} = \phi'_{\Delta\vec{r}}(\vec{r}),
\end{equation}

where $\phi'_{\vec{v}}$ is the \derivative of $\phi$ in the direction of $\vec{v}$. By assuming the simplest path possible from $r_1$ to $r_2$, which is just a line segment, $\Delta r$ can be written as

\begin{equation}
\Delta r = \int_\gammapath\opd r'
\end{equation}

and \eqref{eq:gradient_theorem_divided} becomes

\begin{equation} \label{eq:phi_derivative_integral}
\frac{\phi(\vec{r}_2)-\phi(\vec{r}_1)}{\Delta r} \,=\, \frac{\int_\gammapath\phi'_{\Delta\vec{r}}(\vec{r'})\opd r'}{\int_\gammapath\opd r'}
\end{equation}

where the right hand side can be identified as the \average value of $\phi'_{\Delta\vec{r}}(\vec{r})$ along the path \textgammapath. Provided that $\vec{r}$ is close enough to \textgammapath (preferably equal to \mbox{$(\vec{r}_1\,+\,\vec{r}_2)/2$}), $\phi'_{\Delta\vec{r}}(\vec{r})$ can be \approximated as this average and calculated as

\begin{equation} \label{eq:phi_derivative_final}
\phi'_{\Delta\vec{r}}(\vec{r}) \,=\, \frac{\phi(\vec{r}_2)-\phi(\vec{r}_1)}{\Delta r}.
\end{equation}

The gradient of a scalar field can be written as

\begin{equation} \label{eq:gradient_orthogonal}
\nabla\phi(\vec{r}) \,=\, \left(\sum_{i=0}^{d-1}\frac{\partial}{\partial r_i}\,\normvec{e}_i\right) \phi(\vec{r}) \,=\, \sum_{i=0}^{d-1}\phi'_{\normvec{e}_i}(\vec{r})\,\normvec{e}_i,
\end{equation}

where $\{\normvec{e}_i\}$ is an \idxs{orthonormal}{basis} for $\mathbb{R}^d$, where $d$ is the number of \dimensions and $i = 1,\,2,\,...\,,d-1$; $\normvec{e}_i$ is a \idxs{base}{vector} in $\{\normvec{e}_i\}$ that is aligned with the $i$th \idxs{grid}{axis} and $r_i$ is the $\normvec{e}_i$ component of $\vec{r}$,
such that $r_i = \normvec{e}_i\cdot\vec{r}$. Since we are on an orthogonal grid, we can assume that the location $\vec{r}$ in which the gradient is to be calculated will be the center (or corner) of a cell with $2d$ neighboring cell centers (or cell corners):

\begin{equation} \label{eq:neighboring_locations}
\begin{cases}
\vec{r}_{\normvec{e}_i^-} \,=\, \vec{r} \,-\, \Delta r_i\,\normvec{e}_i\\[1ex]
\vec{r}_{\normvec{e}_i^+} \,=\, \vec{r} \,+\, \Delta r_i\,\normvec{e}_i\\[1ex]
\end{cases}\ ,
\end{equation}

%\begin{samepage}
where $\Delta r_i$ is the grid spacing in $\normvec{e}_i$ direction. By combining \eqrefs \ref{eq:phi_derivative_final}, \ref{eq:gradient_orthogonal} and \ref{eq:neighboring_locations}, we can write the gradient of $\phi$ as

\begin{equation} \label{eq:gradient_final}
\nabla\phi(\vec{r}) \,=\,
\sum_{i=0}^{d-1}\frac{\phi(\vec{r}_{\normvec{e}_i^+})-\phi(\vec{r}_{\normvec{e}_i^-})}{2\,\Delta r_i}\,\normvec{e}_i.
\end{equation}
%\end{samepage}

\section{Navier--Stokes equations}
\label{sec:ns_equations}

The \idxs{Navier--Stokes}{equations} are a statement of the conservation of momentum for a fluid. The \idxse{general form of the}{equations of fluid motion}{general form of the} equations reads

\begin{equation} \label{eq:navier_stokes}
\rho\left(\frac{\partial\vec{u}}{\partial t} + \vec{u}\cdot\nabla\vec{u}\right) = -\nabla p + \nabla\cdot\boldsymbol{\mathsf{T}} + \vec{f},
\end{equation}

where $\rho$ is the density, $\vec{u}$ is the velocity, $p$ is the pressure, $\devstressten$ is the \index{stress tensor|see{deviatoric stress tensor}}\idxs{deviatoric stress}{tensor}, which includes \idxsp{viscous}{force}{s}, and $\vec{f}$ is the external forces per unit volume. For the case of \idxs{Eulerian}{flow}, ${\boldsymbol{\mathsf{T}} = 0}$ so ${\nabla\cdot\boldsymbol{\mathsf{T}}}$ vanishes from the equation. It should be noted that these equations do not fully describe the behavior of the fluid; for example, they do not model the effects of \idxs{surface}{tension} or describe diffusion of various properties such as \temperature within the fluid, nor do they describe how to obtain any of the fields $p$, $\boldsymbol{\mathsf{T}}$ or $\vec{f}$ that are needed in order to solve the Navier--Stokes equations.

By \idxse{time}{discretization}{time discretizing} and rewriting \eqref{eq:navier_stokes}, and choosing the value of $\vec{u}$ in \timestep $n$ and the value of $\partial\vec{u}/\partial t$ in \timestep $n+\frac{1}{2}$, thus introducing an $O(\Delta t)$ error where $\Delta t$ is the length of the time step, we obtain

\begin{equation} \label{eq:navier_stokes_time_discretized}
\vec{u}_{n+1}  = \vec{u}_{n} + \Delta t\left(-\vec{u}_{n}\cdot\nabla\vec{u}_{n} \,+\, \frac{-\nabla p + \nabla\cdot\boldsymbol{\mathsf{T}} + \vec{f}}{\rho}\right),
\end{equation}

where $\vec{u}_{n}$ denotes the velocity in time step $n$. Using a method described e.g.\ in a report from \citeyear{Losasso2004} \citep{Losasso2004}, this \PDE can be solved in two steps. First, an auxiliary velocity $\vec{u}^*_n$ that ignores the pressure term is calculated, that is

\begin{equation} \label{eq:auxiliary_velocity}
\vec{u}^*_n  = \vec{u}_{n} + \Delta t\left(-\vec{u}_{n}\cdot\nabla\vec{u}_{n} \,+\, \frac{\nabla\cdot\boldsymbol{\mathsf{T}} + \vec{f}}{\rho}\right),
\end{equation}

and second, the velocity update is calculated as

\begin{equation} \label{eq:velocity_update}
\vec{u}_{n+1} \,=\, \vec{u}^*_n - \Delta t\nabla p_{n+1},
\end{equation}

where $p_n$ denotes the pressure in time step $n$.

\section{Continuity equation}

For a \idxs{control}{volume} $V$ with surface $S$ and a surface normal $\normal$ pointing outwards, the amount of \idxs{mass}{flux} $\opd m/\opd t$ entering the control volume can be described by

\begin{equation} \label{eq:mass_flux_surface_integral}
\frac{\opd m}{\opd t} \,=\, -\oiint_S(\rho\vec{u}\cdot\normal)\,\opd S,
\end{equation}

where $m$ is the mass of the fluid in $V$. By using the \idxs{divergence}{theorem} (\eqref{eq:divergence_theorem}) and dividing with $V$, we can rewrite \eqref{eq:mass_flux_surface_integral} as

\begin{equation} \label{eq:mass_flux_volume_integral}
\frac{\opd (m/V)}{\opd t} \,=\, -\frac{1}{V}\iiint_V\nabla\cdot(\rho\vec{u})\,\opd V
\end{equation}

and in the limit where $V \,\rightarrow\, 0$, this equation turns into

\begin{equation} \label{eq:density_partial_time_derivative}
\frac{\partial \rho}{\partial t} \,=\, -\nabla\cdot(\rho\vec{u}),
\end{equation}

where the density is defined as $\rho = \opd m/\opd V$. This can be rewritten as

\begin{equation} \label{eq:continuity_equation}
\frac{\partial \rho}{\partial t} + \nabla\cdot(\rho\vec{u}) \,=\, 0.
\end{equation}

This is known as the \idxs{continuity}{equation} and has to be satisfied in order to ensure \idxs{conservation of}{mass}. \idxse{time}{discretization}{Time discretizing} this equation, and choosing to use the values of $\rho$ and $\vec{u}$ in \timestep $n$ and the value of $\partial \rho/\partial t$ in time step $n+\frac{1}{2}$, thus introducing an $O(\Delta t)$ error, gives

\begin{equation} \label{eq:continuity_equation_time_discretized}
\rho_{n+1} \,=\, \rho_{n} - \Delta t\,\nabla\cdot(\rho_{n}\vec{u}_{n}),
\end{equation}

where $\rho_n$ denotes the density in \timestep $n$.

\section{Pressure calculation}
\label{sec:pressure_calculation}

Neither the Euler equations nor the Navier--Stokes equations specify how the pressure should be calculated, so when solving either of these two sets of equations one is essentially free to calculate the pressure in whichever way desired. Together with a \idxs{pressure}{model}, the two sets of equations come in two major forms which usually differs significantly in implementation and stability: The \indexs{compressible Navier--Stokes}{equations}\indexs{compressible Euler}{equations}\compressible forms and the \indexs{incompressible Navier--Stokes}{equations}\indexs{incompressible Euler}{equations}\incompressible forms. For \thisprojectwork, the compressible Euler equations are solved.

\subsection{Compressible flow}

In nature, all fluids are compressible, so a physically correct \idxs{pressure}{model} will let the fluids contract and expand which means that for \indexs{compressible}{fluid}\idxs{compressible}{flow}, the \divergence of the \idxs{velocity}{field} is allowed to be non-zero. The \pressure is then usually expressed as a function of the density, sometimes also taking into account the \temperature and other \properties that may affect the pressure, that is

\begin{equation} \label{eq:pressure_compressible_flow}
p = p\,(\rho,\,T,\,\text{other material properties}),
\end{equation}

where $T$ is the temperature.

However, the set of fluid motion equations including \eqref{eq:pressure_compressible_flow} is \idxse{stiff}{equation}{stiff}, meaning that if an ordinary \idxs{explicit}{method} such as the \idxs{forward Euler}{method} is used, the \timestep has to be taken to be extremely small, or the method will be unstable. Ordinary \idxsp{numerical}{method}{s} for solving this set of equations are known to give rise to \idxs{spurious}{oscillations} in the solutions when the \idxs{speed of}{sound} multiplied by the \idx{time step} becomes too large in relation to the \idxs{characteristic}{length} of the \cells, making the numerical method \unstable. More generally, this stringent restriction of the \timestep is known as the \CFL condition.

Still, not all solvers for compressible flow suffers from this problem. In a work from \citeyear{Kwatra2009} \citep{Kwatra2009}, the \CFL condition is alleviated by introducing a \idxs{pressure}{field}, separated from the \idxs{density}{field}, and updating the pressure and velocity fields by using what looks like the \index{implicit Euler method|see{backward Euler method}}\idxs{backward}{Euler method}, which leads to a \idxs{Poisson}{equation} for solving the pressure field. The remaining fields are then evolved with the standard (forward) \idx{Euler method}. This technique doesn't lead to spurious acoustic oscillations and is similar to the technique used for solving incompressible flow, which also gives rise to a Poisson equation for solving the pressure field. In the limit where the speed of sound goes to infinity, it leads to the same Poisson equation as for incompressible flow \citep{Kwatra2009}.

\subsection{Incompressible Navier--Stokes equations}

When acoustic waves is of no significant importance to the simulation, it is probably most common to model the fluids as \indexs{incompressible}{fluid}\indexs{incompressible}{flow}\incompressible. When simulating incompressible flow, a different approach is taken to calculate the \idxs{pressure}{field}.

Since the flow is incompressible, the density will be constant, which means that \derivatives of $\rho$ vanishes, that is

\begin{equation} \label{eq:density_partial_time_derivative_incompressible_flow}
\frac{\partial \rho}{\partial t} \,=\, 0
\end{equation}

and

\begin{equation} \label{eq:density_divergence_incompressible_flow}
\nabla\rho \,=\, \vec{0}.
\end{equation}

\eqref{eq:continuity_equation} will then turn into

\begin{equation} \label{eq:velocity_divergence_incompressible_flow}
\nabla\cdot\vec{u} \,=\, 0.
\end{equation}

By combining \eqref{eq:density_divergence_incompressible_flow} and \eqref{eq:velocity_divergence_incompressible_flow}, \eqref{eq:continuity_equation_time_discretized} turns into just

\begin{equation} \label{eq:continuity_equation_superfluous}
{\rho_{n+1} \,=\, \rho_n}
\end{equation}

and becomes superfluous. Furthermore, it turns out that when the flow is incompressible, we have

\begin{equation} \label{eq:deviatoric_stress_tensor_incompressible_flow}
\nabla\cdot\boldsymbol{\mathsf{T}} \,=\, \mu\nabla^2\vec{u},
\end{equation}

where $\mu$ is the (dynamic) \index{dynamic viscosity|see{viscosity}}\viscosity. For simplicity, we can assume that we use a set of units where $\rho = 1$. \eqref{eq:auxiliary_velocity} can then be rewritten as

\begin{equation} \label{eq:auxiliary_velocity_reduced}
\vec{u}^*_n \,= \, \vec{u}_{n} + \Delta t(- \vec{u}_{n}\cdot\nabla\vec{u}_{n} \,+\, \mu\nabla^2\vec{u}_{n} + \vec{f}),
\end{equation}

which can be directly solved assuming that $\mu$ and $\vec{f}$ are known.

However, \eqref{eq:velocity_update}, which is used to update the velocity, contains $p$ which is a second unknown and must be calculated before $\vec{u}$ can be calculated. Following standard procedure \citep{Losasso2004}, by rewriting \eqref{eq:velocity_update}, taking the divergence of both sides and using \eqref{eq:velocity_divergence_incompressible_flow} to get rid of $\nabla\cdot\vec{u}$, we obtain the \idxs{Poisson}{equation}

\begin{equation} \label{eq:pressure_poissin_equation_incompressible_flow}
\nabla^2 p_{n+1} \,=\, \frac{\nabla\cdot\vec{u}^*_n}{\Delta t}
\end{equation}

which needs to be solved before we can update the velocity completely.

When \idxse{spatial}{discretization}{discretizing} this equation spatially a \idxs{system of linear}{equations} is obtained, for which there exist many solution methods with varying speed and accuracy. As a comparison, it can be noted that the naive \idxs{Gaussian}{elimination}\index{algorithm!Gaussian elimination|see{Gaussian elimination}} algorithm, or the \idxs{Gauss--Jordan}{elimination}\index{algorithm!Gauss--Jordan elimination|see{Gauss--Jordan elimination}} algorithm for a \idx{multi-core} system, has a \idxs{time}{complexity} of $O(N^3)$ for an $N\times N$ \idx{matrix}, where the $O$ symbol indicates \idxs{big O}{notation}\index{O!big O notation|see{big O notation}} and $N$ is the number of unknowns. However, an $O(N^3)$ time for solving the pressure Poisson equation would slow down the simulation tremendously, since it otherwise runs in $O(N)$ time per time step.

Since \incompressibility is only an \approximate \property of the fluid, it is arguably enough to only approximately solve the pressure Poisson equation, which is the governing equation for incompressibility. This assumption enables a large set of fast, iterative methods for solving the pressure equation, such as the \idxse{multilevel}{acceleration}{multilevel accelerated} \idxs{Jacobi}{method} \citep{Popinet2003}. However, there exist iterative methods that will solve the pressure Poisson equation down to \idxs{machine}{precision} in only a few number of iterations, such as the \PCG method, which can be applied if the matrix is \idxse{symmetric}{matrix}{symmetric}; this method has earlier been used with an \idxs{incomplete}{LU Cholesky factorization}\index{factorization!incomplete LU Cholesky|see{incomplete LU Cholesky factorization}}\index{Cholesky factorization!incomplete LU|see{incomplete LU Cholesky factorization}} as \preconditioner \citep{Losasso2004}.

If the pressure equation is only solved approximately, \eqref{eq:velocity_divergence_incompressible_flow} is not perfectly satisfied, and hence \eqref{eq:continuity_equation_superfluous} doesn't hold. If perfect \idxs{conservation of}{mass} is essential, the deviation of mass has to be recorded so that they can be accounted for, and \eqref{eq:continuity_equation_time_discretized} has to be used again.

Instead of trying to satisfy \eqref{eq:velocity_divergence_incompressible_flow}, which obviously doesn't work too well and would lead to uncontrolled fluctuations in density, one would rather prefer that the density very quickly becomes one again. The time $n+1$ density is given by \eqref{eq:continuity_equation_time_discretized} which is fully determined since $\rho_{n+1}$ is the only unknown in the equation, but the time $n+2$ density is given by substituting $n+1$ for $n$ in \eqref{eq:continuity_equation_time_discretized}, that is

\begin{equation} \label{eq:continuity_equation_time_discretized_postponed}
\rho_{n+2} \,=\, \rho_{n+1} - \Delta t\,\nabla\cdot(\rho_{n+1}\vec{u}_{n+1})
\end{equation}

which is underdetermined since $\vec{u}_{n+1}$ also is unknown. We can therefore make the requirement that

\begin{equation} \label{eq:density_conservation}
\rho_{n+2} \,=\, 1.
\end{equation}

Perfect satisfaction of \eqref{eq:density_conservation} will not take place due to an imperfectly solved pressure equation, but it is still the goal and also what we are going to assume. By rearranging \eqref{eq:continuity_equation_time_discretized_postponed} and expanding the divergence, we obtain

\begin{equation} \label{eq:velocity_divergence_density_conservation_premature}
\nabla\cdot\vec{u}_{n+1} \,=\, \frac{1-\rho_{n+1}^{-1}\,\rho_{n+2}}{\Delta t} \,-\, \rho_{n+1}^{-1}\,\vec{u}_{n+1}\cdot\nabla\rho_{n+1}.
\end{equation}

If we assume that the \idxs{Courant}{number} $C$, which is basically the volume fraction of a cell that is replaced during one \timestep because of flow through the cell walls, is much lower than one, that is

\begin{equation}
C \ll 1,
\end{equation}

or if we assume that $C$ is limited and the \spectrum of $\rho_{n+1}$ is dominated by frequencies much lower than the \idxs{Nyquist}{frequency}, $\rho_{n+1}^{-1}\,\vec{u}_{n+1}\cdot\nabla\rho_{n+1}$ will be a minor term in \eqref{eq:velocity_divergence_density_conservation_premature} and can therefore be \neglected. Furthermore, since $\rho_{n+1}$ is close to $1$, we can \approximate $\rho_{n+1}^{-1}$ as the first-order \idxs{Taylor}{polynomial} centered around the value $1$, which is $2-\rho_{n+1}$. Hence, by using \eqref{eq:density_conservation}, we can rewrite \eqref{eq:velocity_divergence_density_conservation_premature} as

\begin{equation} \label{eq:velocity_divergence_density_conservation}
\nabla\cdot\vec{u}_{n+1} \,=\, \frac{\rho_{n+1}-1}{\Delta t}.
\end{equation}

By rewriting \eqref{eq:velocity_update}, taking the divergence of both sides and using \eqref{eq:velocity_divergence_density_conservation} to get rid of $\nabla\cdot\vec{u}$, we obtain the \idxs{Poisson}{equation}

\begin{equation} \label{eq:pressure_poissin_equation_density_conservation}
\nabla^2 p_{n+1} \,=\, \frac{\nabla\cdot\vec{u}^*_n}{\Delta t} - \frac{\rho_{n+1} - 1}{\Delta t^2}.
\end{equation}

If, on the other hand the pressure equation is solved to a very high accuracy, or perfect \idxs{conservation of}{mass} is not something very important, \eqref{eq:pressure_poissin_equation_incompressible_flow} can equally well be used and then the \idxs{density}{field} becomes superfluous. For \simulation purposes, conservation of mass to this high degree is not important and hence the density field can be omitted. No matter if we need to conserve mass perfectly or not, we will obtain a \idxs{Poisson}{equation} for the \idxs{pressure}{field}, which can be written on the form

\begin{equation} \label{eq:pressure_poissin_equation_general}
\nabla^2 p \,=\, q(\vec{r}),
\end{equation}

where $q$ is a known function of $\vec{r}$.

\subsection{Solution of the pressure Poison equation}

\label{sec:pressure_poison_equation_solution}

There are a number of ways to solve the pressure Poisson equation. To begin with, we can realize that since the system is discretized, and since the Poisson equation is linear by nature, the equation can be written as a system of linear equations, that is

\begin{equation} \label{eq:pressure_poissin_equation_matrix}
\mathbf{A\,p} \,=\, \mathbf{q},
\end{equation}

where $\mathbf{p}$ is the vector containing the pressure in all cells, $\mathbf{q}$ is the vector containing $q(\vec{r})$ for all cells, and $\mathbf{A}$ is an $N\times N$ \idx{matrix} containing the coefficients for the system of linear equations.

First, there is a class of methods called \idxsp{direct}{method}{s}, which will solve the system of linear equations completely. An example is the naive \idxs{Gaussian}{elimination} algorithm, or the \idxs{Gauss--Jordan}{elimination} algorithm for a \idx{multi-core} system, having a \idxs{time}{complexity} of $O(N^3)$ operations in total for a system of $N$ unknowns, which is (as already mentioned) unacceptably slow.

Then there is a class of methods called \idxsp{iterative}{method}{s}, which start with a guessed value, $\mathbf{p}_0$, and then generate a sequence, $\{\mathbf{p}^{(k)}\}$, of improving \approximate solutions, where $k$ is the number of the iterations carried out. Different iterative methods converge differently quickly to the real solution, $\mathbf{p}$.

\subsubsection{The Preconditioned Conjugate Gradient Method}

See also \textit{Incomplete Cholesky Preconditioned Conjugate Gradients method}, described in \textit{\href{http://www.cs.ubc.ca/~rbridson/fluidbook/}{Fluid Simulation for Computer Graphics}}. This method uses the \textit{\href{http://en.wikipedia.org/wiki/Incomplete_Cholesky_factorization}{incomplete Cholesky factorization}} as preconditioner.

One iterative method is the \PCG method, which requires the matrix formulation of the system to be symmetric and positive-definite. It can work both as a direct method if run for $N$ iterations, or as an iterative method if the iterations stop earlier, but will often yield very fast conversion. In a work from \citeyear{Losasso2004} \citep{Losasso2004}, this method was noted to require only about 20 iterations to converge to an \accuracy of \idxs{machine}{precision} and that the pressure solver that was used only accounted for \mbox{25 \%} of the simulation time. However, it also noted that if the equation formulation is nonsymetric, it requires nonoptimal preconditioners which easily leads to an order of magnitude slowdown, and in the worst case even problems with robustly finding a solution at all.
% The \textit{Incomplete Cholesky Preconditioned Conjugate Gradients method} is described in \textit{\href{http://www.cs.ubc.ca/~rbridson/fluidbook/}{Fluid Simulation for Computer Graphics}}.

\subsubsection{The Jacobi Method and the Gauss--Seidel Method}

Another iterative method is the \idxs{Jacobi}{method}, where in each iteration, each equation in the system is solved independently of all other equations, by isolating the unknown central to the equation and by replacing the other unknowns with the values obtained for them from the previous iteration.

If $\mathbf{A}$ is decomposed into a diagonal component $\mathbf{D}$ and the reminder $\mathbf{R}$, such that $\mathbf{A} = \mathbf{D} + \mathbf{R}$, this method can be expressed as

\begin{equation} \label{eq:jacobi_method}
\mathbf{p}^{(k+1)} \,=\, \mathbf{D}^{-1}(\mathbf{q} - \mathbf{R}\mathbf{p}^{(k)}).
\end{equation}

By subtracting $\mathbf{p}$ from both sides, we get

\begin{equation}
\begin{array}{c}
\mathbf{p}^{(k+1)} - \mathbf{p} \\
=\, \mathbf{D}^{-1}(\mathbf{q} - \mathbf{D}\,\mathbf{p} - \mathbf{R}\mathbf{p}^{(k)}) \,=\, \mathbf{D}^{-1}(\mathbf{q} - (\mathbf{D} + \mathbf{R})\,\mathbf{p} - \mathbf{R}(\mathbf{p}^{(k)} - \mathbf{p})) \\
=\, \mathbf{D}^{-1}(\mathbf{q} - \mathbf{A}\,\mathbf{p} - \mathbf{R}(\mathbf{p}^{(k)} - \mathbf{p})) \,=\, -\mathbf{D}^{-1}\mathbf{R}(\mathbf{p}^{(k)} - \mathbf{p}).
\end{array}
\end{equation}

If we define the error of the pressure in each iteration as

\begin{equation} \label{eq:pressure_error}
\mathbf{\epsilon}^{(k)} \,=\, \mathbf{p}^{(k)} - \mathbf{p},
\end{equation}

we see that

\begin{equation} \label{eq:jacobi_method_error}
\mathbf{\epsilon}^{(k+1)} \,=\, -\mathbf{D}^{-1}\mathbf{R}\mathbf{\epsilon}^{(k)},
\end{equation}

which tells us that the error will decrease exponentially to the number of iterations, and that the error is composed of different \eigenvectors that decrease with different speeds. In fact, the eigenvector will represent different wave shapes, and can be assigned approximate \wavelengths in number of cells, where on a three-dimensional grid, wavelengths around 2 cell sizes will be those that decreases the quickest in strength.

However, eigenvectors with shorter wavelength than so will \overshoot the \idxs{zero}{vector} and oscillate around it a few times before they converge to the desired accuracy. The shorter the wavelength the more the eigenvector will overshoot, and the eigenvector with the shortest corresponding wavelength will only converge extremely slowly or, if the wavelength is $2/\sqrt{d}$, overshoot so much that it just changes sign, and \oscillate forth and back between two vectors and never converge.

Besides, wavelengths much longer than the optimal will decrease in strength much slower. The number of iterations needed to for an eigenvector to reduce in strength with a certain amount is limited by $O(l^2)$, where $l$ is the wavelength of the eigenvector in number of cells, which means that this method quickly becomes extremely slow to use for a grid that consists of more than say 10 or 20 cells across, which most grids do.

On the other hand, both of these two issues with the Jacobi Method have remedies. The first one can be solved by switching from the Jacobi method to the \idxs{Gauss--Seidel}{method}, which is basically the same as the Jacobi method with the only difference that the unknowns that are replaced are replaced with the latest values obtained for them, which are either from the previous iteration or from the current iteration.

Another technique that can help leviate both issues to some extent is \SOR, if used in an optimal way. It can resolve the first of the two issues, but can only make the second of them a bit less painful.

Since we can't resolve the other problem completely, we need another way to tackle it.

\subsubsection{The Multigrid Method}

We have concluded that short wavelengths in the Jacobi method or the Gauss--Seidel method is not a problem. But wavelengths that are long compared to the grid size is, which we need to get around somehow. Can we somehow overcome the $O(l^2)$ requirement for the number of iterations for long wavelengths? Unfortunately not. Can we decrease $l$ somehow? Well, $l$ is the wavelength measured in number of cells, or in other words, the wavelength in unit lengths divided by the cell size (also in unit lengths). We cannot decrease the wavelength measured in unit lengths. However, it is possible to increase the cell size, simply by making the grid coarser.

In the \idxs{multigrid}{method}, the grid is continuously coarsened until a desired grid resolution, usually consisting of just a few cells across, is obtained. When it is applied to solving the pressure Poison equation, the pressure field is downsampled into a new discretization each time the grid is coarsened, and a new set of linear equations is created for that discretization. Since the cell size increases each time the grid is coarsened, the convergence of long wavelengths in the pressure field is sped up.

As already mentioned, the grid is continuously coarsened and the pressure field is downsampled until a desired resolution is obtained. Now, starting from the coarsest grid, a few iterations with the Gauss--Seidel method or of \SOR are carried out in order to find an \approximate solution of the pressure equation. The difference between the discretized pressure field stored on that before and after the iterations is upsampled and added to the discretized pressure field stored on the second coarsest grid. A few iterations of the same iterative method are carried out and the difference in the discretized pressure field stored on that grid is upsampled and added to the third coarsest grid, and so on until the finest grid has been reached. Finally, a few iterations of the iterative method used on the other grids are carried out in order to approximately solve the pressure equation even on the finest level.

Because upsampling needs some way to interpolate the discretized pressure field, and often has a slight \idxsp{low-pass}{filter}{ing} effect simply because it tends to smear it out, additional errors are introduced in the pressure field. Hence, the entire process of coarsening and refining the grid usually has to be repeated a number of times before a solution of the desired accuracy can be reached.

Note that although a single grid is unable to quickly make long wavelengths in the pressure field converge with an acceptable speed, the multigrid method makes sure those wavelengths are accounted for in the coarser grids. Besides, thanks to the exponentially decreasing number of grid points in each grid, the total number of grid points that have to be processed can be approximated as a \idxs{geometric}{series} containing the number of grid points in the finest grid, $N$. Considering also that the multigrid method will reduce errors by a constant factor each time the process is repeated, the time complexity of the multigrid method for solving the pressure equation is $O(m\,N)$, where $m$ is the difference between the number of \idxsp{significant}{figure}{s} before and after the multigrid method was applied. It is because of the existence of $m$ in this expression \eqref{eq:pressure_poissin_equation_density_conservation} was developed as a replacement for \eqref{eq:pressure_poissin_equation_incompressible_flow}, in order to make the simulation less sensitive to fluctuations in the divergence of the \idxs{velocity}{field} and hence allow the pressure equation to be solved only \approximately.

Finally, although the multigrid method can be applied on an octree \citep{Popinet2003,Ji2012}, the implementation becomes slightly more complicated than on a regular grid. To ensure optimal performance in the general case, a special method that coarsens on multiple levels simultaneously has to be used \citep{Popinet2003}.

%\subsection{Semicompressible water}

%\section{Boundary conditions}

\chapter{Octrees}

An \octree is a tree \idxs{data}{structure} in which each \idxs{internal}{node} has exactly eight \idxse{child}{node}{children}. Octrees are often used to represent \idxs{three-dimensional}{data}, for which each node corresponds to a \cube in \idxs{three-dimensional}{space}, and where each of those cubes that corresponds to a \idxs{parent}{node} is subdivided into eight smaller cubes with half the side, corresponding to the children of the node. Octrees are especially usefull when the data that needs to be repersented has different requirements for the \LOD in different parts of space. They are a variant of \quadtrees in which each internal node has four children and often are used to represent \idxs{two-dimensional}{data}.

In this report, an octree is going to be considered to be a spatial data structure, and so in the text it will be used interchangeably with the cube that corresponds to the \idxs{root}{node} of the octree.

\begin{figure}
    \centering
    \subcaptionbox{\label{fig:quadtree}}[.415\textwidth]{
        \begin{tikzpicture}[x={(.35\textwidth,0)},y={(0,.35\textwidth)}]
            % Front side
            \draw (0,0,1) \threedimsquarepath{1} {1}{0}{0} {0}{1}{0};
            % Level 1
            \drawthreedimplus{0}{0}{1} {1} {1}{0}{0} {0}{1}{0}
            % Level 2
            \drawthreedimplus{1/2}{1/2}{1} {1/2} {1}{0}{0} {0}{1}{0}
            \drawthreedimplus{1/2}{0/2}{1} {1/2} {1}{0}{0} {0}{1}{0}
            \drawthreedimplus{0/2}{1/2}{1} {1/2} {1}{0}{0} {0}{1}{0}
            % Level 3
            \drawthreedimplus{2/4}{3/4}{1} {1/4} {1}{0}{0} {0}{1}{0}
            \drawthreedimplus{3/4}{3/4}{1} {1/4} {1}{0}{0} {0}{1}{0}
            \drawthreedimplus{2/4}{1/4}{1} {1/4} {1}{0}{0} {0}{1}{0}
            % Level 4
            \drawthreedimplus{5/8}{7/8}{1} {1/8} {1}{0}{0} {0}{1}{0}
            \drawthreedimplus{6/8}{7/8}{1} {1/8} {1}{0}{0} {0}{1}{0}
            \drawthreedimplus{5/8}{6/8}{1} {1/8} {1}{0}{0} {0}{1}{0}
            \drawthreedimplus{6/8}{6/8}{1} {1/8} {1}{0}{0} {0}{1}{0}
            % Level 5
            \drawthreedimplus{12/16}{14/16}{1} {1/16} {1}{0}{0} {0}{1}{0}
            \drawthreedimplus{11/16}{14/16}{1} {1/16} {1}{0}{0} {0}{1}{0}
            \drawthreedimplus{11/16}{13/16}{1} {1/16} {1}{0}{0} {0}{1}{0}
            \drawthreedimplus{10/16}{13/16}{1} {1/16} {1}{0}{0} {0}{1}{0}
        \end{tikzpicture}
    }
    \subcaptionbox{\label{fig:octree}}[.575\textwidth]{
        \begin{tikzpicture}[x={(.35\textwidth,0)},y={(0,.35\textwidth)},z={(-.385*.35\textwidth,-.385*.35\textwidth)}]
            % Front side
            \draw[fill=white!100!black] (0,0,1) \threedimsquarepath{1} {1}{0}{0} {0}{1}{0};
            % Level 1
            \drawthreedimplus{0}{0}{1} {1} {1}{0}{0} {0}{1}{0}
            % Level 2
            \drawthreedimplus{1/2}{1/2}{1} {1/2} {1}{0}{0} {0}{1}{0}
            \drawthreedimplus{1/2}{0/2}{1} {1/2} {1}{0}{0} {0}{1}{0}
            \drawthreedimplus{0/2}{1/2}{1} {1/2} {1}{0}{0} {0}{1}{0}
            % Level 3
            \drawthreedimplus{2/4}{3/4}{1} {1/4} {1}{0}{0} {0}{1}{0}
            \drawthreedimplus{3/4}{3/4}{1} {1/4} {1}{0}{0} {0}{1}{0}
            \drawthreedimplus{2/4}{1/4}{1} {1/4} {1}{0}{0} {0}{1}{0}
            % Level 4
            \drawthreedimplus{5/8}{7/8}{1} {1/8} {1}{0}{0} {0}{1}{0}
            \drawthreedimplus{6/8}{7/8}{1} {1/8} {1}{0}{0} {0}{1}{0}
            \drawthreedimplus{5/8}{6/8}{1} {1/8} {1}{0}{0} {0}{1}{0}
            \drawthreedimplus{6/8}{6/8}{1} {1/8} {1}{0}{0} {0}{1}{0}
            % Level 5
            \drawthreedimplus{12/16}{14/16}{1} {1/16} {1}{0}{0} {0}{1}{0}
            \drawthreedimplus{11/16}{14/16}{1} {1/16} {1}{0}{0} {0}{1}{0}
            \drawthreedimplus{11/16}{13/16}{1} {1/16} {1}{0}{0} {0}{1}{0}
            \drawthreedimplus{10/16}{13/16}{1} {1/16} {1}{0}{0} {0}{1}{0}
            
            % Top side
            \draw[fill=white!98!black] (0,1,0) \threedimsquarepath{1} {1}{0}{0} {0}{0}{1};
            % Level 1
            \drawthreedimplus{0}{1}{0} {1} {1}{0}{0} {0}{0}{1}
            % Level 2
            \drawthreedimplus{0/2}{1}{1/2} {1/2} {1}{0}{0} {0}{0}{1}
            \drawthreedimplus{1/2}{1}{1/2} {1/2} {1}{0}{0} {0}{0}{1}
            \drawthreedimplus{1/2}{1}{0/2} {1/2} {1}{0}{0} {0}{0}{1}
            % Level 3
            \drawthreedimplus{2/4}{1}{3/4} {1/4} {1}{0}{0} {0}{0}{1}
            \drawthreedimplus{3/4}{1}{3/4} {1/4} {1}{0}{0} {0}{0}{1}
            \drawthreedimplus{2/4}{1}{2/4} {1/4} {1}{0}{0} {0}{0}{1}
            % Level 4
            \drawthreedimplus{5/8}{1}{7/8} {1/8} {1}{0}{0} {0}{0}{1}
            \drawthreedimplus{6/8}{1}{7/8} {1/8} {1}{0}{0} {0}{0}{1}
            \drawthreedimplus{5/8}{1}{5/8} {1/8} {1}{0}{0} {0}{0}{1}
            
            % Right side
            \draw[fill=white!90!black] (1,0,0) \threedimsquarepath{1} {0}{1}{0} {0}{0}{1};
            % Level 1
            \drawthreedimplus{1}{0}{0} {1} {0}{1}{0} {0}{0}{1}
            % Level 2
            \drawthreedimplus{1}{1/2}{1/2} {1/2} {0}{1}{0} {0}{0}{1}
            \drawthreedimplus{1}{0/2}{1/2} {1/2} {0}{1}{0} {0}{0}{1}
            \drawthreedimplus{1}{1/2}{0/2} {1/2} {0}{1}{0} {0}{0}{1}
            \drawthreedimplus{1}{0/2}{0/2} {1/2} {0}{1}{0} {0}{0}{1}
            % Level 3
            \drawthreedimplus{1}{3/4}{3/4} {1/4} {0}{1}{0} {0}{0}{1}
            \drawthreedimplus{1}{2/4}{2/4} {1/4} {0}{1}{0} {0}{0}{1}
            \drawthreedimplus{1}{2/4}{1/4} {1/4} {0}{1}{0} {0}{0}{1}
            \drawthreedimplus{1}{1/4}{2/4} {1/4} {0}{1}{0} {0}{0}{1}
            \drawthreedimplus{1}{1/4}{1/4} {1/4} {0}{1}{0} {0}{0}{1}
            % Level 4
            \drawthreedimplus{1}{3/8}{3/8} {1/8} {0}{1}{0} {0}{0}{1}
            \drawthreedimplus{1}{4/8}{3/8} {1/8} {0}{1}{0} {0}{0}{1}
            \drawthreedimplus{1}{4/8}{4/8} {1/8} {0}{1}{0} {0}{0}{1}
        \end{tikzpicture}
    }
    \caption{\subrefp{fig:quadtree} A \quadtree used to partition twodimensional space. \subrefp{fig:octree} An \octree used to partition three-dimensional space. Note that an octree is essentially the same as a quadtree but is extended from two to three dimensions. In this illustration, only half of the surface of the octree is visible.}
    \label{fig:quadtree_and_octree}
\end{figure}

\section{Varying level of detail}

In \thisprojectwork, an octree has been used to partition the \idxs{computational}{domain} (the space in which the simulation will be performed) into the \cells required by the \FVM. Since only the \surface of the water is visible in the \simulation, the \idxsp{surface}{cell}{s} have been given a high \LOD; the \LOD then decreases at the water depth increases, staying a few \idxse{layer of}{cells}{layers of cells} a time on each \LOD, forming a \idxs{LOD}{layer} (a layer consisting of cells which all have the same \LOD, surrounded by cells with other \LODs).

Ideally, although not implemented in \thisprojectwork because of \itslimitedtime, the surface will have a higher \LOD where the \idxsp{surface}{detail}{s} are more important to the simulation, such as close to the \camera where they are more easily seen, and a lower \LOD far away from the camera where the cells take up little \idxse{screen}{space}{space on the screen}, or where they are out of the \FOV.

The total number of cells $N_{\text{t}}$ used in the simulationcan can be \approximated by an expression containing the thicknesses of the different LOD layers and the number of cells visible at the surface that belong to each \LOD accoring to

\begin{equation} \label{eq:number_of_cells_total_sum}
N_{\text{t}} \,=\, \sum_{i\,=\,0}^\infty N_{\text{s},i}\sum_{j\,=\,i}^\infty d_j\cdot 4^{-(j-i)},
\end{equation}

where $N_{\text{s},i}$ is the number of cells visible on the surface belonging to \idxs{LOD}{layer} $i$ (a lod layer is the set of all cells with a distinct given \LOD) and $d_j$ is the thickness in number of cells of LOD layer $j$, where LOD layer $0$ corresponds to the highest \LOD and an increasing LOD layer index corresponds to a lower \LOD. Different LOD layers can have different thickness; if this is the case, it will also be reflected in the simulation and waves with different wavelengths will be simulated with different \accuracy. In \thisprojectwork, all LOD layers have been given the same thickness, $d$. \eqref{eq:number_of_cells_total_sum} therefore turns into

\begin{equation} \label{eq:number_of_cells_total}
N_{\text{t}} \,=\, d\sum_{i\,=\,0}^\infty N_{\text{s},i}\sum_{k\,=\,0}^\infty 4^{-k} \,=\, d\,N_{\text{s}}\,\frac{1}{1-4^{-1}} \,=\, \frac{4}{3}\,d\,N_{\text{s}},
\end{equation}

where $k = j-i$ and $N_{\text{s}}$ is the total number of cells visible on the surface. We can therefore in this case conclude that

\begin{equation} \label{eq:number_of_cells_total_ordo}
N_{\text{t}} \,=\, O(N_{\text{s}}),
\end{equation}

where the $O$ symbol indicates \idxs{big O}{notation}\index{O!big O notation|see{big O notation}}. Hence, the \idx{time step} for updating the \idxs{fluid}{flow} is roughly porportional to the number of \idxsp{surface}{cell}{s} $N_{\text{s}}$, but the simulation still catches all motion under the surface, with decaying \accuracy at increasing \idxsp{water}{depth}{s}.

\section{The differentiating problem}

\subsection{Perturbed cell interfaces method}

\subsection{Distributed velocities method}

\section{Multilevel acceleration}

\section{Spurious wave reflections at level transitions}

\subsection{Free-Surface Modeling (FSM)}
\begin{frame}[<+(1)->]
\frametitle{Free-Surface Modeling (FSM)}

\begin{itemize}
\item Volume Of Fluid (VOF) method
\item Modifierad Hyper-C (en sorst advection scheme)
\end{itemize}

\end{frame}

\subsection{Resultat}

\newlength{\screenshotwidth}
\setlength{\screenshotwidth}{.23\textwidth}

\begin{frame}
\frametitle{Resultat -- Screenshots}

\begin{figure}
\centering
\only<+>{
\includegraphics[width=\screenshotwidth]{../Presentation/Images/Screenshots/alpha/Report-figure2}\ \ 
\includegraphics[width=\screenshotwidth]{../Presentation/Images/Screenshots/alpha/Report-figure3}\ \ 
\includegraphics[width=\screenshotwidth]{../Presentation/Images/Screenshots/alpha/Report-figure4}\ \ 
\includegraphics[width=\screenshotwidth]{../Presentation/Images/Screenshots/alpha/Report-figure5}
\vskip .5em
\includegraphics[width=\screenshotwidth]{../Presentation/Images/Screenshots/alpha/Report-figure6}\ \ 
\includegraphics[width=\screenshotwidth]{../Presentation/Images/Screenshots/alpha/Report-figure7}\ \ 
\includegraphics[width=\screenshotwidth]{../Presentation/Images/Screenshots/alpha/Report-figure8}\ \ 
\includegraphics[width=\screenshotwidth]{../Presentation/Images/Screenshots/alpha/Report-figure9}
}
\only<+>{
\includegraphics[width=\screenshotwidth]{../Presentation/Images/Screenshots/water_coefficient/Report-figure10}\ \ 
\includegraphics[width=\screenshotwidth]{../Presentation/Images/Screenshots/water_coefficient/Report-figure11}\ \ 
\includegraphics[width=\screenshotwidth]{../Presentation/Images/Screenshots/water_coefficient/Report-figure12}\ \ 
\includegraphics[width=\screenshotwidth]{../Presentation/Images/Screenshots/water_coefficient/Report-figure13}
\vskip .5em
\includegraphics[width=\screenshotwidth]{../Presentation/Images/Screenshots/water_coefficient/Report-figure14}\ \ 
\includegraphics[width=\screenshotwidth]{../Presentation/Images/Screenshots/water_coefficient/Report-figure15}\ \ 
\includegraphics[width=\screenshotwidth]{../Presentation/Images/Screenshots/water_coefficient/Report-figure16}\ \ 
\includegraphics[width=\screenshotwidth]{../Presentation/Images/Screenshots/water_coefficient/Report-figure17}
}
\only<+>{
\includegraphics[width=\screenshotwidth]{../Presentation/Images/Screenshots/air_coefficient/Report-figure18}\ \ 
\includegraphics[width=\screenshotwidth]{../Presentation/Images/Screenshots/air_coefficient/Report-figure19}\ \ 
\includegraphics[width=\screenshotwidth]{../Presentation/Images/Screenshots/air_coefficient/Report-figure20}\ \ 
\includegraphics[width=\screenshotwidth]{../Presentation/Images/Screenshots/air_coefficient/Report-figure21}
\vskip .5em
\includegraphics[width=\screenshotwidth]{../Presentation/Images/Screenshots/air_coefficient/Report-figure22}\ \ 
\includegraphics[width=\screenshotwidth]{../Presentation/Images/Screenshots/air_coefficient/Report-figure23}\ \ 
\includegraphics[width=\screenshotwidth]{../Presentation/Images/Screenshots/air_coefficient/Report-figure24}\ \ 
\includegraphics[width=\screenshotwidth]{../Presentation/Images/Screenshots/air_coefficient/Report-figure25}
}
\addtocounter{beamerpauses}{-3}
\only<+>{\caption{$\alpha$-value}}
\only<+>{\caption{Water coefficient}}
\only<+>{\caption{Air coefficient}}
\end{figure}

\end{frame}

\begin{frame}[<+(1)->]
\frametitle{Resultat -- Fortsättning}

\begin{itemize}
\item För låg hastighet
\item Mycket funktionalitet saknas fortfarande
\item Generisk flödes-lösare ($n$-dimensionella flöden)
\item $O(n)$
\end{itemize}

\end{frame}


\section{Theory}

\section{Analysis}

%\chapter{Improvements}

There are a lot of improvements of the method that was implemented in \thisprojectwork that can be performed, many of them which are necessary  to make the method be practically useful in a real-time flight simulator.

\section{Required improvements}

In this section, some of the improvements that are required to make the method practically useful are presented.

\subsection{Adaptive Mesh Refinement}

The \LOD needs to be adapted to the \idxs{camera}{placement} and the \FOV/\idxs{viewing}{frustum} as well as the water depth in order to greatly reduce the number of cells used by the \FVM in the simulation, in order to only use the minimum amount of cells that are actually needed. This is referred to as \AMR.

Without \AMR it would be impossible to simulate larges bodies of water and fine surface details real-time in the same scene, as these would require heaps of orders of magnitude more cells to still be able to simulate waves with the same short \wavelength as with \AMR.

\subsubsection{No simulation of air} In \thisprojectwork, not just the water was simulated, but also the air, which meant that the simulation domain needed to have a boundary in the air. This introduced the additional problem of choosing where the boundary should be located, and what boundary conditions it should have. Additionally, this meant extra computational work since more cells had to be simulated which reduced the simulation speed significantly, and it introduced the phenomenon that velocities could suddenly go to infinity, which froze the simulation.

A better solution would be to exclude the air region and treat that as vacuum, and move the boundary to the interface between the water and the vacuum/air region.

\subsection{Unconditionally stable flows}

In order to make a \FVM simulation become practically useful for \idxsp{real-time}{simulation}{s}, it is required that the flow is \idxse{unconditionally stable}{flow}{unconditionally stable}\index{stable flow!unconditionally|see{unconditionally stable flows}}. This implies that it is necessary that arbitrarily large \timesteps can be taken without making the simulation unstable.

When it comes to physical quantities with a non-damped \oscillating behavior, simulations which use any kind of normal, \idxs{explicit}{time-stepping}\index{stepping!in time|see{time-stepping}} often tend to make them become unstable and create a \idxs{numerical}{explosion}.

In \thisprojectwork, a kind of \idxs{leapfrog}{integration} has been used to model the time evolution of the velocity and the pressure, but even \idxs{leapfrog}{integration} has to obey the \CFL condition, and it becomes numerically unstable when the \period of the oscillations becomes too small in relation to the \timestep. To solve this problem, many methods \approximate the fluid or the fluids as \incompressible, whilst other use \idxse{implicit}{Euler integration}{backwards (or implicit) Euler integration}\index{integration!implicit Euler|see{implicit Euler integration}} (see 
\secref{sec:pressure_calculation} for a more detailed discussion about this).

On the other hand, when it comes to advection, many advection schemes tend to become numerically unstable when the \idxs{Courant}{number} $C$ becomes too large (this normally means larger than one). Whilst most \idxsp{explicit}{Runge--Kutta method}{s}\index{method!explicit Runge--Kutta|see{explicit Runge--Kutta method}} for advecting any form of field tend to damp high frequent features in the field for $C < 1$, they do the opposite for $C > 1$ and instead enhance the high frequent features and become unstable. If a \idxs{backwards}{Runge--Kutta method}\index{method!backwards Runge--Kutta|see{backwards Runge--Kutta method}} is used instead, in an attempt to remedy this instability, the behaviors will be the reversed and high frequent features will be damped for $C > 1$, hence making the simulation stable in those regions, whilst they will be enhanced for $C < 1$, hence still making the simulation unstable.

A remedy for this dilemma is to use semi-Lagrangian advection, which works by calculating which locations will end up on the \idxsp{discretization}{point}{s} in the end of the time step, and assigning interpolations of the field in those locations to the discretization points in the end of the time step.

All of these instability issues can be remedied \citep{Stam1999}.

\subsection{Rendering}

In \thisprojectwork, the water surface is not rendered. To render the surface properly, an isosurface needs to be extracted from the \idxs{phase fraction}{data}.

\subsubsection{Marching Cubes}

In order to find where the \idxs{water}{surface} is located, the \idxs{marching}{cubes} \idxe{algorithm!marching cubes|see{marching cubes}}{algorithm} can be used. Marching cubes is an algorithm for finding \isosurfaces from a \idxs{scalar}{field}, and works by looping over a set of adjacent cubes tiled closely together which just covers the domain that is about to be rendered. For each cube, the algorithm looks at the value of the scalar field in all eight of the cube's corners. If the field values in the corners all lie at the same side of the value $c$ that the isosurface represents, it figures that the surface doesn't cross through that cube, otherwise it does and the algorithm adapts a surface consisting of a finite number of polygons through the cube depending on what the field values in the corners are.

In \thisprojectwork, the scalar field that tells where the surface is located is the \idxs{phase fraction} {field}, $\alpha$, and since $\alpha$ crosses from 0 to 1 over the \interface, the isosurface can really bee chosen for any level $0 < c < 1$ and the marching cubes algorithm will produce an isosurface lying within the interface, although it probably makes more sense to choose $c = 0.5$ which lies in the middle of the interval and makes the isosurface generation symmetric in that sense. One problem though is that the $\alpha$ field isn't discretized in the cell corners, but in the cell centers. This can be remedied by defining a new set of cubes which have their corners in the centers of the cells. This will become a bit problematic in \idxsp{LOD}{transition}{s}, for which the cubes if created in this way will stretch over two different LODs and thus not have a very well defined size. Another possible solution is to interpolate the $\alpha$ field to the cell corners so that the cells can be used directly as the cubes needed by the algorithm.

If the \VOF method for tracing the surface would be switched to using the \LS method instead, a signed distance function, $\phi$, would be used instead of $\alpha$, and $c = 0$ would define where the surface is located. The discretization points for $\phi$ would be in the cell corners, so there would be no need for defining a new set of cubes in order for the marching cubes algorithm to work. However, the marching cubes algorithm would have apparent problems to generate the surface in \idxsp{LOD}{transition}{s}, since, in it's original form, requires that all cubes are of the same size, otherwise glitches may occur.

The problem of generating the isosurface with the marching cubes algorithm when the domain that is to be rendered contains LOD transitions is well known, and remedies exist. For example, the one in \textit{\href{http://www.terathon.com/lengyel/Lengyel-VoxelTerrain.pdf}{Voxel-Based Terrain for Real-Time Virtual Simulations} + \href{http://www.terathon.com/voxels/}{The Transvoxel$^{\textbf{TM}}$ Algorithm}} (Lengyel, Eric. "Voxel-Based Terrain for Real-Time Virtual Simulations". PhD diss., University of California at Davis, 2010.) \citep{temp} remedies this, but only for cases where the cubes used by the algorithm are modeled as an \octree.

\subsection{Parallelization}

In order to obtain a simulation speed-up that is almost certainly needed, it can be processed on multiple \CPUs simultaneously, as well as on the \GPU (if the \octree traversal is implemented to run without recursion, since \GPUs generally don't have any stack which means that they can't do recursion). This is generally referred to as \parallelization. For parallelization to be possible, the \idxs{simulation}{domain} first has to be \partitioned, so that a small part of the domain can be processed on each core and on each \idxs{GPU}{process}.

Since the process that takes the longest time to run will decide the speed of the simulation, the size of the partitions should be roughly the same. Besides, access to data that is located outside of the partition takes extra long time, so it is desired that the number of neighbors cells for each partition that lies outside of the partition itself is kept low, which is somewhat similar to the problem of minimizing the area between the partitions. For a structured grid, the partitioning can be effectively achieved by cutting the simulation domain into square or cubic pieces. However, for an \idxs{unstructured}{grid} like an \octree, the partitioning is more difficult, but still possible.

For octrees, there are some very convenient algorithms that can be used which make use of \idxsp{space-filling}{curve}{s} and which are fast, easy to implement and still produce very acceptable results (see e.g. \textit{\href{http://j.teresco.org/research/publications/octpart02/octpart02.pdf}{Dynamic Octree Load Balancing Using Space-Filling Curves}} \citep{temp} and \textit{\href{http://downloads.isrn.com/journals/appmath/2012/246491.pdf}{Parallel Adaptive Mesh Refinement Combined with Additive Multigrid for the Efficient Solution of the Poisson Equation}} \citep{temp}). Another kind of partitioning is also performed in \textit{\href{http://gfs.sourceforge.net/papers/agbaglah2011.pdf}{Parallel simulation of multiphase flows using octree adaptivity and the volume-of-fluid method}} \citep{temp} which uses \idxs{domain}{decomposition}.

\subsection{Fluid--Structure Interaction}

In \thisprojectwork, no interaction between ships and the surface waves was ever modeled, in other words, no \FSI was modeled. Modeling and simulating \FSI is necessary in order to make large waves affect ships so that they start to rock forth and back, as well as in order for ships to give rise to two \idxsp{wake}{line}{s} as they \sail on the water.
    
Some of the methods for modeling \FSI includes the \IB method (reference \textit{\href{http://www4.ncsu.edu/~zhilin/TEACHING/MA798Z/Peskin1.pdf}{The immersed boundary method}} \citep{temp} or \textit{\href{http://www.cecs.wright.edu/~haibo.dong/wp-content/themes/publications/IBM_JCP_2007.pdf}{A sharp interface immersed boundary method for compressible viscous flows}} \citep{temp}), the \VOS method (reference \textit{The simulation of fluid-rigid body interaction} \citep{temp}, also described in \textit{Numerical Modeling of Deforming Bubble Transport Related to Cavitating Hydraulic Turbines} \citep{temp}) and the \idxs{rigid fluid}{method}\index{fluid!rigid|see{rigid fluid method}} (reference: \textit{\href{http://www.amath.unc.edu/Faculty/mucha/Reprints/siggraph04.pdf}{Rigid Fluid: Animating the Interplay Between Rigid Bodies and Fluid}} \citep{temp})
%See e.g.: \item \textit{\href{http://physbam.stanford.edu/~fedkiw/papers/stanford2010-04.pdf}{Numerically Stable Fluid-Structure Interactions Between Compressible Flow and Solid Structures}} \item \textit{\href{http://efdl.as.ntu.edu.tw/research/papers/JCP03GCIBM.pdf}{A ghost-cell immersed boundary method for flow in complex geometry}} \item \textit{\href{http://www.cs.columbia.edu/~batty/papers/Batty07.pdf}{A Fast Variational Framework for Accurate Solid-Fluid Coupling}} (solid fraction, non-stick to walls)

\subsubsection{Rotation of rigid bodies}

The \idxs{rotation of a}{rigid body}, in this case the rotation of a \ship that is floating on the surface of the water when it is hit by large waves, depends both on the forces that acts on the \idxs{rigid}{body} as well as well as the  \idxs{moment of}{inertia} of the rigid body.

The dynamics of a rigid body is described by \textit{\href{http://en.wikipedia.org/wiki/Euler\%27s_equations_\%28rigid_body_dynamics\%29}{Euler's equations (rigid body dynamics)}} \citep{temp}.

\section{Desired improvements}

In this \levelname, some improvements that are not required in order to make the method practically useful, but still desired in order to improve the quality of the simulation, are presented.

\subsection{Reduction of spurious reflections in LOD transitions}

When a wave hits a boundary to a region where it propagates with a different speed, or cannot propagate at all, a partial or total reflection of the wave will occur. In natural scenes, this typically happens when the depth of the water changes abruptly from deep water to shallow, or when the wave hits a cliff or similar.

In scenes simulated with a grid of finite size, this usually happens on the border of the grid, unless special care has been taken to this when deciding which \idxsp{boundary}{condition}{s} to use. These reflections are unwanted and referred to as \idxs{spurious}{reflections} since they don't occur in nature.

In scenes simulated on an \octree based \grid, this also happens in \LOD transitions, since different \idxsp{LOD}{level}{s} have different lowest wavelengths they can represent called the \idxs{Nyquist}{frequency} (which is double the cell size for rectangular cells), and the fact that a wavelength which can be represented on two different LOD levels often will propagate with different speeds on the two LOD levels, especially if the wavelength is close to the Nyquist frequency for one of the LOD levels but not close to the Nyquist frequency for the other LOD level.

On scenes simulated on an arbitrary grid though, the reflections corresponding to the reflections that takes place in LOD transitions are often less noticeable as the surface of reflection is less well-defined, which causes the waves to be reflected at multiple locations and become scattered in many directions, although the waves are still eventually totally reflected.

\subsubsection{Perfectly matched layers}

One method for greatly reducing spurious reflections at the border of a structured grid is to use \idxs{perfeclty matched}{layers}. See e.g. \textit{\href{http://liu.diva-portal.org/smash/get/diva2:359805/FULLTEXT01}{Memory Efficient Methods for Eulerian Free Surface Fluid Animation}} \citep{temp}, which explains explicit dampening, implicit dampening, and wave absorbing boundaries --- the perfectly matched layer approach, and evaluates the methods.

\subsubsection{Other absorbing boundary conditions}

Another method that works when the waves have a common speed is to use absorbing boundary conditions (see \citep{temp}). This method works ideally in the \idx{one-dimensional} case if all wavelengths have the same simulated speed, and performs decently well in the \idx{two-dimensional} or \idx{three-dimensional} case. See e.g. \textit{\href{http://www.ce.ncsu.edu/centers/cmg/AWWE/}{the AWWE equations}} \citep{temp}.
\subsection{Wind waves}

In order to obtain a realistic \idxs{sea}{state}\index{state of sea|see{sea state}} in a simulation, it is desirable that the \idxs{water}{surface} \interacts with the \wind in order to give rise to \idxsp{wind}{wave}{s}. In this way, the sea state will be a reflection of the \idxs{wind}{velocity}.

\subsubsection{Spectral methods}

A very cheap and easy way to simulate wind waves is to use a wind dependent \idxs{wave}{spectrum}. A wave spectrum tells how the size of waves with different wavelengths are \distributed, and from that distribution the \idxs{Fourier}{transform} of the free surface elevation can easily be \generated by taking a \random \sample from it. This is then used to \initialize the \idxs{sea}{state} in the simulation.

There are different models for describing the wave spectrum for a certain wind velocity, most of which are \empirical. Some of these spectra take into account the wind direction for increased alignment between the surface and the wind, for example the \idxs{Philips}{spectrum} which is used and tweaked in \textit{\href{http://graphics.ucsd.edu/courses/rendering/2005/jdewall/tessendorf.pdf}{Simulating Ocean Water}} \citep{temp}, or the \idxs{Pierson--Moskowitz}{spectrum}, which was developed in \darkblue{\textit{A Proposed Spectral Form for Fully Developed Wind Seas Based on the Similarity Theory of S.A. Kilaigorodskii}} \citep{temp} in 1964. Another commonly used wave spectrum which builds up over time is the \JONSWAP spectrum\index{spectrum!JONSWAP|see{JONSWAP}} (see \textit{\href{http://www.wikiwaves.org/Ocean-Wave\_Spectra\#JONSWAP\_Spectrum}{JONSWAP Spectrum}}) \citep{temp}.

\subsubsection{Air-water interaction}

Another (somewhat dubious) alternative would be to simulate both the \water and the \air using the \FVM. This method would be slower that using a wave spectrum since it has to simulate more cells, and would probably give a quite poor result since it is difficult to catch the correct interaction between the water and the air for the shortest \wavelengths in the simulation.

\subsection{Improved rendering}

In order to improve the rendering from what is just necessary (which would look almost plastic) to what is actually desired, a realistic \idxs{shading}{model} needs to be implemented.

\subsubsection{Light reflectivity and transmissivity}

When \light hits the \idxs{water}{surface}, not all light is transmitted into the water. Some of the light is reflected, and the amount of light that is reflected depends on the angle of incidence as well as the polarization of the light, according to the \idxs{Fresnel}{equations}.

Normally however, when rendering a \scene with \idxs{computer}{graphics}, one doesn't usually start from the light source when calculating the brightness of an object, but from the observer. On the other hand, according to the \idxs{Helmholtz}{reciprocity principle}\index{principle!Helmholtz reciprocity|see{Helmholtz reciprocity principle}}, one can use the Fresnel equations not only to calculate how much of the light is being reflected, but also the other way around to calculate how much of the light that comes from the surface originally hit the surface from above and was reflected, and how much of it hit the surface from below and was transmitted. In that way it is possible to account also for the light that has been reflected under the water, which has been filtered by the objects it has been reflected on under the surface of the water.

\subsubsection{Illumination model for water surfaces}

Typically, when rendering an \idxs{ocean}{scene}, not the entire wave spectrum is used but only the lowest frequencies of it, since too high frequencies will give rise to \folding which may in turn give rise to unwanted effects such as \idxsp{Moire}{pattern}{s}. Besides, because of the limited resolution of the \grid, short \wavelengths will automatically be removed even if the rendering would allow shorter wavelengths. Therefore, not all frequencies will be used in the \idxs{rendered}{image}. If much of the \idxs{wave}{energy} in a \idxs{sea}{state} lies in the part of the \idxs{wave}{spectrum} that will not be used when rendering the image, such as when the surface is viewed from a far distance, this can become very evident.

Since the \idxs{surface}{normal} of the water surface depends on the \gradient of the \idxs{free surface}{elevation}, all surface waves will help to build up a variance in the surface normal. If viewed from a distance that is just large enough, the surface waves themselves will be too small to be visible, but the variance of the normal that is created by the surface waves will be visible since it will give rise to a blur in \reflections in the surface. This is for example what often causes the reflection of the \sun in the sea to look like a long vertical stripe under the sun instead of something that looks like a sun. So if certain wavelengths are removed from the water surface, they should be accounted for by blurring out the image that is reflected in the water, of course in the correct way, or the rendered water will look too \smooth.

A derivation of an illumination model for sea surfaces is found in \appref{chap:illumination_model_derivation}.

\subsection{Visual effects}

\subsubsection{Choppy waves}

Waves of a single frequency is not a perfect \sinusoidal, but a \trochoid, which makes the curvature greater at the \idxs{wave}{crest} and lesser at the \idxs{wave}{valley}, which makes the waves look more \idxse{choppy}{waves}{choppy}. By letting sinusoidals displace the water surface, not only vertically, but also horizontally, trochoids can effectively be created and a more realistic look of the water surface is obtained. These waves are called \idxsp{Gerstner}{wave}{s} and were first found as an approximate solution to the fluid dynamic equations almost 200 years ago, as noted in \textit{\href{http://graphics.ucsd.edu/courses/rendering/2005/jdewall/tessendorf.pdf}{Simulating Ocean Water}}) \citep{temp}, and is today used extensively used in computer graphics for animating water surfaces. What is claimed to be an implementation of Gerstner waves can be seen in \citep{ceribral2012}.

\subsubsection{Splash and foam}

In rough weather conditions, the waves will become higher and choppier, and their crests will break more easily, forming white foam where they break, as can be seen in \figref{fig:sea_storm}, which is known as oceanic whitecaps, or just whitecaps. When high waves hit the side of a ship, they will also shatter and form a spray of small particles in the air, usually just referred to as splashes.

In \textit{\href{http://cg.informatik.uni-freiburg.de/publications/2012_CGI_sprayFoamBubbles.pdf}{Unified Spray, Foam and Bubbles for Particle-Based Fluids}} \citep{temp}, "diffuse particles" are generated based on the water's potential to trap air, its likelihood to be at the crest of a wave and its kinetic energy. In \textit{\href{http://nguyendangbinh.org/Proceedings/Eurographics/2003/cgf/volume22/issue3/paper127/paper127.pdf}{Realistic Animation of Fluid with Splash and Foam}} \citep{temp}, water is turned into "splash particles" where the curvature of the water exceeds a certain \threshold. The results of the use of diffuse/splash particles can bee seen in videos such as and \citep{RealFlowLabs2011,Chandel2009}.

In turn, \textit{\href{http://nguyendangbinh.org/Proceedings/Eurographics/2003/cgf/volume22/issue3/paper127/paper127.pdf}{Realistic Animation of Fluid with Splash and Foam}} \citep{temp} concludes that there seems to be "few papers on handling of effects of splashes and foam with fluid", even though it mentions the paper \textit{\href{https://subversion.assembla.com/svn/gpuocean/trunk/docs/rendering-natural-waters-00.pdf}{Rendering Natural Waters}} \citep{temp} as one of them, which it claims "makes crude approximations". However, this paper includes an empirical formula for the "fractional area of the wind-blown water surface that is covered by foam" $f$ (the time-averaged area of the foam/the total area of the water surface) that looks like

\begin{equation} \label{eq:fractional_area_naiive}
f = 1.59 \times 10^{-5}U^{2.55}\exp[0.0861(T_w - T_a)],
\end{equation}

where $U$ is the wind speed, $T_w$ is the water temperature and $T_a$ is the air temperature (this equation is in turn claimed to have been taken from the book \textit{\href{http://books.google.se/books?id=xuwFz1bPTHgC}{Oceanic Whitecaps: Their Role in Air-Sea Exchange Processes}} by E. C. Monahan and G. MacNiocaill \citep{temp}). This kind of information has been obtained by doing \idxsp{satellite}{measurement}{s}. However, according to the \idxs{Beaufort}{scale}, crests don't begin to break until there is at least a gentle breeze, which starts at about \mbox{3.4 m/s}, while according to \eqref{eq:fractional_area_naiive}, there would always be some whitecaps area. As a remedy, one could subtract a small value ${f_0 = f(U=U_0,\, T_w-T_a=\Delta T_0)}$ (where, tentatively, ${U_0 = 3.4\text{ m/s}}$ and ${\Delta T_0 = 0}$) and create a corrected estimate $f^*$ of the fractional area, defined as

\begin{equation}
f^* = max(f-f_0,\,0).
\end{equation}

Furthermore, tt suggests that "as a crude approximation to the true distribution, one can put whitecaps at positions on the surface where the amplitude of the waves is the largest". However, if following this method, the water surface would preferably first be high-pass filtered before determining on which parts of the surface the free surface elevation is the largest, to prevent that some (large) regions of the surface that would happen to be a bit higher elevated than other regions would get much more whitecaps area.

In \textit{\href{http://web1.see.asso.fr/ocoss2010/Session_4/20100531111216_Monnier_OCOSS2010-Paper_MERCUDA_item_2.pdf}{Real time modelling of multispectral ocean scenes}} \citep{temp}, it is on the other hand the vertical downward acceleration that is used to determine where to put the whitecaps, instead of the free surface elevation. The \threshold for the acceleration is calculated dynamically to keep the total whitecaps area at the correct level. 

As it turns out, the larger the Lagrangian vertical downward acceleration is, the smaller the pressure gradient is. Without a pressure gradient, there will be nothing that keeps the water and the air separated, and the two fluids are much more prone to mix. So where the downward acceleration is large, the pressure gradient will be small, and water and air will naturally be more prone to mix, especially under large wind speeds where the \idxs{velocity}{shear} at the surface is large, which increases the chance of getting \idxs{Kelvin–Helmholtz}{instability} in the system.

Additionally to using a \threshold for some specific property of the surface to determine where the whitecaps are located, some methods also use a \idx{half-life} for the whitecaps that have past to the other side of the \threshold to model a more realistic destruction process for the whitecaps, as for example can be seen in \citep{ozernik2009,cebasVT2010}.



\chapter{Conclusions}
\label{chap:conclusions}

The training in a flight simulator would become more valuable if a realistic, coupled simulation of water waves and ship movement was integrated into the flight simulator. It is therefore required that the wave simulation runs in real-time, has a resolution high enough to represent waves that normally roam the water surfaces and affect the ships as well as the wake lines that sailing ships give rise to. It should also manages to simulate wave dispersion, in at least an approximate way, as well as some form of \FSI to get the coupling between ship and waves.

The method that was implemented in \thisprojectwork was the \FVM on an octree with \FSM, which is an advanced method that takes a long time to implement and to make ideal for the purposes of \thismasterthesiswork \comment{this master thesis work}\nspace, and the implementation would still need a lot of improvement if it was to be applied in a real-time simulator.

The \FVM on an octree with \FSM runs in $O(N)$ time, where $N$ is the number of grid points visible on the surface, which places it among the methods with the lowest time complexity. However, it has a very high time constant associated with the big O notation which, as of today, makes it too slow to be used in real-time applications.

Probably the only one of the methods that were studied in \thismasterthesiswork, that today easily can be constructed to meet all the required properties, is the two-dimensional method using \LPD. Although Fourier synthesis often is used in real-time computer graphics applications for the creation of surface waves, the method is not suitable for modeling \FSI.

Alternatively, a two-dimensional simulation, using \LPD or Fourier synthesis, can be coupled with a local three-dimensional fluid simulation around each ship to allow for strong non-linear phenomenas such as splashes and more natural \FSI, as well as a fast simulation of \idxs{ambient}{waves}, proveded the three-dimensional method is fas enough.

However, it is believed that, after some improvement of the method that was implemented in \thisprojectwork, basically its only bottleneck would be its low speed, which to its defense can be alleviated by parallelizing the code and spread the computational load on many \CPUs and possibly even on \GPUs. And with the continuous increase in processor power it is only a matter of time before the method can run in real-time with an acceptable quality. Hence, it will in time come to compete with methods such as Fourier synthesis and the \LPD method, and will eventually even surpass them even for real-time simulations, as it models the behavior of water in a more realistic way which, intrinsically,  means that its potential of what it can achieve is higher.

\subsection{Belysningsmodell}

\begin{frame}
\frametitle{Belysningsmodell}

\begin{columns}[c]

\column{.5\textwidth}

\uncover<2->
{
\begin{itemize} %[<+(2)->]
\item Fysikaliskt härledd
\item Utgår ifrån renderingsekvationen
\item Statistisk modell (använder vågspektrumet)
\item Snarlik i utseendet till Blinn--Phong
\end{itemize}
}

\column{.5\textwidth}

\uncover<2->
{
\begin{figure}
\centering
\includegraphics[width=\textwidth]{../Report/Images/Public_domain/Sunglint_close_to_the_horizon}
\end{figure}
}

\end{columns}

\end{frame}


\section{Summary (5 min)}

\begin{frame}
\end{frame}


\end{document}
