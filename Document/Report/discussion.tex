\chapter{Discussion}

Whenever a cell with $\alpha = 0$ doesn't border to any cell with $\alpha > 0$, it is not needed any longer.

The idea was to remove them, although this feature was never implemented. Although the advection scheme that is used is ideal in \idxs{one}{dimension}\index{dimensions!one|see{one dimension}}, it proved to smear the \idxs{$\alpha$}{field} out somewhat in \idxse{two}{dimensions}{two} and \idxs{three}{dimensions}, making the interface between water and vacuum become thicker and thicker, which affects both the \idxs{simulation}{quality}, since the surface becomes less and less well-defined as the simulation goes on, as well as the \preformance of the \idxs{numerical}{method} since it will have to consider more and more cells as the interface gets thicker and thicker. Another advection scheme that maybe could remedy this problem is \MULES, described in \citep{Berberovi2009} and further developed to for better handling of more than two phases in \citep{Kissling2010}, although it would have to be modified to cope with compressible flow.

\section{Difficulties and drawbacks with the method}

Difficulties:
\begin{itemize}
    \item Dynamical creation/termination of surface cells and determination of properties in new cells
    \item High Courant numbers
    \item High speeds of sound (remedied in \textit{\href{http://physbam.stanford.edu/~kwatra/papers/compressible_semi_implicit/compressible_semi_implicit.pdf}{A method for avoiding the acoustic time step restriction in compressible flow}})
    \item Keeping a sharp interface
    \item Making it work in realtime
\end{itemize}

\section{Conservation laws}

\begin{itemize}
    \item Mass
    \item Energy
    \item (d/d$t$)\,Momentum $-$ Force = 0
    \item Angular momentum
    \item (d/d$t$)\,(Center of mass) $-$ Momentum = 0
\end{itemize}

\section{Already existing software}

\begin{itemize}
    \item OpenFOAM (\red{FOAM, see} \textit{\href{http://powerlab.fsb.hr/ped/kturbo/openfoam/docs/foam.pdf}{A tensorial approach to computational continuum mechanics using object-oriented techniques}})
\end{itemize}

Reasons to use them: They have been tried out and they work, maybe not exactly for these purtposes, though. It saves time to use them instead of having to develop your own software. Reasons for developing you own software: You get full control of what you are doing, so you can adapt the software after your needs. You can approximate and simplify the things that are usually important but not for this purpose, so you can optimize the code for your purpose. You don't have to pay for any license; this otherwise have a tendency to become really expensive for a large company.

\section{Other methods to use}

For realtime simulation of surface waves in large bodies of water, where non-linear phenomena are not of any significant importance, the best method to use, considering both implementation time and simulation quality, is probably spectral methods.