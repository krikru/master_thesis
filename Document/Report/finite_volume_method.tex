\chapter{Finite volume method}

%TODO: Insert time values for more discretisized variables
%TODO: Insert error bounds for more discretized equations

%The \FVM is a way to realistically simulate the \idx{flow} of a \idx{fluid} by dividing the fluid into a large number of non-moving, adjactent \idxp{cell}{s} and letting the fluid flow between the cells, through the \idxp{cell face}{s}. The magnitude of the \idxs{fluid}{flux} between two cells is directly proportional to the area of the cell face between the cells and the component of the velocity of the fluid in the direction of the cell face \idx{normal}.

%The motion of the fluid is described by a \PDE, or a set of \PDE\s.

%When simulating fluids with the \FVM it is common to \idxe{approximation}{approximate} the fluid as being \idxse{incompressible}{fluid}{incompressible}. This is because many numerical methods for solving the discretized \PDE

%This is because it is difficult to simulate compressible fluids when the \idx{time step} becomes large, due to \idxe{instability}{instabilities} in the \idx{numerical method} which leads to \idxs{spurious}{oscillations}.

\newrobustcmd{\gammapath}{{\gamma[\vec{r}_1,\,\vec{r}_2]}}
\newabbrev{\textgammapath}{\mbox{$\gammapath$}}

The \FVM is a way of solving a \PDE, or a set of \PDEs, where the room is \discretized into a large number of non-moving, adjactent volume elements which we will call \cells. Different \properties are discretized into certain points. \idxse{scalar}{field}{Scalar fields} are usually discretized to the \idxsp{cell}{center}{s}, or sometimes to \idxp{node}{s} of \idxse{cell}{corner}{corners} of the cells, as in \citep{Losasso2004}, which can be convenient since \interpolation of fields discretized to the cell centers tend to be more difficult. In a \idxs{collocated}{grid}, all properties are stored at the same locations, so the \idxse{vector}{property}{vector properties} are discretized to the same locations as the \idxse{scalar}{property}{scalar properties}. On a \idxs{staggered}{grid} on the other hand, the \velocity (or the \momentum, depending on implementation) is discretized to the \idxsp{cell}{face}{s}. For \thiswork, a staggered grid has been used, and throughout the report the discretization locations for the various properties will be called \idxsp{storage}{location}{s}.

\paragraph{Fluid simulation}

The \FVM can handle \simulation of fluids in realistic way. It is natural for representing fluids since it represents fluids, which are continous medias (at least on the level they are simulated), in a continuous ways, in contrast to when representing the fluids as \idxse{fluid}{particles}{particles}, which are discrete.

When using the \FVM, the simulation can be concentrated to the interesting parts of the flow by using \idxs{adaptive}{mesh refinement}, as in \citep{Popinet2003,Losasso2004}, to speed up the simulation orders of magnitude without loosing orders of magnitude in numerical precision in the interesting parts of the flow. This doesn't really have any natural correspondence when using particles. It is possible to initialize the particles with different \idxsp{effective}{size}{s} %TODO: Include reference to work where this has been done. Can't find it now! I know that I'm supposed to have it somewhere.
in order to make some parts of flow have a lower \idxs{particle}{density} and hence require less \idxs{computational}{power} per unit volume; in that way \idxs{numerical}{precision} can be traded for speed. However, this introduces another problem --- as the flow evolves, \advection may cause large particles to end up at places in the flow where high numerical precision is desired and decrease the numerical precision to under the required level. Besides, if the particles have a high velocity relative to each other (a high \temperature), \diffusion will cause large particles and small particles to mix and the large particles will once again end up at places in the flow where high numerical precision is desired.

One naive attempt to solve this problem could be to dynaimcally resize the particles as they end up in parts of the flow with different requirements on the numerical precision. However, this will add or remove mass a those locations, so the simulation will not \idxse{conservation of}{mass}{conserve mass}. A succeeding naive attempt to solve this problem in turn could be to distribute mass that has been removed uniformly over all other particles by scaling them with a factor, but that would lead to a non-physical transportation of mass wich would move the center of mass, and it would even cause the simulation to not \idxse{conservation of}{momentum}{conserve momentum}. A better remedy to this problem is to split too large particles into smaller particles and merge too small particles into larger particles as first done in \citep{Desbrun1999} and later improved in a number of reports, for example in \citep{Yan2009}. One drawback with this method is that it causes the \idxs{force}{field} to change discontinuously, hence \red{emitting weak \idxsp{shock}{wave}{s}}. %TODO: Really? %TODO: Still many particles

When applying the \FVM in \CFD, it simulates the \flow of a \fluid by dividing the fluid into a large number of non-moving, adjactent \cells and letting the fluid flow between the cells, through the \idxsp{cell}{face}{s}. The motion of the fluid is described by a set of \PDEs, usually the \idxs{Euler}{equations} or the \idxs{Navier--Stokes}{equations}.

The main difference between the Navier--Stokes equations and the Euler equations is that the Navier--Stokes equations takes \index{visousity|see{viscous force}}\idxsp{viscous}{force}{s} into account whe\-reas the Euler equations do not. The Euler equations are therefore a special case of the Navier--Stokes equations. Many textbooks also omits \idxsp{external}{force}{s} when writing about the Euler equations, although gravity which is such an external force usually is included when simulating \idxs{free surface}{flow} using the Euler equations.

\section{Divergence calculation}

In the \PDEs, in order to calculate the \divergence of a \idxs{vector}{field}, the \idxs{divergence}{theorem} is used and a \idxs{volume}{integral} of the divergence of the field is converted to a \idxs{surface}{integral} of the vector field itself. The divergence theorem states that

\begin{equation} \label{eq:divergence_theorem}
\iiint_V\nabla\cdot\vec{F}(\vec{r'})\,\opd V \,=\, \oiint_S(\vec{F}(\vec{r'})\cdot\normal)\,\opd S
\end{equation}

where $\vec{F}$ is a vector field, $V$ is a \idxs{control}{volume}, which in our case is the cell surrounding the point $\vec{r}$ in which the divergence is to be calculated, $S$ is the surface of the control volume, with \idxs{normal}{vector} pointing outwards, $\opd V$ and $\opd S$ are \infinitesimal elements in $V$ and $S$ respectively, $\normal$ is the normal of $\opd S$ and $\vec{r'}$ is the position of $\opd V$ and $\opd S$ respectively. The divergence of $\vec{F}(\vec{r})$ is then \approximated as the \average divergence of $\vec{F}$ in $V$ and calculated as

\begin{equation} \label{eq:divergence_surface_integral}
\nabla\vec{F}(\vec{r}) \,=\, \frac{1}{V}\,\oiint_S(\vec{F}\cdot\normal)\,\opd S.
\end{equation}

In the \FVM, the surface of a cell consists of \idxsp{cell}{face}{s}, $S_i$, between the cell itself and \neighboring cells, so \eqref{eq:divergence_surface_integral} can be rewritten as

\begin{equation} \label{eq:divergence_cell_face_sum}
\nabla\vec{F}(\vec{r})\ =\ \frac{1}{V}\,\sum_{S_i} \oiint_{S_i}(\vec{F}\cdot\normal)\,\opd S\ =\ \frac{1}{V}\,\sum_{S_i} F_i\,S_i,
\end{equation}

where $S_i$ is the \area of the cell face to the $i$:th neighbor cell, and $F_i$ is the average \idxs{field}{flux} through $S_i$, defined as

\begin{equation} \label{eq:fi_integral}
F_i \,=\, \frac{1}{S_i}\oiint_{S_i}(\vec{F}\cdot\normal)\,\opd S.
\end{equation}

In the \FVM, cell faces are usually flat, which means that the normal vector $\normal$ is constant for a certain cell face $S_i$. \eqref{eq:fi_integral} can therefore be rewritten as

\begin{equation} \label{eq:fi_flat_cell_face}
F_i \,=\, \frac{1}{S_i}\,\normal_i\cdot\oiint_{S_i}(\vec{F})\,\opd S,
\end{equation}

where $\normal_i$ is the normal of $S_i$. $F_i$, which is now just the $\normal_i$-component of the average value of the field on the cell face $S_i$, is on a staggered grid stored directly on $S_i$.

\section{Gradient calculation}

For \idxsp{orthogonal}{grid}{s}, the \gradient of a \idxs{scalar}{field} is calculated in a similar way, but in this case the \idxs{gradient}{theorem} is used. The gradient theorem states that

\begin{equation} \label{eq:gradient_theorem}
\phi(\vec{r}_2)-\phi(\vec{r}_1) \,=\, \int_\gammapath\nabla\phi(\vec{r'})\cdot \opd\vec{r'},
\end{equation}

where $\phi$ is a scalar field, \textgammapath is a path within $\phi$'s domain, connecting the vectors $\vec{r}_1$ and $\vec{r}_2$ and $\int_\gammapath$ denotes a \idxs{path}{integral} along \textgammapath. By dividing both sides of \eqref{eq:gradient_theorem} with \mbox{$\Delta r = |\vec{r}_2\,-\,\vec{r}_1|$}, we obtain

\begin{equation} \label{eq:gradient_theorem_divided}
\frac{\phi(\vec{r}_2)-\phi(\vec{r}_1)}{\Delta r} \,=\, \frac{\int_\gammapath\nabla\phi(\vec{r'})\cdot \opd\vec{r'}}{\Delta r} \,=\, \frac{\int_\gammapath\nabla\phi(\vec{r'})\cdot\frac{\Delta\vec{r}}{|\Delta\vec{r}|} \opd r'}{\Delta r}
\end{equation}

where $\Delta\vec{r} = \vec{r}_2 -  \vec{r}_1$ and

\begin{equation}
\nabla\phi(\vec{r})\cdot\frac{\Delta\vec{r}}{|\Delta\vec{r}|} = \phi'_{\Delta\vec{r}}(\vec{r}),
\end{equation}

which is just the \derivative of $\phi(\vec{r})$ in the direction of $\Delta\vec{r}$. By assuming the simplest path possible from $r_1$ to $r_2$, which is just a straight line, $\Delta r$ can be written as

\begin{equation}
\Delta r = \int_\gammapath\opd r'
\end{equation}

and \eqref{eq:gradient_theorem_divided} becomes

\begin{equation} \label{eq:phi_derivative_integral}
\frac{\phi(\vec{r}_2)-\phi(\vec{r}_1)}{\Delta r} \,=\, \frac{\int_\gammapath\phi'_{\Delta\vec{r}}(\vec{r'})\opd r'}{\int_\gammapath\opd r'}
\end{equation}

where the right hand side can be identified as the \average value of $\phi'_{\Delta\vec{r}}(\vec{r})$ along the path \textgammapath. Provided that $\vec{r}$ is close enough to \textgammapath (preferably equal to \mbox{$(\vec{r}_1\,+\,\vec{r}_2)/2$}), $\phi'_{\Delta\vec{r}}(\vec{r})$ is \approximated as this average and calculated as

\begin{equation} \label{eq:phi_derivative_final}
\phi'_{\Delta\vec{r}}(\vec{r}) \,=\, \frac{\phi(\vec{r}_2)-\phi(\vec{r}_1)}{\Delta r}.
\end{equation}

The gradient of a scalar field can be written as

\begin{equation} \label{eq:gradient_orthogonal}
\nabla\phi(\vec{r}) \,=\, \left(\frac{\partial}{\partial x}\,\normvec{x}\,+\,\frac{\partial}{\partial y}\,\normvec{y}\,+\,\frac{\partial}{\partial z}\,\normvec{z}\right) \phi(\vec{r}) \,=\, \phi'_{\normvec{x}}(\vec{r})\,\normvec{x}\,+\,\phi'_{\normvec{y}}(\vec{r})\,\normvec{y}\,+\,\phi'_{\normvec{z}}(\vec{r})\,\normvec{z}
\end{equation}

where $\normvec{x}$, $\normvec{y}$ and $\normvec{z}$ are the \normalized \idxsp{base}{vector}{s} along the three \orthogonal \idxse{grid}{axis}{grid axes} and $x$, $y$ and $z$ are the \idxp{coordinate}{s} along the grid axes. Since we are on an orthogonal grid, we can assume that the location $\vec{r}$ in which the gradient is to be calculated will be the center (or corner) of a cell with 6 neighboring cell centers (or cell corners):

\begin{equation} \label{eq:neighboring_locations}
\begin{cases}
\vec{r}_{x^-} \,=\, \vec{r} \,-\, \Delta x\,\normvec{x}\\[1ex]
\vec{r}_{x^+} \,=\, \vec{r} \,+\, \Delta x\,\normvec{x}\\[1ex]
\vec{r}_{y^-} \,=\, \vec{r} \,-\, \Delta y\,\normvec{y}\\[1ex]
\vec{r}_{y^+} \,=\, \vec{r} \,+\, \Delta y\,\normvec{y}\\[1ex]
\vec{r}_{z^-} \,=\, \vec{r} \,-\, \Delta z\,\normvec{z}\\[1ex]
\vec{r}_{z^+} \,=\, \vec{r} \,+\, \Delta z\,\normvec{z}
\end{cases}\ ,
\end{equation}

%\begin{samepage}
where $\Delta x$, $\Delta y$ and $\Delta z$ are the grid spacings in the $\normvec{x}$, $\normvec{y}$ and $\normvec{z}$ directions respectively. By combining \eqrefs \ref{eq:phi_derivative_final}, \ref{eq:gradient_orthogonal} and \ref{eq:neighboring_locations}, we can write the gradient of $\phi$ as

\begin{equation} \label{eq:gradient_final}
\nabla\phi(\vec{r}) \,=\,
\frac{\phi(\vec{r}_{x^+})-\phi(\vec{r}_{x^-})}{2\,\Delta x}\,\normvec{x} \,+\,
\frac{\phi(\vec{r}_{y^+})-\phi(\vec{r}_{y^-})}{2\,\Delta y}\,\normvec{y} \,+\,
\frac{\phi(\vec{r}_{z^+})-\phi(\vec{r}_{z^-})}{2\,\Delta z}\,\normvec{z}.
\end{equation}
%\end{samepage}

\section{Navier--Stokes equations}

The \idxs{Navier--Stokes}{equations} are a statement of the conservation of momentum for a fluid. The \idxse{general form of the}{equations of fluid motion}{general form of the} equations reads

\begin{equation} \label{eq:navier_stokes}
\rho\left(\frac{\partial\vec{u}}{\partial t} + \vec{u}\cdot\nabla\vec{u}\right) = -\nabla p + \nabla\cdot\boldsymbol{\mathsf{T}} + \vec{f}
\end{equation}

where $\rho$ is the dencity of the fluid, $\vec{u}$ is the velocity, $p$ is the pressure, $\boldsymbol{\mathsf{T}}$ is the \index{stress tensor|see{deviatoric stress tensor}}\idxs{deviatoric stress}{tensor} and $\vec{f}$ is the external forces per unit volume. It should be noted that these equations do not fully describe the fluid for; for example, they do not describe convection of the fluid, nor do they describe how to obtain any of the fields $p$, $\boldsymbol{\mathsf{T}}$ or $\vec{f}$ that the Navier--Stokes equations depends on.

By \idxse{time}{discretization}{time discretizing} and rewriting \eqref{eq:navier_stokes}, and choosing the value of $\vec{u}$ in timestep $n$ and the value of $\partial\vec{u}/\partial t$ in timestep $n+\frac{1}{2}$, thus introducing an $O(\Delta t)$ error where $\Delta t$ is the length of the time step, we obtain

\begin{equation} \label{eq:navier_stokes_time_discretized}
\vec{u}_{n+1}  = \vec{u}_{n} + \Delta t\left(-\vec{u}_{n}\cdot\nabla\vec{u}_{n} \,+\, \frac{-\nabla p + \nabla\cdot\boldsymbol{\mathsf{T}} + \vec{f}}{\rho}\right),
\end{equation}

where $\vec{u}_{n}$ denotes the velocity in time step $n$. Using a method described e.g.\ in \citep{Losasso2004}, this \PDE can be solved in two steps. First, an intermediate velocity field $\vec{u}^*_{n+1}$ is calculated ignoring the pressure term, that is

\begin{equation} \label{eq:intermediate_velocity}
\vec{u}^*_{n+1}  = \vec{u}_{n} + \Delta t\left(-\vec{u}_{n}\cdot\nabla\vec{u}_{n} \,+\, \frac{\nabla\cdot\boldsymbol{\mathsf{T}} + \vec{f}}{\rho}\right),
\end{equation}

and second, the velocity update is calculated as

\begin{equation} \label{eq:velocity_update}
\vec{u}_{n+1} \,=\, \vec{u}^*_{n+1} - \Delta t\nabla p_{n+1}.
\end{equation}

%In \thiswork, a \idxs{staggered}{grid} has been used, so the velocities are stored on the \idxsp{cell}{face}{s}. Furthermore, only the velocity component relevant to \eqref{eq:fi_flat_cell_face} is stored, that is, only the velocity component in the direction of the cell face normal $\normal$. We can therefore calculate the scalar multiplication of $\normal$ with both sides of \eqref{eq:velocity_update} to get

%\begin{equation} \label{eq:velocity_update_staggered_grid_pressure_derivative}
%u_{\normal,\,n+1} \,=\, u^*_{\normal,\,n+1} - \Delta t\,p'_{\normal,\,n+1},
%\end{equation}

%where $u_{\normal,\,n}$ and $u^*_{\normal,\,n}$ are the $\normal$-direction components of $\vec{u}_{n}$ and $\vec{u}^*_{n}$ respectivelly at timestep $n$, and $p'_{\normal\,n}$ is the pressure derivative in the direction of $\normal$ at timestep $n$. Since the cell face on which $\vec{u}_{\normal}$ is stored separates two adjactent cells whose cell centers are separated by a vector parallel to $\normal$, we can using \eqref{eq:phi_derivative_final} to rewrite this equation as

%\begin{equation} \label{eq:velocity_update_staggered_grid_final}
%u_{\normal,\,m+1} \,=\, u^*_{\normal,\,m+1} - \Delta t\,\frac{p_{2,\,n+1} - p_{1,\,n+1}}{\Delta r},
%\end{equation}

%where $p_{1,\,n}$ is the pressure in the cell center of the cell which $\normal$ points from at time step $n$, $p_{2,\,n+1}$ is the pressure in the cell center of the cell which $\normal$ points to at time step $n$ and $\Delta r$ is the distance between the two cell centers.

\section{Continuity equation}

For a \idxs{control}{volume} $V$ with surface $S$ and a surface normal $\normal$ pointing outwards, the amount of \idxs{mass}{flux} $\partial m/\partial t$ entering the control volume can be described by

\begin{equation} \label{eq:mass_flux_surface_integral}
\frac{\partial m}{\partial t} \,=\, -\oiint_S(\rho\vec{u}\cdot\normal)\,\opd S.
\end{equation}

Using the \idxs{divergence}{theorem} (\eqref{eq:divergence_theorem}) and dividing with $V$, we can rewrite \eqref{eq:mass_flux_surface_integral} as

\begin{equation} \label{eq:mass_flux_volume_integral}
\frac{\partial (m/V)}{\partial t} \,=\, -\frac{1}{V}\iiint_V\nabla\cdot(\rho\vec{u})\,\opd V
\end{equation}

and in the limit where $V \,\rightarrow\, 0$, this equation turns into

\begin{equation} \label{eq:density_partial_time_derivative}
\frac{\partial \rho}{\partial t} \,=\, -\nabla\cdot(\rho\vec{u}),
\end{equation}

where the density is defined as $\rho = \opd m/\opd V$. By subtracting the right hand side from both sides we obtain

\begin{equation} \label{eq:continuity_equation}
\frac{\partial \rho}{\partial t} + \nabla\cdot(\rho\vec{u}) \,=\, 0.
\end{equation}

This is known as the \idxs{continuity}{equation} and has to be satisfied in order to ensure \idxs{conservation of}{mass}. \idxse{time}{discretization}{Time discretizing} this equation, and choosing to use the value of $\rho$ in timestep $n+1$ and the value of $\partial \rho/\partial t$ in time step $n+\frac{1}{2}$, thus giving an $O(\Delta t)$ error, gives

\begin{equation} \label{eq:continuity_equation_time_discretized}
\rho_{n+1} \,=\, \rho_{n} - \Delta t\,\nabla\cdot(\rho_{n}\vec{u}_{n}),
\end{equation}

where $\rho_{n}$ is the value of $\rho$ at timestep $n$.

\section{Velocity advection and conservation of momentum}

\section{Pressure calculation}

Neither the Euler equations nor the Navier--Stokes equations specify how the pressure should be calculated, so when solving either of these two set of equations one is essentially free to calculate the pressure in whichever way desired. Together with a \idxs{pressure}{model}, the two sets of equations come in two major forms which usually differes significantly in implementation and stability: The \indexs{compressible Navier--Stokes}{equations}\indexs{compressible Euler}{equations}\compressible forms and the \indexs{incompressible Navier--Stokes}{equations}\indexs{incompressible Euler}{equations}\incompressible forms. For \thiswork, the compressible Euler equations have been used. %TODO: Have they been fully solved? (advection of the velocity field)

\subsection{Compressible flow}

In nature, all fluids are compressible, so a physically correct \idxs{pressure}{model} will let the fluids contract and expand which means that for \indexs{compressible}{fluid}\idxs{compressible}{flow}, the \divergence of the \idxs{velocity}{field} is allowed to be non-zero. The \pressure is then usually expressed as a function of the density; sometimes also taking into account the \temperature and other \properties that may affect the pressure, that is

\begin{equation} \label{eq:pressure_compressible_flow}
p = p\,(\rho,\,T,\,\text{other material properties}),
\end{equation}

where $T$ is the temperature.

However, the set of fluid motion equations including \eqref{eq:pressure_compressible_flow} is \idxse{stiff}{equation}{stiff}, and ordinary \idxsp{numerical}{method}{s} for solving this set of equations are known to give rise to \indexs{spurious}{oscillations}\indexs{spurious}{oscillations}\indexify{spurious spurious oscillations} in the solutions when the \idxs{speed of}{sound} multiplied by the \idx{time step} becomes too large in relation to the \idxs{characteristic}{length} of the \cells, making the numerical methud \unstable. More generally, this stringent restriction is known as the \CFL condition.

Still, not all solvers for compressible flow suffers from this problem. In \citep{Kwatra2009}, the \CFL condition is alleviated by introducing a \idxs{pressure}{field}, separated from the \idxs{density}{field}, and updating the pressure and velocity fields using what looks like the \index{implicit Euler method|see{backward Euler method}}\idxs{backward}{Euler method}, which leads to a \idxs{Poisson}{equation} for solving the pressure field. The remaining fields are then updated with the standard (forward) \idx{Euler method}. This technique doesn't lead to spurious acoustic oscillations and is similar to the technique used for solving incompressible flow, which also gives rise to a Poisson equation for solving the pressure field. As notet by \citep{Kwatra2009}, it also leads to the same Poisson equation as for incompressible flow in the limit where the \idxs{speed of}{sound} goes to infinity.

\subsection{Incompressible Navier--Stokes equations}

When accoustic waves is of no significant importance to the simulation, it is probably most common to model the fluids as \indexs{incompressible}{fluid}\indexs{incompressible}{flow}\incompressible. When simulating incompressible flow, a different approach is taken to calculate the \idxs{pressure}{field}.

Since the flow is incompressible, the density will be a constant, which means that \derivatives of $\rho$ vanishes, that is

\begin{equation} \label{eq:density_partial_time_derivative_incompressible_flow}
\frac{\partial \rho}{\partial t} \,=\, 0
\end{equation}

and

\begin{equation} \label{eq:density_divergence_incompressible_flow}
\nabla\rho \,=\, \vec{0}.
\end{equation}

\eqref{eq:continuity_equation} will then turn into

\begin{equation} \label{eq:velocity_divergence_incompressible_flow}
\nabla\cdot\vec{u} \,=\, 0.
\end{equation}

Furthermore, %TODO: Furthermore what?

\begin{equation} \label{eq:deviatoric_stress_tensor_incompressible_flow}
\nabla\cdot\boldsymbol{\mathsf{T}} \,=\, \mu\nabla^2\vec{u}.
\end{equation}

For simplicity, we can assume that we use a set of units where $\rho = 1$. \eqref{eq:intermediate_velocity} can then be rewritten as

\begin{equation} \label{eq:intermediate_velocity_reduced}
\vec{u}^*_{n+1} \,= \, \vec{u}_{n} + \Delta t(- \vec{u}_{n}\cdot\nabla\vec{u}_{n} \,+\, \mu\nabla^2\vec{u}_{n} + \vec{f}),
\end{equation}

which can be directly solved assuming $\mu$ and $\vec{f}$ are known.

However, \eqref{eq:velocity_update}, which is used to update the velocity, contains $p$ which is a second unknown and must be calculated before $\vec{u}$ can be calculated. As outlined in e.g.\ \citep{Losasso2004}, by rewriting \eqref{eq:velocity_update}, taking the divergence of both sides and using \eqref{eq:velocity_divergence_incompressible_flow} to get rid of $\nabla\cdot\vec{u}$, we obtain the \idxs{Poisson}{equation}

\begin{equation} \label{eq:pressure_poissin_equation_incompressible_flow}
\nabla^2 p_{n+1} \,=\, \frac{\nabla\cdot\vec{u}^*_{n+1}}{\Delta t}
\end{equation}

which need to be solved before we can update the velocity completely.

When \idxse{spatial}{discretization}{discretizing} this equation spatially a \idxs{system of linear}{equations} is obtained, for which there exists many solution methods with varing speed and accuracy. As a comparison, it can be notet that the naive \idxs{Gaussian}{elimination}\index{algorithm!Gaussian elimination|see{Gaussian elimination}} algorithm, or the \idxs{Gauss--Jordan}{elimination}\index{algorithm!Gauss--Jordan elimination|see{Gauss--Jordan elimination}} algorithm for a multicore system, has a \idxs{time}{complexity} of $O(N^3)$ for an $N\times N$ \idx{matrix}. However, an $O(N^3)$ time for solving the pressure Poisson equation would slow down the simulation tremendously, since it otherwise runs in $O(N)$ time per time step.

Since \incompressibility is only an \approximate \property of the fluid, it is arguably enough to only approximately solve the pressure Poisson equation, which is the governing equation for incompressibility. This assumptions enables a large set of fast, iterative methods for solving the pressure equation, such as the \idxse{multilevel}{acceleration}{multilevel accelerated} \idxs{Jacobi}{method} used in \citep{Popinet2003}. However, there exist iterative methods that will solve the pressure Poisson equation down to \idxs{machine}{precision} in only a few number of iterations, such as the \PCG method, which can be applied if the matrix is \idxse{symmetric}{matrix}{symmetric}; this method was used in \citep{Losasso2004} with an incomplete \idxs{LU Cholesky}{factorization} as \preconditioner.

If the pressure equation is only solved approximately, \eqref{eq:velocity_divergence_incompressible_flow} is not perfectly satisfied, so each cell will gain or lose some mass during each time step. If perfect \idxs{conservation of}{mass} is essential the deviation of mass has to be recorder, so the \idxs{density}{field} $\rho$ must be reintroduced. Instead of satisfying \eqref{eq:velocity_divergence_incompressible_flow}, one would rather prefer that the density very quickly becomes one again. The time $n+1$ density is given by \eqref{eq:continuity_equation_time_discretized} which is fully determined since $\rho_{n+1}$ is the only unknown in the equation, but the time $n+2$ density is given by substituting $n$ for $n+1$ in \eqref{eq:continuity_equation_time_discretized}, that is

\begin{equation} \label{eq:continuity_equation_time_discretized_postponed}
\rho_{n+2} \,=\, \rho_{n+1} - \Delta t\,\nabla\cdot(\rho_{n+1}\vec{u}_{n+1})
\end{equation}

which is underdetermined since $\vec{u}_{n+1}$ also is unknown. We can therefore make the requirement that

\begin{equation} \label{eq:density_conservation}
\rho_{n+2} \,=\, 1.
\end{equation}

Perfect satisfaction of \eqref{eq:density_conservation} will not take place due to a not perfectly solved pressure equation, but it is still the goal and also what we are going to assume. By rearanging \eqref{eq:continuity_equation_time_discretized_postponed} and expanding the divergence, we obtain

\begin{equation} \label{eq:velocity_divergence_density_conservation_premature}
\nabla\cdot\vec{u}_{n+1} \,=\, \frac{1-\rho_{n+1}^{-1}\,\rho_{n+2}}{\Delta t} \,-\, \rho_{n+1}^{-1}\,\vec{u}_{n+1}\cdot\nabla\rho_{n+1}.
\end{equation}

If we assume that the \idxs{Courant}{number} is much lower than one, or that the \idxs{Courant}{number} is limited and that the \spectrum of $\rho_{n+1}$ is dominated by frequencies much lower than the \idxs{Nyquist}{frequency}, $\rho_{n+1}^{-1}\,\vec{u}_{n+1}\cdot\nabla\rho_{n+1}$ will be a minor term in \eqref{eq:velocity_divergence_density_conservation_premature} and can therefore be \neglected. Furthermore, since $\rho_{n+1}$ is close to $1$, we can \approximate $\rho_{n+1}^{-1}$ as the first-order \idxs{Taylor}{series} about the value $1$, which is $2-\rho_{n+1}$. Hence, by using \eqref{eq:density_conservation}, we can rewrite \eqref{eq:velocity_divergence_density_conservation_premature} as

\begin{equation} \label{eq:velocity_divergence_density_conservation}
\nabla\cdot\vec{u}_{n+1} \,=\, \frac{\rho_{n+1}-1}{\Delta t}.
\end{equation}

By rewriting \eqref{eq:velocity_update}, taking the divergence of both sides and using \eqref{eq:velocity_divergence_density_conservation} to substitute $\nabla\cdot\vec{u}$, we obtain the \idxs{Poisson}{equation}

\begin{equation} \label{eq:pressure_poissin_equation_density_conservation}
\nabla^2 p_{n+1} \,=\, \frac{\nabla\cdot\vec{u}^*_{n+1}}{\Delta t} - \frac{\rho_{n+1} - 1}{\Delta t^2}.
\end{equation}

If, on the other hand the pressure equation is solved to a very high accuracy, or perfect \idxs{conservation of}{mass} is not something very important, \eqref{eq:pressure_poissin_equation_incompressible_flow} can equally well be used and then the \idxs{density}{field} becomes superfluous. For \simulation puropses, conservation of mass to this high degree is not important and hence the density field can be omitted. No matter if we need to conserve mass perfectly or not, we will obtain a \idxs{Poisson}{equation} for the \idxs{pressure}{field}, which can be written on the form

\begin{equation} \label{eq:pressure_poissin_equation_general}
\nabla^2 p \,=\, q(\vec{r}),
\end{equation}

where $q$ is a known function of $\vec{r}$.

\subsubsection{Iterative methods}

\paragraph{Gauss--Seidel method}

\subsubsection{Acceleration of iterative methods}

\paragraph{Preconditioned conjugate gradient method}

\begin{itemize}
    \item Extension of the gradient descent
\end{itemize}

See also \textit{Incomplete Cholesky Preconditioned Conjugate Gradients method}, described in \textit{\href{http://www.cs.ubc.ca/~rbridson/fluidbook/}{Fluid Simulation for Computer Graphics}}. This method uses the \textit{\href{http://en.wikipedia.org/wiki/Incomplete_Cholesky_factorization}{incomplete Cholesky factorization}} as preconditioner.

\paragraph{Multigrid method}

See
\begin{itemize}
    \item \textit{\href{http://developer.download.nvidia.com/books/cuda-by-example/cuda-by-example-sample.pdf}{CUDA by Example: An Introduction to General-Purpose GPU Programming}}
    \item \textit{\href{http://people.freebsd.org/~snb/school/hp_mg.pdf}{High Performance Multigrid for Poisson's Equation in 3D}}
    \item Or maybe \textit{\href{http://downloads.isrn.com/journals/appmath/2012/246491.pdf}{Parallel Adaptive Mesh Refinement Combined with Additive Multigrid for the Efficient Solution of the Poisson Equation}} has the answer to all your questions.
\end{itemize}


\subsection{Semicompressible water}

\section{Boundary conditions}