\chapter{Motivation}
\label{chap:motivation}

Real-time simulation of water surface waves in a flight simulator can potentially lead to several advantages.

\section{Landing on ships with helicopters}

In order for a pilot to get the most out of training in a flight simulator, the pilot has to face similar challenges as in a real world scenario. For some helicopter missions, offshore landings on a ship have to be preformed. The landings are sometimes performed so far offshore that the pilot passes the limit where the fuel that's left in the fuel tank is just enough to return to solid ground. This limit is referred to as \idxs{bingo}{fuel}. If a helicopter that is supposed to land on a ship passes this limit, the only remaining option is to land on the ship.

However, when the ship is small --- such as those that can be seen in a few videos on the web \citep{MrOawal2009,PrismDefence2010,KopulaDK2010} --- and is exposed to large waves, landing on it with a helicopter becomes difficult \citep{PrismDefence2010}. If the pilot isn't used to land the helicopter under such circumstances, this could mean disaster. It is therefore vital that the pilots can train landing on ships under similar circumstances in a simulator, before landing with a helicopter on a rolling ship for real. Hence, it becomes important that ships in the simulator behave realistically and are affected by the state of the sea in a realistic manner, which means that they start to rock forth and back when they are hit by large waves, in order to give the pilot a good \idxs{training}{value}.

\section{Visual cueing}

In order for the \idxs{human}{perception} to work, and for the \brain to be able to estimate properties in the environment, such as distances, speeds, etc. the brain uses a number of clues it gets from processing and interpreting \idxs{sensory}{information} by comparing them with already processed sensory information from similar, earlier experienced situations. These clues are commonly referred to as \idxsp{sensory}{cue}{s}.

%\subsection{Surface disturbances caused by the wind flow from the rotor blades}
\subsection{Height estimation}

When a helicopter pilot flies over a field of grass, he can for example look at how much the grass bends due to the air flow from the rotor blades in order to estimate the distance to the ground. The level of bending of the grass is therefore a \idxs{visual}{cue} when estimating the distance to the ground.

Similarly, when a helicopter pilot flies over a body of water on a day with little wind, he can estimate the distance to the surface by looking at surface disturbances cause by the air flow from the rotor blades. The level of disturbances on the surface is therefore an important visual cue when flying over a body of water.

%\subsection{Wind waves}

On a day with much wind, however, it might be difficult to see precisely how large the surface disturbances caused by the air flow from the rotor blades are, since those disturbances are dominated by the waves caused by the wind. On the other hand, those waves also act as visual cues. First, the higher the \idxs{wind}{speed} is, the larger the waves will be and the more likely it is that white foam will form on the crests of the waves; this foamy part of the wave is due to a breaking of the crest which according to the \idxs{Beaufort}{scale} start to occur already in \idxs{gentle}{breeze} (which starts at approximately 3.4~m/s) and is commonly referred to as \idxs{oceanic}{whitecaps} or just whitecaps (also known as \idxs{white}{horses}). So whether whitecaps are present or not is a visual cue when estimating the wind speed, and much foam signalizes that the wind speeds are very high and that sudden \gusts may strike. Besides, whitecaps tend to line up parallelly to the wind, and travel in the direction of the wind with a speed that is higher for higher wind speeds, since the wind is partly responsible for transporting the surface. All this combined makes whitecaps an important visual cue when estimating both the speed and the direction of the wind. The wind speed in turn determines the size of the waves, so whether whitecaps are present or not is also a visual cue when estimating the size of the waves, including their wavelength.

Besides, waves with \idxs{short}{wavelength} will have a higher frequency with which the \idxs{wave}{pattern} changes than waves with \idxs{long}{wavelength} which makes the frequency another important visual cue when estimating the \idxs{overall}{wavelength} of the waves. Furthermore, if the wavelength is known, the pilot can \estimate the distance to the surface, since the farther the distance to the surface is the smaller the characteristic size of the pattern that is formed on the pilot's retina will be. Therefore, the size of the whitecaps that are present (if they are present), the speed with which they travel, and the frequency with which the wave pattern changes are all important visual cues when estimating the distance to the surface.

%trim option's parameter order: left bottom right top
\begin{figure}
    \centering
    \subcaptionbox{ \label{fig:small_waves}}{\includegraphics[width=.495\textwidth]{Images/Attribute/Baltic_Sea_from_ferry_small}}
    \subcaptionbox{Source: Public domain. \label{fig:sea_storm}}{\includegraphics[width=0.495\textwidth]{Images/Public_domain/Wea00816}}
    \caption{Visual cues present on surfaces of large bodies of water include cues that make it possible to estimate the speed and direction of the wind, as well as cues that make it possible to estimate the distance to the surface. \subrefp{fig:small_waves} \idxse{low}{wind speed}{Low wind speeds} give rise to waves with \idxs{short}{wavelength} and no whitecaps. The temporal frequency with which the wave pattern changes is an important visual cue when estimating the size of the waves (those with a high wavenumber also have a high temporal frequency), which in turn is necessary when estimating the distance to the surface. \subrefp{fig:sea_storm} \idxse{high}{wind speed}{High wind speeds}, starting already at approximately 3.4~m/s, tend to break the crests of the waves and causes whitecaps. This photo was taken during a storm in the North Pacific Ocean the winter 1989.}
    \label{fig:sea_states}
\end{figure}

\subsection{Landing on ships with aircraft}

A fighter aircraft that is about to land on an aircraft carrier has to approach the carrier from behind, and often slightly from the side. Therefore, the pilot has to know the orientation of the carrier. During landing, the carrier travels with full speed against the wind in order to give the pilot as high relative speed to the air as possible, creating a \wake behind itself, with a \backwash that is usually clearly visible for the pilot. The reason the backwash is usually seen so well is partly because it often contains a lot of \idxsp{air}{bubble}{s} that have been dragged down into the water or is covered by foam, and is therefore much brighter than the rest of the water, and partly because of the \turbulence that is created in the water as the ship passes by which makes the surface more blank. This backwash can be seen in \figref{fig:aircraft_carriers_and_backwash}.

In \figref{fig:aircraft_carrier_full_wake}, \idxsp{V-shaped}{wavefront}{s} after the ship can also be observed. These wavefronts form two \idxsp{wake}{line}{s}, one on each side of the backwash, which together are known as the \idxs{Kelvin}{wake pattern}\index{pattern!wake|see{Kelvin wake pattern}} after the British scientist \idxs{Lord}{Kelvin}.

\begin{figure}
    \centering
    \subcaptionbox{Source: Public domain. \label{fig:aircraft_carrier_full_wake}} {\includegraphics[width=0.495\textwidth]{Images/Public_domain/The_aircraft_carrier_USS_Enterprise_(CVN_65)}}
    \subcaptionbox{Source: Public domain. \label{fig:aircraft_carrier_landing_backwash}} {\includegraphics[width=0.495\textwidth]{Images/Public_domain/An_F-A-18C_Hornet_lands_on_the_aircraft_carrier_USS_George_Washington_(CVN_73)}}
    \caption{Aircraft carriers. Note the backwash created behind the ships, serving as a reference for pilots during landing. \subrefp{fig:aircraft_carrier_full_wake} The aircraft carrier USS Enterprise (CVN 65). \subrefp{fig:aircraft_carrier_landing_backwash} An F/A-18 Hornet coming in for landing on an aircraft carrier.}
    \label{fig:aircraft_carriers_and_backwash}
\end{figure}

In a real situation, the pilot can look at the backwash and the wake lines created by the carrier in order to find the right angle with which to approach the carrier, as can be seen in some videos on the web \citep{Alivewithpassion2007,MatteoBram2007} (see also \figref{fig:aircraft_carrier_landing_backwash}). In a simulation, it is important that the pilot has the same possibility. It is therefore necessary that ships in the simulation leave a wake behind themselves that looks and behaves as a real wake, and that consists of both waves created by the ship and of a clearly visible backwash. It is valuable that this wake lives and looks like a real wake for as long time as possible. As an example of what this could be used for in a \idxs{flight}{simulator}, it would become possible for the pilot to determine the direction to a ship, even if the only thing that is spotted was a single \idxs{wake}{line}.