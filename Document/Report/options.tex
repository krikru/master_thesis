% % % % % % % % % % % % % % % % % % % % % % % % % % % % % % % % % % % % % % % % % % % % % % % %
% Options.tex
% -----------
%
% This file contains all options that can be tuned throughout the document. This is the only file that should contain any parameters that can be changed or adjusted in order to tune the document.
% % % % % % % % % % % % % % % % % % % % % % % % % % % % % % % % % % % % % % % % % % % % % % % %


% % % % % % % % % % % % % % % % % % % % % % % % % % % % % % % % % % % % % % % % % % % % % % % %
% Dimensions
% ----------
%
% (see http://nwalsh.com/tex/texhelp/Plain.html#dimensions, http://en.wikipedia.org/wiki/Point_%28typography%29)
%
% pt: Point
% pc: pica (12 pt)
% in: inch (72.27 pt)
% bp: Big point (72 bp = 1 in)
% cm: Centimeter
% mm: Millimeter
% dd: Didot point
% cc: cicero (12 dd)
% sp: Scaled point (65,536 sp = 1 pt), the smallest TeX unit
% ex: Nomimal x-height
% em: Nominal m-width (M-width?)
%
%   Available in math mode:
%
% mu: math unit, 1 em = 18 mu, where em is taken from the math symbols family, various lengths are derived from it (thinspace, thickspace, etc.)
%
%   Additionally available in pdfTeX and LuaTeX:
%
% px: "pixel", the dimension given to the \pdfpxdimen primitive; default value is 1 bp, corresponding to a pixel density of 72 dpi
% % % % % % % % % % % % % % % % % % % % % % % % % % % % % % % % % % % % % % % % % % % % % % % %


% % % % % % % % % % % % % % % % % % % % % % % % % % % % % % % % % % % % % % % % % % % % % % % %
% Spacings in math mode
% ---------------------
%
% \,       thin space (normally 1/6 of a quad)
% \> or \: medium space (normally 2/9 of a quad)
% \;       thick space (normally 5/18 of a quad)
% \!       negative thin space (normally 1/6 of a quad)
%
% TexXBook definition: \def\,{\mskip\thinmuskip} \def\!{\mskip-\thinmuskip}
%
% \thinmuskip would normally be .16667em (= 3 mu), though it might be redefined.
%
% \quad    usually 1em (derived from identities above)
% % % % % % % % % % % % % % % % % % % % % % % % % % % % % % % % % % % % % % % % % % % % % % % %



% % % % % %
% FLAGS   %
% % % % % %

% DEBUG (should be false when publishing)
% Affects: Extra information written into the document
\newflag{DEBUG}{true} % For debugging the document
%\newflag{DEBUG}{false} % For publishing the document

% PAPERPRINT
% Affects: Link colors
%\newflag{PAPERPRINT}{true} % For debugging the document
\newflag{PAPERPRINT}{false} % For publishing the document

% Miscellaneous flags
\newflag{SCRIPTSIZEINDEX}       {false} % Controls the size of the index
\newflag{SCRIPTSIZEBIBLIOGRAPHY}{false} % Controls the size of the bibliography

% % % % % % % %
% APPEARANCE  %
% % % % % % % %

% Names
\renewcommand{\contentsname}{Table of Contents} % Rename Contents to Table of contents

% Names possible to change
%
% \abstractname   Abstract
% \alsoname       see also (makeidx package)
% \appendixname   Appendix
% \bibname        Bibliography (report,book)
% \ccname         cc (letter)
% \chaptername    Chapter (report,book)
% \contentsname   Contents
% \enclname       encl (letter)
% \figurename     Figure (for captions)
% \headtoname     To (letter)
% \indexname      Index
% \listfigurename List of Figures
% \listtablename  List of Tables
% \pagename       Page (letter)
% \partname       Part
% \refname        References (article)
% \seename        see (makeidx package)
% \tablename      Table (for caption)

%\figurename             *\figurename*
%\tablename              *\tablename*
%\partname               *\partname*
%\appendixname           *\appendixname*
%\equationname           *\equationname*
%\Itemname               *\Itemname*
%\chaptername            *\chaptername*
%\sectionname            *\sectionname*
%\subsectionname         *\subsesctionname*
%\subsubsectionname      *\subsubsectionname*
%\paragraphname          *\paragraphname*
%\Hfootnotename          *\Hfootnotename*
%\AMSname                *\AMSname*
%\theoremname            *\theoremname*

% Links
\hypersetup{
    pdfborder = {0 0 0}, % Remove the frame around links
    colorlinks=\iftoggle{PAPERPRINT}{false}{true}, % Don't color links on paper prints
    citecolor=black, %Used for links to the bibliography
    linkcolor=black, %Used for internal links to labels
    urlcolor=blue, %Used for external links
}

% Header
\setlength{\headheight}{14pt} % To prevent warning " \headheight is too small (12.0pt): Make it at least 14.0pt."

% Maths

% % % Vectors
\robustify{\vec}
%\renewrobustcmd{\vec}[1]{\bar{#1}} % For bars over vectors
%\renewrobustcmd{\vec}[1]{\mathbf{#1}} % For bold font vectors (deosn't work for all characters, for example \pi\pixi)
% % % Operators
\newrobustcmd{\sop}[1]{\widehat{#1}} % For scalar operators
\newrobustcmd{\vop}[1]{\widehat{\vec{#1}}} % For vector operators
% % % Fourier transform
\newrobustcmd{\fdfunc}[1]{\widetilde{#1}} % For a function in the frequency domain
% % % Integrals
\newrobustcmd{\HalfBetweenIntegralSigns}{\!\!}
\newrobustcmd{\BetweenIntegralSigns}{\HalfBetweenIntegralSigns\HalfBetweenIntegralSigns}
% Double integral
\robustify{\iint}
\renewrobustcmd{\iint}{\int\BetweenIntegralSigns\int}
% Triple integral
\robustify{\iiint}
\renewrobustcmd{\iiint}{\int\BetweenIntegralSigns\int\BetweenIntegralSigns\int}
% Closed double integral
\newrobustcmd{\oiint}{\begingroup
    \displaystyle \unitlength 1pt
    %\let\CharactaristicSize 3pt
    %\int\mkern-7.2mu
    \int\HalfBetweenIntegralSigns\mkern-1.2mu
    \begin{picture}(0,3)
    %\put(0,3){\oval(10,8)} %\put uses units of \unitlength
    \put(0,3){\oval(10,8)} %\put uses units of \unitlength
    \end{picture}
    %\mkern-7mu\int
    \HalfBetweenIntegralSigns\mkern-1mu\int
\endgroup}
% Closed trippel integral (yet to be defined)

% Indexing
%\newrobustcmd{\indexify}[1]{#1} % For leaving indexed text in the body text as it is
\newrobustcmd{\indexify}[1]{\textit{#1}} % For italicizing the indexed text in the body text

% Referencing
\robustify{\eqref}
\renewrobustcmd{\eqref}[1]{Equation \ref{#1}}
%\renewrobustcmd{\eqref}[1]{\equationname\ \ref{#1}}

% % % % % % % % %
% ABBREVIATIONS %
% % % % % % % % %

% Abreviations
% ------------

% The abbreviations are sorted by the abbreviated forms
\declareabbreviationqi{threedim}{three-dimensional}
\declareabbreviationqi{twodim}  {two-dimensional}

% Acronyms
% --------

% The acronyms are sorted by the abbreviated forms
\declareacronym {BEM} {Boundary Element Methods}
\declareacronyms{CFD} {Computational}{Fluid Dynamics}
\declareacronyms{CFMM}{Continuous}{Fast Multipole Method}
\declareacronym {FMM} {Fast Multipole Method}
\declareacronym {FVM} {Finite Volume Method}
\declareacronym {MAC} {Marker-and-Cell}
\declareacronym {PDE} {Partial Differential Equation}
\declareacronyms{VOF} {Volume of}{Fluid}
\declareacronyms{VOS} {Volume of}{Solid}

% % % % % % % % % % % % %
% INDEX SINGLE KEYWORDS %
% % % % % % % % % % % % %

\declareindexkey      {cell}
\declareindexkeypair      {cell}{cells}
\declareindexkey      {divergence}
\declareindexkeypair      {divergence}{divergences}
\declareindexkey      {discretization}
\declareindexkeypair      {discretization}{discretize}
\declareindexkeypair      {discretization}{discretized}