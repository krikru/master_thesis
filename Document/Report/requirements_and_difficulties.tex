\chapter{Requirements and difficulties}

As noted previously, there has to be a \idxs{two-way}{interaction} between water and ships. For most of the two-dimensional wave models, which just treats the surface as a \idxs{height}{map}, there is no natural way to make water and ships interact with each other. The ships can quite easily be made to roll in a realistic way by just looking at the free surface elevation and from that approximate the \idxs{pressure}{field} felt by the \idxs{ship}{hull}, even though this method is slightly incorrect since it doesn't take into account the deviations in the pressure field caused by the ship itself. But to make the ships give rise to waves as they are traveling on the water is more challenging. One can just use a separate, static height map for the wake which moves after the ship as it is traveling. This wake will always look the same no matter where the ship is traveling and will not be affected by obstacles in the water. However, if the ship suddenly changes speed or course, so does the wake, which is a highly unnatural behaviour of a wake. A better approach may be to use map that tells the water how to respond to a boat traveling on it. Then the wake will not be stored as a separate height map, but be incorporated in the same height map that is used to simulate the waves that affects the ship.

responce will then be merged with the primary 

Difficulties:
\begin{itemize}
    \item Wave dispersion
    \item Different depths gives different speeds
    \item The water should interact with moving objects
\end{itemize}