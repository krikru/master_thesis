\part{Method}

\chapter{Finite volume method}

%The \FVM is a way to realistically simulate the \idx{flow} of a \idx{fluid} by dividing the fluid into a large number of non-moving, adjactent \idxp{cell}{s} and letting the fluid flow between the cells, through the \idxp{cell face}{s}. The magnitude of the \idxs{fluid}{flux} between two cells is directly proportional to the area of the cell face between the cells and the component of the velocity of the fluid in the direction of the cell face \idx{normal}.

%The motion of the fluid is described by a \PDE, or a set of \PDE\s.

%When simulating fluids with the \FVM it is common to \idxe{approximation}{approximate} the fluid as being \idxse{incompressible}{fluid}{incompressible}. This is because many numerical methods for solving the discretized \PDE

%This is because it is difficult to simulate compressible fluids when the \idx{time step} becomes large, due to \idxe{instability}{instabilities} in the \idx{numerical method} which leads to \idxs{spurious}{oscillations}.

\newrobustcmd{\gammapath}{{\gamma[\vec{r}_1,\,\vec{r}_2]}}
\newabbrev{\textgammapath}{\mbox{$\gammapath$}}

The \FVM is a way of solving a \PDE, or a set of \PDEs, where the room is \discretized into a large number of non-moving, adjactent volume elements which we will call \cells, and different \properties are discretized into certain points. \idxse{scalar}{field}{Scalar fields} are usually discretized to the \idxsp{cell}{center}{s}, or sometimes to \idxp{node}{s} of \idxse{cell}{corner}{corners} of the cells, as in \citep{Losasso2004}, which can be convenient since \interpolation of fields discretized to the cell centers tend to be more difficult. In a \idxs{collocated}{grid}, all properties are stored at the same locations, so the vector properties are discretized to the same locations as the scalar properties. On a \idxs{staggered}{grid} on the other hand, the \velocity (or the \momentum, depending on implementation) is discretized to the \idxsp{cell}{face}{s}. For the work described in this report, a staggered grid has been used, and throughout the report the discretization locations for the various properties will be called \idxsp{storage}{location}{s}.

\paragraph{Fluid simulation}

When applying the \FVM in \CFD, it is used to realistically \simulate the \flow of a \fluid by dividing the fluid into a large number of non-moving, adjactent \cells and letting the fluid flow between the cells, through the \idxsp{cell}{face}{s}. The motion of the fluid is described by a set of \PDEs, usually the \idxs{Euler}{equations} or the \idxs{Navier-Stokes}{equations}.

The main difference between the Navier-Stokes equations and the Euler equations is that the Navier-Stokes equations takes \index{visousity|see{viscous force}}\idxsp{viscous}{force}{s} into account whe\-reas the Euler equations do not. The Euler equations are therefore a special case of the Navier-Stokes equations. Many textbooks also omits \idxsp{external}{force}{s} when writing about the Euler equations, although gravity which is such an external force usually is included when simulating \idxs{free surface}{flow} using the Euler equations.

\section{Divergence calculation}

In the \PDEs, in order to calculate the \divergence of a \idxs{vector}{field}, the \idxs{divergence}{theorem} is used and a \idxs{volume}{integral} of the divergence of the field is converted to a \idxs{surface}{integral} of the vector field itself. The divergence theorem states that

\begin{equation} \label{eq:divergence_theorem}
\iiint_V\nabla\cdot\vec{F}(\vec{r'})\,\opd V \,=\, \oiint_S(\vec{F}(\vec{r'})\cdot\normvec{n})\,\opd S
\end{equation}

where $\vec{F}$ is a vector field, $V$ is a \idxs{control}{volume}, which in our case is the cell surrounding the point $\vec{r}$ in which the divergence is to be calculated, $S$ is the surface of the control volume, with \idxs{normal}{vector} pointing outwards, $\opd V$ and $\opd S$ are \infinitesimal elements in $V$ and $S$ respectively, $\normvec{n}$ is the normal of $\opd S$ and $\vec{r'}$ is the position of $\opd V$ and $\opd S$ respectively. The divergence of $\vec{F}(\vec{r})$ is then \approximated as the \average divergence of $\vec{F}$ in $V$ and calculated as

\begin{equation} \label{eq:divergence_surface_integral}
\nabla\vec{F}(\vec{r}) \,=\, \frac{1}{V}\,\oiint_S(\vec{F}\cdot\normvec{n})\,\opd S.
\end{equation}

In the \FVM, the surface of a cell consists of \idxsp{cell}{face}{s}, $S_i$, between the cell itself and \neighboring cells, so \eqref{eq:divergence_surface_integral} can be rewritten as

\begin{equation} \label{eq:divergence_cell_face_sum}
\nabla\vec{F}(\vec{r})\ =\ \frac{1}{V}\,\sum_{S_i} \oiint_{S_i}(\vec{F}\cdot\normvec{n})\,\opd S\ =\ \frac{1}{V}\,\sum_{S_i} F_i\,S_i,
\end{equation}

where $S_i$ is the \area of the cell face to the $i$:th neighbor cell, and $F_i$ is the average \idxs{field}{flux} through $S_i$, defined as

\begin{equation} \label{eq:fi_integral}
F_i \,=\, \frac{1}{S_i}\oiint_{S_i}(\vec{F}\cdot\normvec{n})\,\opd S.
\end{equation}

In the \FVM, cell faces are usually flat, which means that the normal vector $\normvec{n}$ is constant for a certain cell face $S_i$. \eqref{eq:fi_integral} can therefore be rewritten as

\begin{equation} \label{eq:fi_flat_cell_face}
F_i \,=\, \frac{1}{S_i}\normvec{n}_i\cdot\oiint_{S_i}(\vec{F})\,\opd S,
\end{equation}

where $\normvec{n}_i$ is the normal of $S_i$. $F_i$, which is now just the $\normvec{n}_i$-component of the average value of the field on the cell face $S_i$, is on a staggered grid stored directly on $S_i$.

\section{Gradient calculation}

For \idxsp{orthogonal}{grid}{s}, the \gradient of a \idxs{scalar}{field} is calculated in a similar way, but in this case the \idxs{gradient}{theorem} is used. The gradient theorem states that

\begin{equation} \label{eq:gradient_theorem}
\phi(\vec{r}_2)-\phi(\vec{r}_1) \,=\, \int_\gammapath\nabla\phi(\vec{r'})\cdot \opd\vec{r'},
\end{equation}

where $\phi$ is a scalar field, \textgammapath is a path within $\phi$'s domain, connecting the vectors $\vec{r}_1$ and $\vec{r}_2$ and $\int_\gammapath$ denotes a \idxs{path}{integral} along \textgammapath. By dividing both sides of \eqref{eq:gradient_theorem} with \mbox{$\Delta r = |\vec{r}_2\,-\,\vec{r}_1|$}, we obtain

\begin{equation} \label{eq:gradient_theorem_divided}
\frac{\phi(\vec{r}_2)-\phi(\vec{r}_1)}{\Delta r} \,=\, \frac{\int_\gammapath\nabla\phi(\vec{r'})\cdot \opd\vec{r'}}{\Delta r} \,=\, \frac{\int_\gammapath\nabla\phi(\vec{r'})\cdot\frac{\Delta\vec{r}}{|\Delta\vec{r}|} \opd r'}{\Delta r}
\end{equation}

where $\Delta\vec{r} = \vec{r}_2 -  \vec{r}_1$ and

\begin{equation}
\nabla\phi(\vec{r})\cdot\frac{\Delta\vec{r}}{|\Delta\vec{r}|} = \phi'_{\Delta\vec{r}}(\vec{r}),
\end{equation}

which is just the \derivative of $\phi(\vec{r})$ in the direction of $\Delta\vec{r}$. By assuming the simplest path possible from $r_1$ to $r_2$, which is just a straight line, $\Delta r$ can be written as

\begin{equation}
\Delta r = \int_\gammapath\opd r'
\end{equation}

and \eqref{eq:gradient_theorem_divided} becomes

\begin{equation} \label{eq:phi_derivative_integral}
\frac{\phi(\vec{r}_2)-\phi(\vec{r}_1)}{\Delta r} \,=\, \frac{\int_\gammapath\phi'_{\Delta\vec{r}}(\vec{r'})\opd r'}{\int_\gammapath\opd r'}
\end{equation}

where the right hand side can be identified as the \average value of $\phi'_{\Delta\vec{r}}(\vec{r})$ along the path \textgammapath. Provided that $\vec{r}$ is close enough to \textgammapath (preferably equal to \mbox{$(\vec{r}_1\,+\,\vec{r}_2)/2$}), $\phi'_{\Delta\vec{r}}(\vec{r})$ is \approximated as this average and calculated as

\begin{equation} \label{eq:phi_derivative_final}
\phi'_{\Delta\vec{r}}(\vec{r}) \,=\, \frac{\phi(\vec{r}_2)-\phi(\vec{r}_1)}{\Delta r}.
\end{equation}

The gradient of a scalar field can be written as

\begin{equation} \label{eq:gradient_orthogonal}
\nabla\phi(\vec{r}) \,=\, \left(\frac{\partial}{\partial x}\,\normvec{x}\,+\,\frac{\partial}{\partial y}\,\normvec{y}\,+\,\frac{\partial}{\partial z}\,\normvec{z}\right) \phi(\vec{r}) \,=\, \phi'_{\normvec{x}}(\vec{r})\,\normvec{x}\,+\,\phi'_{\normvec{y}}(\vec{r})\,\normvec{y}\,+\,\phi'_{\normvec{z}}(\vec{r})\,\normvec{z}
\end{equation}

where $\normvec{x}$, $\normvec{y}$ and $\normvec{z}$ are the \normalized \idxsp{base}{vector}{s} along the three \orthogonal \idxse{grid}{axis}{grid axes} and $x$, $y$ and $z$ are the \idxp{coordinate}{s} along the grid axes. Since we are on an orthogonal grid, we can assume that the location $\vec{r}$ in which the gradient is to be calculated will be the center (or corner) of a cell with 6 neighboring cell centers (or cell corners):

\begin{equation} \label{eq:neighboring_locations}
\begin{cases}
\vec{r}_{x^-} & = \vec{r} \,-\, \Delta x\,\normvec{x}\\[1ex]
\vec{r}_{x^+} & = \vec{r} \,+\, \Delta x\,\normvec{x}\\[1ex]
\vec{r}_{y^-} & = \vec{r} \,-\, \Delta y\,\normvec{y}\\[1ex]
\vec{r}_{y^+} & = \vec{r} \,+\, \Delta y\,\normvec{y}\\[1ex]
\vec{r}_{z^-} & = \vec{r} \,-\, \Delta z\,\normvec{z}\\[1ex]
\vec{r}_{z^+} & = \vec{r} \,+\, \Delta z\,\normvec{z}
\end{cases}\ ,
\end{equation}

%\begin{samepage}
where $\Delta x$, $\Delta y$ and $\Delta z$ are the grid spacings in the $\normvec{x}$, $\normvec{y}$ and $\normvec{z}$ directions respectively. By combining \eqrefs \ref{eq:phi_derivative_final}, \ref{eq:gradient_orthogonal} and \ref{eq:neighboring_locations}, we can write the gradient of $\phi$ as

\begin{equation} \label{eq:gradient_final}
\nabla\phi(\vec{r}) \,=\,
\frac{\phi(\vec{r}_{x^+})-\phi(\vec{r}_{x^-})}{2\,\Delta x}\,\normvec{x} \,+\,
\frac{\phi(\vec{r}_{y^+})-\phi(\vec{r}_{y^-})}{2\,\Delta y}\,\normvec{y} \,+\,
\frac{\phi(\vec{r}_{z^+})-\phi(\vec{r}_{z^-})}{2\,\Delta z}\,\normvec{z}.
\end{equation}
%\end{samepage}

\section{Pressure calculation}

Both the Euler equations and the Navier-Stokes equations come in two forms which usually differes significantly in implementation and stability: The \indexs{compressible Navier-Stokes}{equations}\indexs{compressible Euler}{equations}\compressible forms and the \indexs{incompressible Navier-Stokes}{equations}\indexs{incompressible Euler}{equations}\incompressible forms. For the work described in this report, the compressible Euler equations have been used. %TODO: Have they been fully solved? (advection of the velocity field)

\subsection{Compressible flow}

For \idxs{compressible}{flow}, the \divergence of the \idxs{velocity}{field} is allowed to be non-zero. The \pressure is then usually expressed purely as a function of the density; sometimes also taking into account the \temperature and other properties that may affect the pressure, that is

\begin{equation} \label{eq:pressure_compressible_flow}
p = p\,(\rho,\,T,\,\text{other material properties}),
\end{equation}

where $p$ is the pressure, $\rho$ is the density and $T$ is the temperature.

However, the set of equations including \eqref{eq:pressure_compressible_flow} is \idxse{stiff}{equation}{stiff}, and ordinary \idxsp{numerical}{method}{s} for solving this set of equations are known to give rise to \idxs{spurious}{oscillations} in the solutions when the \idxs{speed of}{sound} multiplied by the \idx{time step} becomes too large in relation to the \idxs{characteristic}{length} of the \cells, making the numerical methud \unstable.

Still, not all solvers for compressible flow suffers from this problem. In \citep{Kwatra2009}, this problem is remedied by introducing a \idxs{pressure}{field}, separated from the \idxe{density}{field}, and updating the pressure and velocity fields using what looks like the \index{implicit Euler method|see{backward Euler method}}\idxs{backward}{Euler method}, which leads to a \idxs{Poisson}{equation} for solving the pressure field. The remaining fields are then updated with the standard (forward) \idx{Euler method}. This technique prohobits \idx{A-stability} and therefore doesn't lead any spurious oscillations, but tends to \damp \idxse{high}{frequency}{high frequent} \idxsp{acoustic}{wave}{s} as the standard Euler method would instead make them grow in strenght. This technique is similar to the technique used for solving incompressible flow which also gives rise to a Poisson equation for solving the pressure field.

\subsection{Incompressible Navier-Stokes equations}

For \idxs{incompressible}{flow}, a different approach is taken to calculate the \idxs{pressure}{field}. Starting from the the \idxs{general form of the}{equations of fluid motion},

\begin{equation} \label{eq:navier_stokes_general}
\rho\left(\frac{\partial\vec{u}}{\partial t} + \vec{u}\cdot\nabla\vec{u}\right) = -\nabla p + \nabla\cdot\boldsymbol{\mathsf{T}} + \vec{f}
\end{equation}

where $\vec{u}$ is the velocity, $\boldsymbol{\mathsf{T}}$ is the \index{stress tensor|see{deviatoric stress tensor}}\idxs{deviatoric stress}{tensor} and $\vec{f}$ is the external forces per unit volume. Since the flow is incompressible, the \divergence of the \idxs{velocity}{field} $\vec{u}$ is zero, that is

\begin{equation} \label{eq:vecolity_divergence_incompressible_flow}
\nabla\cdot\vec{u} \,=\, 0,
\end{equation}

and the density is equal in all parts of the fluid, which means that we have

\begin{equation} \label{eq:density_gradient_incompressible_flow}
\nabla\rho \,=\, \vec{0},
\end{equation}

so \eqref{eq:navier_stokes_general} reduces to

\begin{equation} \label{eq:navier_stokes_general_simplified}
\frac{\partial\vec{u}}{\partial t} + \vec{u}\cdot\nabla\vec{u} = -\nabla p' + \nabla\cdot\boldsymbol{\mathsf{T}}' + \vec{f}',
\end{equation}

where $'$ denotes reduces forms, which are $1/\rho$ times their equivalents without the $'$ notation. As described in \citep{Losasso2004}, this \PDE can be solved it in two steps. First, an intermediate velocity field $\vec{u}^*$ is calculated ignoring the pressure term, and second

\begin{equation} \label{eq:navier_stokes_general_discretized}
\frac{\partial\vec{u}}{\partial t} + \vec{u}\cdot\nabla\vec{u} = -\nabla p' + \nabla\cdot\boldsymbol{\mathsf{T}}' + \vec{f}',
\end{equation}

\subsubsection{Iterative methods}

\paragraph{Gauss-Seidel method}

\subsubsection{Acceleration of iterative methods}

\paragraph{Preconditioned conjugate gradient method}

\begin{itemize}
    \item Extension of the gradient descent
\end{itemize}

See also \textit{Incomplete Cholesky Preconditioned Conjugate Gradients method}, described in \textit{\href{http://www.cs.ubc.ca/~rbridson/fluidbook/}{Fluid Simulation for Computer Graphics}}. This method uses the \textit{\href{http://en.wikipedia.org/wiki/Incomplete_Cholesky_factorization}{incomplete Cholesky factorization}} as preconditioner.

\paragraph{Multigrid method}

See
\begin{itemize}
    \item \textit{\href{http://developer.download.nvidia.com/books/cuda-by-example/cuda-by-example-sample.pdf}{CUDA by Example: An Introduction to General-Purpose GPU Programming}}
    \item \textit{\href{http://people.freebsd.org/~snb/school/hp_mg.pdf}{High Performance Multigrid for Poisson's Equation in 3D}}
\end{itemize}


\subsection{Semicompressible water}

\section{Boundary conditions}

\chapter{Octrees}

\section{Determining level of detail}

\section{The differentiating problem}

\subsection{Velocity advection term}

\section{Multilevel acceleration}

\section{Wave reflection at level transitions}

\chapter{Free Surface Modelling (FSM)}

There are both surface trackibng methods and surface capturing methods.

See also \textit{\href{http://physbam.stanford.edu/~fedkiw/papers/cam1998-17.pdf}{A Non-Oscillatory Eulerian Approach to Interfaces in Multimaterial Flows (The Ghost Fluid Method)}}

\begin{itemize}
    \item Lecture: \textit{\href{http://www.ims.nus.edu.sg/Programs/fluiddynamic/files/Lecture1-basics.pdf}{Moving Interface Problems: Methods \& Applications Tutorial Lecture I}}
\end{itemize}

\section{Level set method}

\subsection{Marching cubes}

\section{Volume of fluid method}

\begin{itemize}
    \item Reference: \textit{\href{http://pages.csam.montclair.edu/~yecko/icodes/HirtNichols_Surfer_JCP1981.pdf}{Volume of Fluid (VOF) Method for the Dynamics of Free Boundaries}}
    \item Comparsion: \textit{\href{http://capfluidicslit.mme.pdx.edu/reference/Numerics/Gopala_ChemEngJ2008_VOFMethodsFreeSurfaceFlow.pdf}{Volume of fluid methods for immiscible-fluid and free-surface flows}}
\end{itemize}

\subsection{VOF vs. Pseudo VOF}

\begin{itemize}
    \item Explanation: \textit{\href{http://www.flow3d.com/cfd-101/cfd-101-VOF.html}{VOF (Volume of Fluid) - What's in a Name?}}
\end{itemize}

\subsection{Interface reconstruction}
%\section{Internal alpha distribution}

\subsection{Smearing during advection}

\subsection{Geometric advection schemes}

\begin{itemize}
    \item A simple (at least so it seems) scheme: \textit{\href{http://www.lmm.jussieu.fr/~zaleski/nota02.pdf}{A geometrical area-preserving Volume-of-Fluid advection method}}
\end{itemize}

\subsection{Algebraic advection schemes}

\subsubsection{Convection Boundedness Criterion (CBC)}

\sloppy
\begin{itemize}
    \item Reference: \textit{Curvature-compensated Convective Transport: SMART a New Boundedness- Preserving Transport Algorithm}
    \item Extended Convective Boundedness Criterion (ECBC): \textit{Discussion on Numerical Stability and Boundedness of Convective Discretized Scheme}
    \item General Convective Boundedness Criterion (GCBC): \textit{\href{http://gr.xjtu.edu.cn:8080/upload/PUB.1673.4/Wei_NHT.pdf}{A New General Convective Boundedness Criterion}}
    \item Convection Boundedness Criterion for arbitrarily unstructured meshes: \textit{\href{http://powerlab.fsb.hr/ped/kturbo/openfoam/papers/GammaPaper.pdf}{High resolution NVD differencing scheme for arbitrarily unstructured meshes}}
\end{itemize}
\fussy

More:
\begin{itemize}
    \item Normalised Variable Diagram (NVD)
    \item \textit{\href{http://warminski.pollub.plwww.ptmts.org.pl/Waclaw-Koron-2-08.pdf}{Comparison of CICSAM and HRIC High-resolution Schemes for Interface Capturing}}
    \item \textit{\href{http://proceedings.fyper.com/eccomascfd2006/documents/85.pdf}{MODELING OF THE WAVE BREAKING WITH CICSAM AND HRIC HIGH-RESOLUTION SCHEMES}}
\end{itemize}

\subsubsection{Multidimensional Universal Limiter with Explicit Solution (MULES)}

See \textit{OpenFOAM-1.5.x/src/finiteVolume/fvMatrices/solvers/MULES/MULES.H} for details

% Escape characters
%\& \% \$ \# \_ \{ \}
%\textasciitilde  = ~
%\textasciicircum = ^
%\textbackslash   = \

\begin{itemize}
    \item Described here: \textit{\href{http://link.libris.kb.se/sfxliub?sid=?url_ver=Z39.88-2004&rfr_id=info:sid/bibl.liu.se\%3Axerxes+\%28+PubMed+LiU\%29&rft.genre=article&rft_val_fmt=info\%3Aofi\%2Ffmt\%3Akev\%3Amtx\%3Ajournal&rft.issn=15393755&rft.date=2009&rft.jtitle=Phys+Rev+E+Stat+Nonlin+Soft+Matter+Phys&rft.volume=79&rft.issue=3+Pt+2&rft.spage=036306&rft.atitle=Drop+impact+onto+a+liquid+layer+of+finite+thickness+\%3A+dynamics+of+the+cavity+evolution+&rft.aulast=Berberovi\%C4\%87&rft.aufirst=Edin}{Drop impact onto a liquid layer of finite thickness: Dynamics of the cavity evolution}}
    \item An improvement for more than two phases: \textit{\href{http://www.mathematik.uni-ulm.de/numerik/staff/urban/reports/ECCOMASCFD2010paperfinal.pdf}{A Coupled Pressure Based Solution Algorithm Based on the Volume-Of-Fluid Approach for Two or More Immiscible Fluids}}
\end{itemize}

\subsubsection{SOLA-VOF}

\begin{itemize}
    \item Reference: \textit{\href{http://www.ewp.rpi.edu/hartford/~ernesto/Su2012/CFD/Readings/SOLA-VOF-1980-P1.pdf}{SOLA-VOF: A Solution Algorithm for Transient Fluid Flow with Multiple Free Boundaries}}
\end{itemize}

\subsubsection{Hyper-C flux limiter}

\begin{itemize}
    \item Reference: \textit{\href{http://www.water.tkk.fi/wr/kurssit/Yhd-12.112/TVD1.pdf}{The Ultimate Conservative Difference Scheme Applied to Unsteady One-Dimensional Advection}}
\end{itemize}

\paragraph{Floating mixed cells}

\begin{itemize}
    \item Remedy: \textit{\href{https://e-reports-ext.llnl.gov/pdf/245038.pdf}{A Simple Advection Scheme for Material Interface}}
\end{itemize}

\subsubsection{Compressive Interface Capturing Scheme for Arbitrary Meshes (CICSAM)}

\begin{itemize}
    \item Reference: \textit{\href{http://ac.els-cdn.com/S0021999199962769/1-s2.0-S0021999199962769-main.pdf?_tid=85161b57da5f4401e55c9d07495e24ea&acdnat=1336167249_a59e4f578adbacf3bff69936c48cdd57}{A Method for Capturing Sharp Fluid Interfaces on Arbitrary Meshes}}
    \item Also described in (by the same author): \textit{\href{http://powerlab.fsb.hr/ped/kturbo/OpenFOAM/docs/OnnoUbbinkPhD.pdf}{Numerical prediction of two fluid systems with sharp interfaces}}
    \item Test with different Courant numbers: \textit{\href{http://www.marin.nl/upload_mm/8/2/c/1807524470_1999999096_2007-ECCOMAS_HoekstraVazAbeilBunnik.pdf}{Free Surface Flow Modelling with Interface Capturing Techniques}}
    \item Improvement 1: \textit{\href{http://powerlab.fsb.hr/ped/kturbo/openfoam/docs/HenrikRuschePhD2002.pdf}{Computational Fluid Dynamics of Dispersed Two-Phase Flows at High Phase Fractions}}
\end{itemize}

\subsubsection{High Resolution Interface Capturing (HRIC) scheme}

\begin{itemize}
    \item Described here: \textit{\href{http://warminski.pollub.plwww.ptmts.org.pl/Waclaw-Koron-2-08.pdf}{Comparison of CICSAM and HRIC High-resolution Sche\-mes for Interface Capturing}}
\end{itemize}

\subsubsection{Switching Technique for Advection and Capturing of Surfaces scheme (STACS)}

\begin{itemize}
    \item Reference: \textit{\href{http://webfea-lb.fea.aub.edu.lb/cfd/pdfs/publications2/STACS-Complete.pdf}{Convective Schemes for Capturing Interfaces of Free-Surface Flows on Unstructured Grids}}
\end{itemize}

\subsubsection{Inter-Gamma Scheme}

\begin{itemize}
    \item Reference: \textit{\href{http://powerlab.fsb.hr/ped/kturbo/openfoam/docs/InterTrack.pdf}{Interface Tracking Capabilities of the Inter-Gamma Differencing Scheme}}
\end{itemize}

\subsubsection{Constrained Interpolation Profile (CIP) method}

%TODO: Used for advecting fluid interfaces?? At least apparently very good for simple advection.

\begin{itemize}
    \item Reference: \textit{\href{http://www.mech.titech.ac.jp/~ryuutai/paper/JCP2001CIPReviewYabe.pdf}{The Constrained Interpolation Profile Method for Multiphase Analysis}}
\end{itemize}

\subsection{Advection schemes for compressible water}

\begin{itemize}
    \item Remedy: Advect both water volume and total volume and then define alpha as the ration between them
\end{itemize}

\subsubsection{Fast Compressive Surface Capturing Formulation (FCSCF)}

\begin{itemize}
    \item Reference: \textit{\href{http://researchspace.csir.co.za/dspace/bitstream/10204/5282/1/Heyns_2011.pdf}{Free-Surface Modelling Technology for Compressible and Violent Flows}}
\end{itemize}

\section{Coupled Level Set/Volume of Fluid  method}

\begin{itemize}
    \item Reference: \textit{\href{http://pages.csam.montclair.edu/~yecko/icodes/SussmanPuckett_LevelSetVOF.pdf}{A Coupled Level Set and Volume-of-Fluid Method for Computing 3D and Axisymmetric Incompressible Two-Phase Flows}}
\end{itemize}

\chapter{Interaction with moving objects}

see e.g.:

\begin{itemize}
    \item \textit{\href{http://physbam.stanford.edu/~fedkiw/papers/stanford2010-04.pdf}{Numerically Stable Fluid-Structure Interactions Between Compressible Flow and Solid Structures}}
    \item \textit{\href{http://efdl.as.ntu.edu.tw/research/papers/JCP03GCIBM.pdf}{A ghost-cell immersed boundary method for flow in complex geometry}}
    \item \textit{\href{http://www.cs.columbia.edu/~batty/papers/Batty07.pdf}{A Fast Variational Framework for Accurate Solid-Fluid Coupling}} (solid fraction, non-stick to walls)
\end{itemize}

\section{Immersed Boundary Method}

\begin{itemize}
    \item Reference: \textit{\href{http://www4.ncsu.edu/~zhilin/TEACHING/MA798Z/Peskin1.pdf}{The immersed boundary method}}
    \item For compressible flow: \textit{\href{http://www.cecs.wright.edu/~haibo.dong/wp-content/themes/publications/IBM_JCP_2007.pdf}{A sharp interface immersed boundary method for compressible viscous flows}}
\end{itemize}

\section{Volume of Solid Method (VOS)}

\begin{itemize}
    \item Reference: \textit{The simulation of fluid-rigid body interaction}
    \item Described in \textit{Numerical Modeling of Deforming Bubble Transport Related to Cavitating Hydraulic Turbines}
\end{itemize}

\section{Rotation of rigid bodies}

See \textit{\href{http://en.wikipedia.org/wiki/Euler\%27s_equations_\%28rigid_body_dynamics\%29}{Euler's equations (rigid body dynamics)}}

\chapter{Visualization}

\chapter{Conservation laws}

\begin{itemize}
    \item Mass
    \item Energy
    \item Momentum
    \item Angular momentum
    \item (d/dt)(Center of mass) - momentum = 0
\end{itemize}