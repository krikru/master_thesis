%\DeclareOption{s5paper}
%    {\setlength\paperheight {240mm}%
%     \setlength\paperwidth  {165mm}}
% S5: 165 x 242 mm (This is also called statsformat (in Swedish), it.s a special government format used in Sweden) 
% Standard format for master thesis works, according to LIU-tryck

\documentclass[]{report}
%\documentclass[a4paper]{report}
%\documentclass[s5paper]{report}

% Available sectioning commands for report: \part{}, \part{}, \chapter{}, \section{}, \subsection{}, \paragraph{}, \subparagraph{}.

% "-": Hyphen, used to join words and to separate syllables in words
% "$-$": Minus
% "--": En-dash, used for number ranges, to mark off nested clausees and phrases (use)
% "---": Em-dash, used to mark off nested clausees and phrases (don't use)

% Escape characters
%\& \% \$ \# \_ \{ \}
%\textasciitilde  = ~
%\textasciicircum = ^
%\textbackslash   = \


% % % % % % %
% PREAMBLE  %
% % % % % % %

% Packages

% Miscellaneous tools
\usepackage{etoolbox} % Includes the macro \ifstrequal
% Paper size
\iffalse
    \usepackage{./Packages/s5paper/s5paper}
\fi
% Linking
\iftrue
    \usepackage{hyperref} % Links to pages
    \hypersetup{
        pdfborder = {0 0 0},
        colorlinks=true,
        citecolor=violet,
        linkcolor=black,
        urlcolor=blue,
        } % Remove the frame around links
\else
    \usepackage{url} % Doesn't link to pages
\fi
% Graphics
\usepackage{graphicx} % Necessary in order to include images
\usepackage{titlepic} % Necessary in order to be able to have an image on the title page
% Referencing
\iftrue
    \usepackage[round, authoryear]{natbib} % Harward style referencing
\else
    \usepackage[square, numbers]{natbib} %
\fi

%Define commands
\newcommand{\comment}[1]{}
\newcommand{\HRule}{\rule{\linewidth}{0.5mm}}

\begin{document}

% % % % % % % %
% TOP MATTER  %
% % % % % % % %

%\begin{titlepage}
% Top matter
\title{Simulating Ocean Waves and Ships on a Tree Structure}
\author{Kristofer Krus}
\date{\today}
\titlepic{\includegraphics[width=0.6\textwidth, clip=true]{./Images/lith_en_vert_col.pdf}}
%\includegraphics[width=0.6\textwidth, clip=true]{./Images/lith_en_vert_col.pdf}
\maketitle
%\texttt{krikr808@student.liu.se}
%\end{titlepage}

% % % % % % % % % %
% COMPYRIGHT NOTE %
% % % % % % % % % %

\section*{Copyright}
\iffalse
\thispagestyle{empty}
\fi

The publishers will keep this document online on the Internet -- or its possible replacement -- for a period of 25 years starting from the date of publication barring exceptional circumstances.

The online availability of the document implies permanent permission for anyone to read, to download, or to print out single copies for his/hers own use and to use it unchanged for non-commercial research and educational purpose. Subsequent transfers of copyright cannot revoke this permission. All other uses of the document are conditional upon the consent of the copyright owner. The publisher has taken technical and administrative measures to assure authenticity, security and accessibility.

According to intellectual property law the author has the right to be mentioned when his/her work is accessed as described above and to be protected against infringement.

For additional information about the Linköping University Electronic Press and its procedures for publication and for assurance of document integrity, please refer to its www home page: \url{http://www.ep.liu.se/}.
%\hypersetup{colorlinks=true, urlcolor=cyan} % Remove the frame around links

% % % % % % %
% ABSTRACT  %
% % % % % % %

\begin{abstract}
    Your abstract goes here...
\end{abstract}

% Table of contents and various lists
\tableofcontents
\listoftables
\listoffigures

% % % % % % % % %
% INTRODUCTION  %
% % % % % % % % %

\part{Introduction}

\chapter{Motivation}

\chapter{Difficulties}

\begin{itemize}
    \item Wave dispersion
    \item Different depths gives different speeds
    \item The water should interact with moving objects
\end{itemize}

\chapter{Earlier approaches}

\section{Twodimensional methods}

\subsection{Twodimensional PDEs for shallow water}

\subsection{Spectral methods}

\begin{itemize}
    \item Reference 1: \textit{\href{http://web1.see.asso.fr/ocoss2010/Session_4/20100531111216_Monnier_OCOSS2010-Paper_MERCUDA_item_2.pdf}{Real time modelling of multispectral ocean scenes}}
    \item Reference 2: \textit{GPU-based simulation of Radar sea clutter}
\end{itemize}

\section{Threedimensional methods}

\subsection{Particles}

\subsection{Marker-and-Cell method (MAC)}

\begin{itemize}
    \item Reference: \textit{Numerical calculation of time-dependent viscous incompressible flow of fluid with a free surface}
    \item Described in: \textit{\href{http://people.sc.fsu.edu/~jburkardt/pdf/fluid_flow_for_the_rest_of_us.pdf}{Fluid Flow for the Rest of Us: Tutorial of the Marker and Cell Method in Computer Graphics}}
\end{itemize}

\subsection{Boundary element method}

\subsection{Finite element method}

\subsection{Finite volume method with tall cells}

\subsection{Finite volume method with octrees}

\begin{itemize}
    \item Reference 1: \textit{\href{http://gfs.sourceforge.net/gerris.pdf}{Gerris: a tree-based adaptive solver for the incompressible Euler equations in complex geometries}}
    \item Reference 2: \textit{\href{http://physbam.stanford.edu/~fedkiw/papers/stanford2004-02.pdf}{Simulating Water and Smoke with an Octree Data Structure}}
\end{itemize}

\section{Twodimensional and threedimensional hybrid methods}

% % % % % %
% METHOD  %
% % % % % %

\part{Method}

\chapter{Finite volume method}

\section{Artificial compression}

\subsection{Instability}

\section{Incompressible Navier-Stokes equations}

\subsection{Iterative methods}

\subsubsection{Gauss-Seidel method}

\subsection{Acceleration of iterative methods}

\subsubsection{Preconditioned conjugate gradient method}

\begin{itemize}
    \item Extension of the gradient descent
\end{itemize}

See also \textit{Incomplete Cholesky Preconditioned Conjugate Gradients method}, described in \textit{\href{http://www.cs.ubc.ca/~rbridson/fluidbook/}{Fluid Simulation for Computer Graphics}}. This method uses the \textit{\href{http://en.wikipedia.org/wiki/Incomplete_Cholesky_factorization}{incomplete Cholesky factorization}} as preconditioner.

\subsubsection{Multigrid method}

See \textit{\href{http://developer.download.nvidia.com/books/cuda-by-example/cuda-by-example-sample.pdf}{CUDA by Example: An Introduction to General-Purpose GPU Programming}}

\section{Semicompressible water}

\section{Boundary conditions}

\chapter{Octrees}

\section{Determining level of detail}

\section{The differentiating problem}

\subsection{Velocity advection term}

\section{Multilevel acceleration}

\section{Wave reflection at level transitions}

\chapter{Free Surface Modelling (FSM)}

There are both surface trackibng methods and surface capturing methods.

See also \textit{\href{http://physbam.stanford.edu/~fedkiw/papers/cam1998-17.pdf}{A Non-Oscillatory Eulerian Approach to Interfaces in Multimaterial Flows (The Ghost Fluid Method)}}

\begin{itemize}
    \item Lecture: \textit{\href{http://www.ims.nus.edu.sg/Programs/fluiddynamic/files/Lecture1-basics.pdf}{Moving Interface Problems: Methods \& Applications Tutorial Lecture I}}
\end{itemize}

\section{Level set method}

\subsection{Marching cubes}

\section{Volume of fluid method}

\begin{itemize}
    \item Reference: \textit{\href{http://pages.csam.montclair.edu/~yecko/icodes/HirtNichols_Surfer_JCP1981.pdf}{Volume of Fluid (VOF) Method for the Dynamics of Free Boundaries}}
    \item Comparsion: \textit{\href{http://capfluidicslit.mme.pdx.edu/reference/Numerics/Gopala_ChemEngJ2008_VOFMethodsFreeSurfaceFlow.pdf}{Volume of fluid methods for immiscible-fluid and free-surface flows}}
\end{itemize}

\subsection{VOF vs. Pseudo VOF}

\begin{itemize}
    \item Explanation: \textit{\href{http://www.flow3d.com/cfd-101/cfd-101-VOF.html}{VOF (Volume of Fluid) - What's in a Name?}}
\end{itemize}

\subsection{Interface reconstruction}
%\section{Internal alpha distribution}

\subsection{Smearing during advection}

\subsection{Geometric advection schemes}

\begin{itemize}
    \item A simple (at least so it seems) scheme: \textit{\href{http://www.lmm.jussieu.fr/~zaleski/nota02.pdf}{A geometrical area-preserving Volume-of-Fluid advection method}}
\end{itemize}

\subsection{Algebraic advection schemes}

\subsubsection{Convection Boundedness Criterion (CBC)}

\sloppy
\begin{itemize}
    \item Reference: \textit{Curvature-compensated Convective Transport: SMART a New Boundedness- Preserving Transport Algorithm}
    \item Extended Convective Boundedness Criterion (ECBC): \textit{Discussion on Numerical Stability and Boundedness of Convective Discretized Scheme}
    \item General Convective Boundedness Criterion (GCBC): \textit{\href{http://gr.xjtu.edu.cn:8080/upload/PUB.1673.4/Wei_NHT.pdf}{A New General Convective Boundedness Criterion}}
    \item Convection Boundedness Criterion for arbitrarily unstructured meshes: \textit{\href{http://powerlab.fsb.hr/ped/kturbo/openfoam/papers/GammaPaper.pdf}{High resolution NVD differencing scheme for arbitrarily unstructured meshes}}
\end{itemize}
\fussy

More:
\begin{itemize}
    \item Normalised Variable Diagram (NVD)
    \item \textit{\href{http://warminski.pollub.plwww.ptmts.org.pl/Waclaw-Koron-2-08.pdf}{Comparison of CICSAM and HRIC High-resolution Schemes for Interface Capturing}}
    \item \textit{\href{http://proceedings.fyper.com/eccomascfd2006/documents/85.pdf}{MODELING OF THE WAVE BREAKING WITH CICSAM AND HRIC HIGH-RESOLUTION SCHEMES}}
\end{itemize}

\subsubsection{Multidimensional Universal Limiter with Explicit Solution (MULES)}

See \textit{OpenFOAM-1.5.x/src/finiteVolume/fvMatrices/solvers/MULES/MULES.H} for details

% Escape characters
%\& \% \$ \# \_ \{ \}
%\textasciitilde  = ~
%\textasciicircum = ^
%\textbackslash   = \

\begin{itemize}
    \item Described here: \textit{\href{http://link.libris.kb.se/sfxliub?sid=?url_ver=Z39.88-2004&rfr_id=info:sid/bibl.liu.se\%3Axerxes+\%28+PubMed+LiU\%29&rft.genre=article&rft_val_fmt=info\%3Aofi\%2Ffmt\%3Akev\%3Amtx\%3Ajournal&rft.issn=15393755&rft.date=2009&rft.jtitle=Phys+Rev+E+Stat+Nonlin+Soft+Matter+Phys&rft.volume=79&rft.issue=3+Pt+2&rft.spage=036306&rft.atitle=Drop+impact+onto+a+liquid+layer+of+finite+thickness+\%3A+dynamics+of+the+cavity+evolution+&rft.aulast=Berberovi\%C4\%87&rft.aufirst=Edin}{Drop impact onto a liquid layer of finite thickness: Dynamics of the cavity evolution}}
    \item An improvement for more than two phases: \textit{\href{http://www.mathematik.uni-ulm.de/numerik/staff/urban/reports/ECCOMASCFD2010paperfinal.pdf}{A Coupled Pressure Based Solution Algorithm Based on the Volume-Of-Fluid Approach for Two or More Immiscible Fluids}}
\end{itemize}

\subsubsection{SOLA-VOF}

\begin{itemize}
    \item Reference: \textit{\href{http://www.ewp.rpi.edu/hartford/~ernesto/Su2012/CFD/Readings/SOLA-VOF-1980-P1.pdf}{SOLA-VOF: A Solution Algorithm for Transient Fluid Flow with Multiple Free Boundaries}}
\end{itemize}

\subsubsection{Hyper-C flux limiter}

\begin{itemize}
    \item Reference: \textit{\href{http://www.water.tkk.fi/wr/kurssit/Yhd-12.112/TVD1.pdf}{The Ultimate Conservative Difference Scheme Applied to Unsteady One-Dimensional Advection}}
\end{itemize}

\paragraph{Floating mixed cells}

\begin{itemize}
    \item Remedy: \textit{\href{https://e-reports-ext.llnl.gov/pdf/245038.pdf}{A Simple Advection Scheme for Material Interface}}
\end{itemize}

\subsubsection{Compressive Interface Capturing Scheme for Arbitrary Meshes (CICSAM)}

\begin{itemize}
    \item Reference: \textit{\href{http://ac.els-cdn.com/S0021999199962769/1-s2.0-S0021999199962769-main.pdf?_tid=85161b57da5f4401e55c9d07495e24ea&acdnat=1336167249_a59e4f578adbacf3bff69936c48cdd57}{A Method for Capturing Sharp Fluid Interfaces on Arbitrary Meshes}}
    \item Also described in (by the same author): \textit{\href{http://powerlab.fsb.hr/ped/kturbo/OpenFOAM/docs/OnnoUbbinkPhD.pdf}{Numerical prediction of two fluid systems with sharp interfaces}}
    \item Test with different Courant numbers: \textit{\href{http://www.marin.nl/upload_mm/8/2/c/1807524470_1999999096_2007-ECCOMAS_HoekstraVazAbeilBunnik.pdf}{Free Surface Flow Modelling with Interface Capturing Techniques}}
    \item Improvement 1: \textit{\href{http://powerlab.fsb.hr/ped/kturbo/openfoam/docs/HenrikRuschePhD2002.pdf}{Computational Fluid Dynamics of Dispersed Two-Phase Flows at High Phase Fractions}}
\end{itemize}

\subsubsection{High Resolution Interface Capturing (HRIC) scheme}

\begin{itemize}
    \item Described here: \textit{\href{http://warminski.pollub.plwww.ptmts.org.pl/Waclaw-Koron-2-08.pdf}{Comparison of CICSAM and HRIC High-resolution Sche\-mes for Interface Capturing}}
\end{itemize}

\subsubsection{Switching Technique for Advection and Capturing of Surfaces scheme (STACS)}

\begin{itemize}
    \item Reference: \textit{\href{http://webfea-lb.fea.aub.edu.lb/cfd/pdfs/publications2/STACS-Complete.pdf}{Convective Schemes for Capturing Interfaces of Free-Surface Flows on Unstructured Grids}}
\end{itemize}

\subsubsection{Inter-Gamma Scheme}

\begin{itemize}
    \item Reference: \textit{\href{http://powerlab.fsb.hr/ped/kturbo/openfoam/docs/InterTrack.pdf}{Interface Tracking Capabilities of the Inter-Gamma Differencing Scheme}}
\end{itemize}

\subsubsection{Constrained Interpolation Profile (CIP) method}

%TODO: Used for advecting fluid interfaces?? At least apparently very good for simple advection.

\begin{itemize}
    \item Reference: \textit{\href{http://www.mech.titech.ac.jp/~ryuutai/paper/JCP2001CIPReviewYabe.pdf}{The Constrained Interpolation Profile Method for Multiphase Analysis}}
\end{itemize}

\subsection{Advection schemes for compressible water}

\begin{itemize}
    \item Remedy: Advect both water volume and total volume and then define alpha as the ration between them
\end{itemize}

\subsubsection{Fast Compressive Surface Capturing Formulation (FCSCF)}

\begin{itemize}
    \item Reference: \textit{\href{http://researchspace.csir.co.za/dspace/bitstream/10204/5282/1/Heyns_2011.pdf}{Free-Surface Modelling Technology for Compressible and Violent Flows}}
\end{itemize}

\section{Coupled Level Set/Volume of Fluid  method}

\begin{itemize}
    \item Reference: \textit{\href{http://pages.csam.montclair.edu/~yecko/icodes/SussmanPuckett_LevelSetVOF.pdf}{A Coupled Level Set and Volume-of-Fluid Method for Computing 3D and Axisymmetric Incompressible Two-Phase Flows}}
\end{itemize}

\chapter{Interaction with moving objects}

see e.g.:

\begin{itemize}
    \item \textit{\href{http://physbam.stanford.edu/~fedkiw/papers/stanford2010-04.pdf}{Numerically Stable Fluid-Structure Interactions Between Compressible Flow and Solid Structures}}
    \item \textit{\href{http://efdl.as.ntu.edu.tw/research/papers/JCP03GCIBM.pdf}{A ghost-cell immersed boundary method for flow in complex geometry}}
    \item \textit{\href{http://www.cs.columbia.edu/~batty/papers/Batty07.pdf}{A Fast Variational Framework for Accurate Solid-Fluid Coupling}} (solid fraction, non-stick to walls)
\end{itemize}

\section{Immersed Boundary Method}

\begin{itemize}
    \item Reference: \textit{\href{http://www4.ncsu.edu/~zhilin/TEACHING/MA798Z/Peskin1.pdf}{The immersed boundary method}}
    \item For compressible flow: \textit{\href{http://www.cecs.wright.edu/~haibo.dong/wp-content/themes/publications/IBM_JCP_2007.pdf}{A sharp interface immersed boundary method for compressible viscous flows}}
\end{itemize}

\section{Volume of Solid Method (VOS)}

\begin{itemize}
    \item Reference: \textit{The simulation of fluid-rigid body interaction}
    \item Described in \textit{Numerical Modeling of Deforming Bubble Transport Related to Cavitating Hydraulic Turbines}
\end{itemize}

\section{Rotation of rigid bodies}

See \textit{\href{http://en.wikipedia.org/wiki/Euler\%27s_equations_\%28rigid_body_dynamics\%29}{Euler's equations (rigid body dynamics)}}

\chapter{Conservation laws}

\begin{itemize}
    \item Mass
    \item Energy
    \item Momentum
    \item Angular momentum
    \item (d/dt)(Center of mass) - momentum = 0
\end{itemize}

% % % % % % % % % % % % % %
% RESULTS AND CONCLUSIONS %
% % % % % % % % % % % % % %

\part{Results and discussion}

\chapter{Results}

\section{Speed}

\section{Accuracy}

\chapter{Discussion}

\section{Conclusions}

\subsection{Difficulties and drawbacks with the method}

Difficulties:
\begin{itemize}
    \item Dynamical creation/termination of surface cells and determination of properties in new cells
    \item High Courant numbers
    \item High speeds of sound (remedied in \textit{\href{http://physbam.stanford.edu/~kwatra/papers/compressible_semi_implicit/compressible_semi_implicit.pdf}{A method for avoiding the acoustic time step restriction in compressible flow}})
    \item Keeping a sharp interface
    \item Making it work in realtime
\end{itemize}

\chapter{Improvements}

\section{Wind waves}

\subsection{Spectral methods}

\subsection{Air-water interaction}

\section{Splash and foam}

See \textit{\href{http://en.wikipedia.org/wiki/Sea_foam}{Wikipedia -- Sea foam}} or search for \textit{protein skimming} or \textit{foam fractionation}

\section{Parallellization}

\subsection{Space filling curves}

\section{Local-time stepping}

\section{Remedy for regions with high Courant number}

\section{Perfectly matched layers}

\section{Incompressibility or semi-incompressibility}

\begin{itemize}
    \item See: \textit{\href{http://physbam.stanford.edu/~kwatra/papers/compressible_semi_implicit/compressible_semi_implicit.pdf}{A method for avoiding the acoustic time step restriction in compressible flow}}
\end{itemize}

\section{Sharpening of various advected fields}

\subsection{Backward Error Compensation and Forward Error Correction}

\begin{itemize}
    \item Reference: \textit{\href{http://smartech.gatech.edu/xmlui/bitstream/handle/1853/29473/2002-389.pdf}{Back and forth error compensation and correction methods for removing errors induced by uneven gradients of the level set function}}
    \item Applied to the velocity field and images: \textit{\href{http://www.gvu.gatech.edu/~jarek/papers/FlowFixer.pdf}{FlowFixer: Using BFECC for Fluid Simulation}}
\end{itemize}

\section{Code optimization}

% % % % % % %
% APENDIZES %
% % % % % % %

\part{Appendixes}

\appendix
%\chapter{Derivation of twodimensional PDEs for water waves at random intermediate depths}
\chapter{Derivation of twodimensional PDEs for water waves at varying depths}

% % % % % % % % %
% BIBLIOGRAPHY  %
% % % % % % % % %

\iffalse

\bibliographystyle{plain}
\bibliography{sample1,sample2,...,samplen}
% Note the lack of whitespace between the commas and the next bib file.

\else

\begin{thebibliography}{99}
    
    %TODO: Remove this bibitem
    \bibitem{lamport94}
    Leslie Lamport,
    \emph{\LaTeX: A Document Preparation System}.
    Addison Wesley, Massachusetts,
    2nd Edition,
    1994.
    
\end{thebibliography}

\fi

\end{document}

%TODO: Check up:
% Turbulens model (Large Eddy Simulation for example)

%TODO: Mention Zalesak’s problem (for advection tests of sharp interfaces)
