\chapter{Conclusions}
\label{chap:conclusions}

The training in a flight simulator would become more valuable if a realistic, coupled simulation of water waves and ship movement was integrated into the flight simulator. It is therefore required that the wave simulation runs in real-time, has a resolution high enough to represent waves that normally roam the water surfaces and affect the ships as well as the wake lines that sailing ships give rise to. It should also manage to simulate wave dispersion, in at least an approximate way, as well as some form of \FSI to get the coupling between ship and waves.

The method that was implemented in this thesis work was the \FVM on an octree grid with \FSM, which is an advanced method that takes a long time to implement and to make ideal for the purposes of this thesis work, and the implementation that was developed in this thesis work would still need a lot of improvements if it was to be applied in a real-time simulator. For example, it suffered from instabilities in the air phase, that easily arose when the density became too low, and there were problems with keeping the interface between the air phase and the water phase intact as it tended to grow, which introduces several new problems.

Furthermore, the \timestep was severely restricted by the \CFL condition, which would have to be alleviated by using the semi-Lagrangian scheme in order to make arbitrarily large \timesteps possible. Adaptive refinement of the octree would also have to be implemented in order to keep the number of cells that have to be processed in each \timestep down on a reasonable level.

The \FVM on an octree with \FSM runs in $O(N)$ time, where $N$ is the number of grid points visible on the surface, which places it among the methods with the lowest time complexity. However, it has a very high time constant associated with the big O notation which, as of today, makes it too slow to be used in real-time applications.

Probably the only two of the methods that were studied in this thesis work, that today easily can be constructed to meet all the required properties, are the Fourier synthesis method and the two-dimensional method using \LPD. If it is important that wave shoaling is simulated correctly, or water flowing down a stream, the Fourier synthesis method should be avoided as it can't cope with those cases.

Alternatively, a two-dimensional simulation, using \LPD or Fourier synthesis, can be coupled with a local three-dimensional fluid simulation around each ship to allow for strong non-linear phenomenas such as splashes and more natural \FSI, as well as a fast simulation of \idxs{ambient}{waves}, provided the three-dimensional method is fast enough.

However, it is believed that, after some improvement of the method that was implemented in this thesis work, basically the method's only bottleneck would be its low speed, which to its defense can be alleviated by parallelizing the code and spread the computational load on many \CPUs, probably even on \GPUs. And with the continuous increase in processor power it is only a matter of time before the method can run in real-time with an acceptable quality. It will therefore sooner or later come to compete with methods such as Fourier synthesis and the \LPD method, and will eventually even surpass them even for real-time simulations as it models the behavior of water much more accurately.