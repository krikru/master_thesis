\chapter{Derivation of an illumination model for sea surfaces}
\label{chap:illumination_model_derivation}

%To be written \comment
{
If we simplify a water surface by cutting off a part of the \idxs{wave}{spectrum}, we need to --- for \rendering purposes --- compensate for that by making the \reflection in the \idxs{simplified}{water surface} \idxse{diffuse}{reflection}{diffuse}. The \idxs{illumination}{model} derived in this appendix models this diffusion and was developed outside of the scope of \thismasterthesiswork.

\section{Microfacet distribution}

Although the high frequencies have been removed from the rendering, which makes it impossible to treat them directly, it is possible to treat them statistically by making a few \assumptions and \approximations, and by using something known as \microfacets, which are basically infinitesimal surface elements for which the \idxs{surface}{normal} is \stochastic but \assumed to have a known \idxs{probability}{distribution}, tiled one after each other.

\section{The rendering equation}

When doing \idxs{physically based}{rendering}, one often starts from an equation known as the \idxs{rendering}{equation}, which was simultaneously introduced in \idxs{computer}{graphics} in \citep{temp} and \citep{temp}, and is given as

\begin{equation} \label{eq:rendering_equation_original}
\begin{array}{c}
L_{\text{o}}(\vec{r},\,\normvec{\omega}_{\text{o}},\, \lambda,\, t) \,= \\
\displaystyle L_{\text{e}}(\vec{r},\, \normvec{\omega}_{\text{o}},\, \lambda,\, t) \ + \ \int_\Omega \rho'_{\textbf{r}}(\vec{r},\, \normvec{\omega}_{\text{i}},\, \normvec{\omega}_{\text{o}},\, \lambda,\, t)\, L_{\text{i}}(\vec{r},\, \normvec{\omega}_{\text{i}},\, \lambda,\, t)\, (\normvec{\omega}_{\text{i}}\,\cdot\,\normvec{n})\, \operatorname d \normvec{\omega}_{\text{i}},
\end{array}
\end{equation}

where $L_{\text{o}}$ is the total \idxs{spectral}{radiance} of \index{light!wavelength}\idxe{wavelength!light}{wavelength} $\lambda$ directed outward along direction $\normvec{\omega}_{\text{o}}$ at time $t$, from location $\vec{r}$ at the surface; $L_{\text{e}}$ is the spectral radiance emitted by the surface itself, $\rho'_{\textbf{r}}$ is the \BRDF which was first defined in \citep{temp} and describes how light from different directions are reflected on the surface, $L_{\text{i}}$ is the \idxs{spectral}{radiance} incoming from direction $\normvec{\omega}_{\text{i}}$, $\normvec{n}$ is the \idxs{surface}{normal}, and $\Omega$ is the \idxs{unit}{hemisphere} containing all possible directions for the incoming reflected light, and thus all possible values for $\normvec{\omega}_{\textbf{i}}$ (but also all possible values for $\normvec{\omega}_{\textbf{o}}$). Note that both $\normvec{\omega}_{\text{i}}$ and $\normvec{\omega}_{\text{o}}$ are directed outwards from the surface.

While this equation can lay the foundation for many very realistic renderings of \threedimensional scenes, and can make the reflective properties of a surface vary for different positions, for different points in time, for different wavelengths, which allows the surface to have a color and not just a gray-scale, and even for different rotations of the surface, there are several aspects of light it can't grasp. Some of these aspects, which are relevant to \surfacewaterrendering, include

\begin{itemize}
\item \textbf{\idxe{polarisation}{Polarization}:} Light polarized differently will sometimes have different reflection distributions, as in the case of light being reflected at a water surface.

\item \textbf{\idxe{transmission}{Transmission}:} Occurs when light is transmitted through the surface, as when it hits a glass object or a water surface.

\item \textbf{\idxse{subsurface}{scattering}{Subsurface scattering}:} Many materials exhibits the property that much of the incoming light is transmitted through the surface at one location, scattered, and transmitted back through the surface at a slightly different location. If such a material is rendered without taking subsurface scattering into account, it may appear plastic, and sometimes also unnaturally opaque.

However, it is not necessary to account specifically for this in the rendering equation if it includes transmission, since that will effectively also include light scattered under the surface, even if the rendering equation still doesn't provide a model for how the light is scattering under the surface.
\end{itemize}

Of these aspects, polarization and transmission are the two aspects that are the easiest to model. Subsurface scattering also plays a role, though, but it is not going to be modeled in this appendix.

Other aspects of light, which are not relevant to \surfacewaterrendering (but are still included in this report for leisure reading), include

\begin{itemize}
\item \textbf{\idxe{phosphorescence}{Phosphorescence}:} Light or other electromagnetic radiation is sometimes absorbed at one point in time and emitted at a later point in time, usually with a lower frequency (unless the absorbed electromagnetic radiation is very intense).

If the absorption and the emission occurs at the same point in time, but with different frequencies, this is called \idx{fluorescence}.
    
\item \textbf{\idxe{interference}{Interference}:} This can occur if the wave properties of light are exhibited, for example when light is passing though a \idxs{thin}{slit} or a \idxs{double}{slit}.
    
\item \textbf{\idxse{non-linear}{effect}{Non-linear effects}:} If the light is very intensive, two or more photons can sometimes hit the same electron in a material at the same time, increasing the energy of the electron with more than the energy of the individual photons. When the electron makes a transition back to a lower energy level, emission of a photon with a higher frequency is possible.
    
\item \textbf{\index{\idxse{effect!relativistic Doppler|see{relativistic Doppler effect}}}\idxse{relativistic}{Doppler effect}{Relativistic Doppler effect}:} Light that is reflected on an object that is moving with a very high speed relative to the reference frame (or to something that is observing the light) will get its wavelength changed. If the light is reflected on an object that is moving towards it, the impact will compress the photons, making the wavelength shorter which in turn makes the light blueshifted. The photons will also be packed more closely, so the photon flux will be increased. If the light instead is reflected on an object that is moving away from it, the opposite thing will happen.
\end{itemize}

To account for transmission and polarization, we will modify \eqref{eq:rendering_equation_original} into

\begin{equation} \label{eq:rendering_equation_improved}
\renewcommand*{\arraystretch}{1.5}
\begin{array}{cl}
L_{\text{o}}(\vec{r},\,\normvec{\omega}_{\text{o}},\, \lambda,\, t) \,=\, L_{\text{e}}(\vec{r},\, \normvec{\omega}_{\text{o}},\, \lambda,\, t) \ + \\
\displaystyle \int_{\Omega_{\textbf{r}}} \rho'_{\textbf{r}}(\vec{r},\, \normvec{\omega}_{\text{i}},\, \normvec{\omega}_{\text{o}},\, \lambda,\, t)\, L_{\text{i}}(\vec{r},\, \normvec{\omega}_{\text{i}},\, \lambda,\, t)\, (\normvec{\omega}_{\text{i}}\,\cdot\,\normvec{n})\, \operatorname d \normvec{\omega}_{\text{i}} \ + \\
\displaystyle \int_{\Omega_{\textbf{t}}} \rho'_{\textbf{t}}(\vec{r},\, \normvec{\omega}_{\text{i}},\, \normvec{\omega}_{\text{o}},\, \lambda,\, t)\, L_{\text{i}}(\vec{r},\, \normvec{\omega}_{\text{i}},\, \lambda,\, t)\, (\normvec{\omega}_{\text{i}}\,\cdot\,(-\normvec{n}))\, \operatorname d \normvec{\omega}_{\text{i}} & \!\!\!\! ,
\end{array}
\end{equation}

where $\Omega_{\textbf{r}}$ is the \idxs{unit}{hemisphere} containing all possible directions for the incoming reflected light, $\Omega_{\textbf{t}}$ is the \idxs{unit}{hemisphere} containing all possible directions for the incoming transmitted light (which is also the \idxs{relative}{complement} of $\Omega_{\textbf{r}}$ in the \idxs{unit}{sphere}, i.e. all points in the unit sphere that are not contained in $\Omega_{\textbf{r}}$), and $\rho'_{\textbf{t}}$ is the \BTDF which describes how light from different directions are transmitted through the surface.

The parameters $\vec{r}$, $t$ and $\lambda$ can be removed from the equation since they are of no importance to the derivation of the illumination model; one will just have to keep in mind that the functions depend on $\vec{r}$ and $t$ --- however, they don't have any specific dependence on $\lambda$. Even the term $L_{\text{e}}$ can be removed since water surfaces is generally not considered to emit any electromagnetic radiation. We can therefore simplify \eqref{eq:rendering_equation_improved} somewhat to

\begin{equation} \label{eq:rendering_equation_reduced}
\renewcommand*{\arraystretch}{1.5}
\begin{array}{cl}
L_{\text{o}}(\normvec{\omega}_{\text{o}}) \,= \\
\displaystyle \int_{\Omega_{\textbf{r}}} \rho'_{\textbf{r}}(\normvec{\omega}_{\text{i}},\, \normvec{\omega}_{\text{o}})\, L_{\text{i}}(\normvec{\omega}_{\text{i}})\, (\normvec{\omega}_{\text{i}}\,\cdot\,\normvec{n})\, \operatorname d \normvec{\omega}_{\text{i}} \ + \\[1ex]
\displaystyle \int_{\Omega_{\textbf{t}}} \rho'_{\textbf{t}}(\normvec{\omega}_{\text{i}},\, \normvec{\omega}_{\text{o}})\, L_{\text{i}}(\normvec{\omega}_{\text{i}})\, (\normvec{\omega}_{\text{i}}\,\cdot\,(-\normvec{n}))\, \operatorname d \normvec{\omega}_{\text{i}} & \!\!\!\! .
\end{array}
\end{equation}












\HRule

Suppose that we want to calculate the \shading on the simplified water surface caused by a \idxs{point}{light source}, we then need to know what the \idxs{bidirectional reflectance}{distribution function}\index{function!bidirectional reflectance distribution|see{bidirectional reflectance distribution function}}\index{reflectance distribution function!bidirectional|see{bidirectional reflectance distribution function}} of the simplified surface looks like.

Suppose that the non-simplified \idxs{free surface}{elevation}\index{surface elevation!free|see{free surface elevation}} on the location $\vec{r}$ is $\eta(\vec{r})$, where $\vec{r}$ is a \idxs{two-dimensional}{vector}, and that the simplified free surface elevation on the location $\vec{r}$, after the short wavelengths have been removed, is $\eta_0(\vec{r})$. Also suppose that $\eta_0$ is known, whereas $\eta$ is not. We can then only calculate the normal of the simplified surface, $\normvec{n}_0$. Let's also define the \idx{anti-normal} $\vec{\xi}$ of a normal $\normvec{\xi}$ as

\begin{equation}
\vec{\xi} \,=\, \frac{\normvec{\xi}}{\normvec{z}\cdot\normvec{\xi}}\,,
\end{equation}

where $\normvec{z}$ is the \idxs{unit}{vector} pointing \up (where up is the negation of the direction of the \idxs{gravitational acceleration}{vector}); this implies that the $\normvec{z}$-component of an anti-normal is always 1. The \idx{anti-normal} $\vec{n}_0$ of $\eta_0$ becomes

\begin{equation}
\vec{n}_0 \,=\, \left(\!\!\!\begin{array}{c}\nabla\eta_0(\vec{r}) \\ 1\end{array}\!\!\!\right).
\end{equation}

The gradient in turn can be rewritten as

\begin{equation}
\nabla\eta_0(\vec{r}) \,=\, \nabla\mathcal{F}^{-1}\{\fdfunc{\eta}_0(\vec{k})\}(\vec{r}) \,=\, \mathcal{F}^{-1}\{\vec{k}\,\fdfunc{\eta}_0(\vec{k})\}(\vec{r}),
\end{equation}

where $\mathcal{F}^{-1}$ is the \idxs{inverse}{Fourier transform} and $\fdfunc{\eta}_0(\vec{k})$ is the \idxs{Fourier}{transform} of $\eta_0$. The anti-normal of the real surface cannot be calculated since $\eta$ is unknown, but it can still be written as

\begin{equation}
\vec{n} \,=\, \left(\!\!\!\begin{array}{c}\nabla\eta(\vec{r}) \\ 1\end{array}\!\!\!\right),
\end{equation}

where the gradient can be written as

\begin{equation}
\nabla\eta(\vec{r}) \,=\, \mathcal{F}^{-1}\{\vec{k}\,\fdfunc{\eta}(\vec{k})\}(\vec{r}),
\end{equation}

where $\fdfunc{\eta}(\vec{k})$ is the \idxs{Fourier}{transform} of $\eta$. However, only the part of $\fdfunc{\eta}(\vec{k})$ where $\vec{k}$ is small enough is known; in fact, this part is equal to $\fdfunc{\eta}_0(\vec{k})$, while the rest of $\fdfunc{\eta}(\vec{k})$,

\begin{equation}
\Delta\fdfunc{\eta}(\vec{k}) \,=\, \fdfunc{\eta}(\vec{k}) - \fdfunc{\eta}_0(\vec{k}),
\end{equation}

is unknown. On the other hand, , if we know what the \idxs{wave}{spectrum} looks like, we can calculate the distribution of $\eta$ and thus calculate the \idxs{probability}{density} that it will match the anti-normalized halfway vector.

On the other hand, if we assume that the suppressed waves are small, we can make the \approximation that the \idxs{expectation}{value} of $\fdfunc{\eta}$ where the \idxs{view}{ray} hits is $\fdfunc{\eta}_0$, which means that

\begin{equation}
E_{\vec{k}}[\Delta\fdfunc{\eta}] \,=\, 0,
\end{equation}

where $E_{\vec{k}}$ denotes the expectation value for a specific value of $\vec{k}$, and if the \idxs{wave}{spectrum} is known, $\Var_{\vec{k}}(\Delta\fdfunc{\eta})$ is also known, where $\Var_{\vec{k}}$ denotes the \variance for a specific value of $\vec{k}$. Knowing this, we can calculate the mean value $\vec{\mu}$ of $\nabla\eta$,

\begin{equation}
\begin{array}{c}
\vec{\mu} \,=\, E[\nabla\eta(\vec{r})] \,=\, E\left[\mathcal{F}^{-1}\left\{\vec{k}\,\fdfunc{\eta}(\vec{k})\right\}(\vec{r})\right] \\
=\, E\left[\mathcal{F}^{-1}\left\{\vec{k}\,\left(\fdfunc{\eta}_0(\vec{k})+\Delta\fdfunc{\eta}(\vec{k})\right)\right\}(\vec{r})\right] \\
=\, \mathcal{F}^{-1}\left\{\vec{k}\,\left(E_{\vec{k}}[\fdfunc{\eta}_0]+E_{\vec{k}}[\Delta\fdfunc{\eta}]\right)\right\}(\vec{r}) \,=\, \mathcal{F}^{-1}\{\vec{k}\,\fdfunc{\eta}_0(\vec{k})\}(\vec{r}) \,=\, \nabla\eta_0(\vec{r})\,.
\end{array}
\end{equation}

\textit{\href{http://en.wikipedia.org/wiki/Multivariate\_normal\_distribution\#Non-degenerate\_case}{Multivariate normal distribution}}