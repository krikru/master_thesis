\part{Analysis and conclusions}

\chapter{Method summary}

For \thisprojectwork, the \FVM on a \idxs{stagered}{grid} has been used. The \PDEs that have been solved are the \idxs{Euler}{equations} and the flow has been considered to be \idxse{compressible}{flow}{compressible}.

The grid has been modeled with an \octree. The \LOD of the \idxsp{water surface}{cell}{s}\indexs{surface}{cell} depends on how visially important they are, and the \LOD of the \idxsp{bulk}{cell}{s} decreases the further down under the surface the cells are located. This  ensures that the total number $N_t$ of cells used in the simulation is roughly proportional to the number $N_s$ of cells visible on the surface according to \eqref{eq:number_of_cells_total_ordo}.

The interface has been modeled using the \VOF method and a variant of the \idxs{Hyper-C}{flux limiter}\index{limiter!Hyper-C flux limiter|see{Hyper-C flux limiter}} for compressible flow as \idxs{advection}{scheme}.

The region over the \idxs{water}{surface} is assumed to be \vacuum and is therefore not represented by cells at all, in order to save \idxs{computational}{power} (and since there would basically be an infinite number of those cells if they would have been represented). In order for the advection scheme to work properly, all cells with at least some water in them, i.e. $\alpha > 0$, also have to have neighbor cells in all directions the water can be advected, so whenever $\alpha$ for one cell goes from $\alpha = 0$ to $\alpha > 0$, all surfaces of the cell that don't border to another cell or to a \idxs{solid}{boundary} are found and cells with $\alpha = 0$ are created adjacent to those surfaces.

Also, whenever a cell with $\alpha = 0$ doesn't border to any cell with $\alpha > 0$, it is not needed any longer. The idea was to remove them, although this feature was never implemented. Although the advection scheme that is used is ideal in \idxs{one}{dimension}\index{dimensions!one|see{one dimension}}, proved to smear the \idxs{$\alpha$}{field} out somewhat in \idxse{two}{dimensions}{two} and \idxse{three}{dimensions}, making the interface between water and vacuum become thicker and thicker, which affects both the \idxs{simulation}{quality}, since the surface becomes less and less well-defined as the simulation goes on, as well as the \preformance of the \idxs{numerical}{method} since it will have to consider more and more cells as the interface gets thicker and thicker. Another advection scheme that maybe could remedy this problem is \MULES, described in \citep{Berberovi2009} and further developed to cope with more than two phases in \citep{Kissling2010}, although it would have to be modified to cope with compressible flow.


\chapter{Results}

The method chosen to simulate water waves in realtime, which was the \FVM on an \octree datastructure together with the \VOF, proved to be quite advanced and difficult to implemented properly within the time assigned for \thismasterthesiswork, which was \masterthesisworktime. As for the speed, it is not possible to 

For realtime simulation of surface waves in large bodies of water, where non-linear phenomena are not of any significant importance, the best method to use, considering both implementation time and simulation quality, is probably spectral methods.

\section{Speed}

\section{Accuracy}

\chapter{Discussion}

\section{Conclusions}

\subsection{Difficulties and drawbacks with the method}

Difficulties:
\begin{itemize}
    \item Dynamical creation/termination of surface cells and determination of properties in new cells
    \item High Courant numbers
    \item High speeds of sound (remedied in \textit{\href{http://physbam.stanford.edu/~kwatra/papers/compressible_semi_implicit/compressible_semi_implicit.pdf}{A method for avoiding the acoustic time step restriction in compressible flow}})
    \item Keeping a sharp interface
    \item Making it work in realtime
\end{itemize}

\section{Already existing software}

\begin{itemize}
    \item OpenFOAM (\red{FOAM, see} \textit{\href{http://powerlab.fsb.hr/ped/kturbo/openfoam/docs/foam.pdf}{A tensorial approach to computational continuum mechanics using object-oriented techniques}})
\end{itemize}

\chapter{Improvements}

\section{Parallellization}

See e.g. \textit{\href{http://gfs.sourceforge.net/papers/agbaglah2011.pdf}{Parallel simulation of multiphase flows using octree adaptivity and the volume-of-fluid method}}

\subsection{Space filling curves}

See e.g. \textit{\href{http://j.teresco.org/research/publications/octpart02/octpart02.pdf}{Dynamic Octree Load Balancing Using Space-Filling Curves}}

and \textit{\href{http://downloads.isrn.com/journals/appmath/2012/246491.pdf}{Parallel Adaptive Mesh Refinement Combined with Additive Multigrid for the Efficient Solution of the Poisson Equation}}

\section{Local-time stepping}

\section{Remedy for regions with high Courant number}

\section{Perfectly matched layers}

See e.g. \textit{\href{http://liu.diva-portal.org/smash/get/diva2:359805/FULLTEXT01}{Memory Efficient Methods for Eulerian Free Surface Fluid Animation}}

\section{Wind waves}

\subsection{Spectral methods}

\subsection{Air-water interaction}

\section{Visual effects}

\subsection{Splash and foam}

See \textit{\href{http://en.wikipedia.org/wiki/Sea_foam}{Wikipedia -- Sea foam}} or search for \textit{protein skimming} or \textit{foam fractionation}

Implemented in \textit{\href{http://nguyendangbinh.org/Proceedings/Eurographics/2003/cgf/volume22/issue3/paper127/paper127.pdf}{Realistic Animation of Fluid with Splash and Foam}}

\section{Sharpening of various advected fields}

\subsection{Backward Error Compensation and Forward Error Correction}

\begin{itemize}
    \item Reference: \textit{\href{http://smartech.gatech.edu/xmlui/bitstream/handle/1853/29473/2002-389.pdf}{Back and forth error compensation and correction methods for removing errors induced by uneven gradients of the level set function}}
    \item Applied to the velocity field and images: \textit{\href{http://www.gvu.gatech.edu/~jarek/papers/FlowFixer.pdf}{FlowFixer: Using BFECC for Fluid Simulation}}
\end{itemize}

\section{Code optimization}