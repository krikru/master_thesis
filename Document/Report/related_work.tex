\chapter{Related work}

%TODO: Why is CFD needed for this thesis work? (maybe this should go under motivation)

In computational physics, \CFD is a well established area of research, and a large number of widely differing methods have been developed within this field over the years. This \levelname will present and briefly discuss some of the most common methods for modeling and simulating fluids, and how suitable they are for \thisprojectwork.

\section{Two-dimensional methods}

\idxse{two-dimensional}{method}{Two-dimensional methods} for simulating water waves, sometimes also called 2.5 methods, are the easiest to understand and to implement and are often the fastest when it comes to simulation speed, but are not completely realistic models and therefore can't simulate all effects that can be simulated with a \idxs{three-dimensional}{method}, such as splashes or wave breaking, although attempts to extend such methods to cope with wave breaking have been made \citep[e.g.][]{Miklos2009}. Modeling \FSI with these methods can also be quite unintuitive.

\subsection{Two-dimensional PDEs for shallow water}

There exist a number of different two-dimensional \PDEs which describe the evolution of the \idx{free surface elevation}, $\eta$. The ordinary wave equation, \eqref{eq:wave_equation}, will as concluded work badly for simulating surface waves in \idxs{deep}{water}, since it doesn't handle wave dispersion at all, which quickly becomes obvious when wave patters such as the \idxs{Kelvin}{wave pattern} has to be simulated.

On the other hand, the wave equation does work for simulating very weak, low-frequent waves in \idxs{shallow}{water} with constant or very \idxs{mildly varying}{depth}, since those are close to linear, and are almost not affected by wave dispersion at all thanks to the low water depth. For waves with higher amplitude, there exists the \idxs{Boussinesq}{equations} and the \idxs{shallow water}{equations}, which both can be used to simulate large, non-breaking waves in shallow waters. Therefore, these equations are commonly used in \SWS. Shallow water equations are usually used to simulate waves whose \wavelength is similar to or greater than the overall \idxs{water}{height} \citep{Thurey2006}. These equations are often easily extended to handle \FSI.

Most of these method ensure the time complexity $O(N)$ per time step, where $N$ is the number of \idxs{surface}{grid points}.

In \appref{chap:pde_derivation}, a (family of) new two-dimensional \PDEs are derived and discussed, in an attempt to create something that could be used to simulate water waves at varying, arbitrary water depths.

\subsection{Fourier Synthesis}
%\subsection{Spectral methods}

This method \citep{Mastin1987} builds on \idxsp{Fourier}{transform}{ing} a representation of the surface in the frequency domain and has been extensively used and described in the world of computer graphics \citep{Monnier}. This method operates in the \idxs{frequency}{domain}, and uses \FFT, which here is the most time consuming process, to calculate the free surface elevation. It is characterized by high speed and doesn't (in contrast to commonly used two-dimensional \PDEs) have any problem with \idxs{wave}{dispersion}.

However, this method has been noted to be incompatible with \FSI \citep{Chentanez2011a}. Besides, it requires a constant water depth, and hence cannot simulate \idxs{wave}{shoaling}. This issue becomes noticeable when animating the surface water close to the \idxs{shore}{line}, where the water is shallow and waves naturally behave differently than what they do on deep water.

This method ensures the time complexity $O(N\,\log(N))$ per time step for trivial grid setups, where $N$ is the number of \idxs{surface}{grid points}.

\subsection{Laplacian Pyramid Decomposition}

A method that, just like the Fourier synthesis method, handles \idxs{wave}{dispersion} well but, unlike it, operates in the \idxs{spatial}{domain} and hence can handle local variations in the topography has recently been presented in a number of different works \citep{Ottosson2011,temp} \textit{\href{http://dice.se/wp-content/uploads/InteractiveWater_Lennartsson_dice.pdf}{Data Oriented Interactive Water}}.

The general idea behind the method (not necessarily fully described in the work by \citeauthor{Ottosson2011}) is to use a hierarchy of grids, where in the simplest case all grids cover the same surface, and where one step down in the hierarchy means a doubling in grid resolution or a division by two of the grid cell size, which in practice also means a division by two of the minimum wavelength that can be efficiently simulated on the specific grid.

Each grid is therefore responsible to store and process a specific set of wave vectors corresponding to this minimum wavelength. It should be easy to reconstruct any wave with a wave vector within that set, from its discretization on the grid. This set must also not contain any of the wave vectors that are processed on the grid one level up in the hierarchy. Together, all sets of wave vectors form a large set of wave vectors (including the zero vector) in which a very detailed free surface elevation, $\eta$, can be represented. The top level grid is only a few cells across and is responsible for processing all the shortest wave vectors. The free surface elevation is then reconstructed as the sum of all components, that is

\begin{equation} \label{eq:laplacian_pyramid_composition}
\eta \,=\, \sum_i \eta_i,
\end{equation}

where $\eta_i$ is the component represented on the $i$th grid.

In order to decompose the free surface elevation into the components that are to be processed on the different grids, the method uses \LPD, which is a general decomposition method that can operate on a field of any number of dimensions (but is in this case used on a discretized two-dimensional field). An \LPD closely resembles the \idxs{wavelet}{transform} (although unlike wavelets, which are somewhat localized both in the spatial domain and in the frequency domain, the components of a Laplacian pyramid are somewhat localized in the frequency domain, but generally completely delocalized in the spatial domain).

This decomposition makes it possible to describe the time evolution of the free surface elevation component represented on each grid with a \PDE represented in the spatial domain, and still include dispersion. By substituting $f$ for $\omega^2$ and $x$ for $k^2$ in \eqref{eq:dispersion} and by realizing that $\omega, k \geq 0$, we obtain the relation

\begin{equation} \label{eq:dispersion_modified}
f(x) \,=\, \left(g\,+\,\frac{\gamma}{\rho}\,x\right)\,\sqrt{x}\,\tanh(\sqrt{x}\,h).
\end{equation}

Since the wavelengths that are represented on the $i$th grid in the method are restricted to a certain range, $x$ also becomes restricted to a certain range, and we can \approximate $f$ as a low order Taylor series around a value $a_i$ chosen to be somewhere in that range. That is

\begin{equation} \label{eq:dispersion_modified_taylor}
\renewcommand*{\arraystretch}{2}
\begin{array}{c}
\displaystyle f(x) \,=\, f(a_i) \,+\, \frac{f'(a_i)}{1!}(x-a_i)^1 \,+\, \frac{f''(a_i)}{2!}(x-a_i)^2 \,+\, \dots \\
\displaystyle +\, \frac{f^{(n_i)}(a_i)}{n_i!}(x-a_i)^{n_i},
\end{array}
\end{equation}

where $n_i$ is the order of the Taylor series, and by substituting back $\omega^2$ for $f(x)$ and $k^2$ for $x$ we obtain

\begin{equation} \label{eq:dispersion_taylor}
\renewcommand*{\arraystretch}{2}
\begin{array}{c}
\displaystyle \omega^2 \,=\, f(a_i) \,+\, \frac{f'(a_i)}{1!}(k^2-a_i)^1 \,+\, \frac{f''(a_i)}{2!}(k^2-a_i)^2 \,+\, \dots \\
\displaystyle +\, \frac{f^{(n_i)}(a_i)}{n_i!}(k^2-a_i)^{n_i}.
\end{array}
\end{equation}

Since a wave of a single wavelength can be expressed on the form

\begin{equation} \label{eq:component_sin}
\eta_{\vec{k}}(\vec{r},\,t) \,=\, A\,\sin(\alpha + \vec{k}\cdot\vec{r}\,-\,\omega\,t),
\end{equation}

where $\eta_{\vec{k}}$ is the free surface elevation consisting only of a wave with the wave vector $\vec{k}$, $A$ is the amplitude of the wave, and $\alpha$ is the phase of the wave, we can conclude that

\begin{equation} \label{eq:component_second_time_derivative}
\frac{\partial^2}{\partial t^2}\eta_{\vec{k}} \,=\, -\omega^2 A\,\sin(\alpha + \vec{k}\cdot\vec{r}\,-\,\omega\,t) \,=\, -\omega^2\eta_{\vec{k}},
\end{equation}

and 

\begin{equation} \label{eq:component_laplacian}
\nabla^2\eta_{\vec{k}} \,=\, -k^2 A\,\sin(\alpha + \vec{k}\cdot\vec{r}\,-\,\omega\,t) \,=\, -k^2\eta_{\vec{k}}.
\end{equation}

If we multiply both sides of \eqref{eq:dispersion_taylor} with $\eta$, we can use \eqref{eq:component_second_time_derivative} and \eqref{eq:component_laplacian} to obtain

\begin{equation} \label{eq:taylor_pde}
\renewcommand*{\arraystretch}{2}
\begin{array}{c}
\displaystyle \frac{\partial^2}{\partial t^2}\eta_{i,\vec{k}} \,=\, -\left(f(a_i) \,+\, \frac{f'(a_i)}{1!}(-\nabla^2-a_i)^1 \,+\, \frac{f''(a_i)}{2!}(-\nabla^2-a_i)^2 \,+\, \dots \right.\\
\displaystyle \left. +\, \frac{f^{(n_i)}(a_i)}{n_i!}(-\nabla^2-a_i)^{n_i}\right)\eta_{i,\vec{k}},
\end{array}
\end{equation}

where an $i$ has been added to the index of $\eta$ to indicate that this \PDE is only valid for waves that are described by \eqref{eq:component_sin} and are represented in $\eta_i$. Since this \PDE otherwise holds for all $\omega$ and $\vec{k}$, if we assume that the surface is linear, the superposition principle tells us that the equation can be used to describe the time evolution of $\eta_i$, and we can remove the $\vec{k}$ completely from the equation.

However, one has to keep in mind that $f$, as well as all of its derivatives, depend on the water depth $h$, which in turn can vary between different locations $\vec{r}$ and times $t$. If $k\,h \gg 1$, the $\tanh$ factor in \eqref{eq:dispersion_modified} can be \approximated as 1 and then the $h$ dependence vanishes. But if there exists waves with wavelengths comparable or longer than the water depth, this can be problematic. In that case, for the coarser grids, on which waves are represented for which this is the case, as well as on finer grids close to the shoreline where the water is shallow, the derivatives of $f$ used in the Taylor series may have to be computed separately for each grid point, and perhaps also recomputed at even intervals if the water is not calm and the water depth changes much.

Another problem, that it is easy to overlook, is the discretization of the $\nabla^2$ operator. To use a naive three-point second order derivative operator $D^2$ in each dimension in $\nabla^2$ is not sufficient. To illustrate, if $D^2$ operates on a one-dimensional wave, $\sin(k\,x)$, the result will be

\begin{equation}
\renewcommand*{\arraystretch}{2}
\begin{array}{c}
\displaystyle D^2\sin(k\,x) \,=\, \frac{\sin(k\,(x-\Delta x)) - 2\,\sin(k\,x) + \sin(k\,(x+\Delta x))}{\Delta x^2} \\
\displaystyle \,=\, -\sin(k\,x)\,\frac{2\,(1-\cos(k\,\Delta x))}{\Delta x^2},
\end{array}
\end{equation}

where $\Delta x$ is the grid cell size. Hence, $D^2$ will underestimate the real second order derivative which is $-k^2\sin(k\,x)$ (this is especially true for high $k$-values). A more suitable operator can be constructed as an $n_i$th order polynomial of $D^2$. When \eqref{eq:taylor_pde} is then discretized, this polynomial is substituted for $\nabla^2$, and any terms containing powers of $D^2$ with an exponent higher than $n_i$, after the powers in \eqref{eq:taylor_pde} have been expanded, are simply neglected because they are of too high order.

This method can be extended to handle \FSI \citep{Ottosson2011}, and by assuming that all grids cover the same surface, it has the time complexity $O(N)$ per time step, where $N$ is the number of \idxs{surface}{grid points} in the grid with the highest resolution. If the \idxs{level of}{detail} depends on the location on the surface, why the grid with the highest resolution would not stretch over the entire simulated region, $N$ is the number of grid points located on a grid and on a location for which there are no grid with higher resolution, stretching over the same region.

\section{Three-dimensional methods}

\idxse{three-dimensional}{method}{Three-dimensional methods} are often highly realistic in the sense that they will be able to simulate all different kind of phenomenas that can be described with the \idxs{Navier--Stokes}{equations} (see \secref{sec:ns_equations}). There are a few exceptions though.

\subsection{Smoothed-Particle Hydrodynamics}

The \SPH method is a highly realistic model that simulates a flow by simulating a large number of small particles. Between each pair of particles that are within a certain \idxs{cut-off}{distance} from each other, there is a repelling or attractive force, described by an \ODE. The interaction between two particles in the simulation is usually modeled by a potential like those used in \MD, for example a \LJ potential that has been softened to limit the maximum acceleration that particles can be exposed to. The cut-off distance is used in order to ensure that the number of interactions is $O(N)$, an not $O(N^2)$ as for a system where all pair of particles interact with each other, where $N$ is the number of particles in the system.

There are a few major advantages with using this method. When the \idxs{Eulerian}{specification of the flow field} is used to describe the fluid motion, the equations tend to become more complicated as they contain \idxsp{advection}{term}{s}. In \SPH on the other hand, the \idxs{Lagrangian}{specification of the flow field} is used and no advection terms are therefore present in the equations which makes the model relatively simple, and it is easily implemented. Besides, no advection of fields with an Eulerian representation is simulated, which prevents additional problems that can arise during the advection, and conservation of various properties, like \momentum and \energy is usually automatically well preserved as a result of that. In fluid simulations, there is no need to model the air, and there is no need to keep track of where the surface of the fluid is located since this is information that can be extracted during the \idx{post-processing} phase.

On the other hand, the \SPH method requires that the entire simulation domain is filled with small particles, which often means that an \emph{extremely} large number of particles, proportionally to the volume of the fluid, have to be simulated. This implies a very heavy workload on the computer, and as a result of that, \SPH is very seldom used in \idxsp{real-time}{simulation}{s}. However, adaptive particle sizes have been used in order to reduce the amount of particles needed in the less important parts of the fluid, like in the bulk, in order to remedy this problem. It has since its first implementation \citep{Desbrun1999} been improved upon a number of times \citep[e.g.][]{Yan2009}.

For a grid with a random access time complexity of $O(1)$ and for particles that all have the same size, this method ensures the time complexity $O(N)$ per time step, where $N$ is the number of particles.

\subsection{Finite Volume Method}

The \FVM is a highly realistic model that solves a set of \PDEs by dividing the region of interest into small \idxsp{volume}{element}{s}, and by discretizing the fields that are described by the \PDEs into points in the volume elements or on the border of the volume elements, usually with a fixed number of points per volume element, as well as discretizing the \PDEs into a number of \ODEs that describe the evolution of the discretized fields. The volume elements are commonly referred to as \cells. The \FVM and its usage in \CFD is described in greater detail in \chapref{chap:ns_equations}.

The \FVM ensures the time complexity $O(N)$ per time step, where $N$ is the number of cells.

\subsection{Finite Volume Method on a restricted tall cell grid}

This approach \citep{Chentanez2011} uses an orthogonal grid, where the water closest to the surface is modeled with small \idxsp{cubic}{cell}{s}, and the water deeper down is modeled with tall cells that stretch vertically down all the way from where the surface cells end to where the bottom is located (where the water ends). The horizontal size of the tall cells is on the other hand the same as for the cubic cells close to the surface.

The advantages with this approach is that it significantly reduces the number of cells that have to be processed by \approximating a large number of small cells as a much smaller number of tall cells, and it will still catch surface disturbances and simulate waves with short wavelengths with a high accuracy, and it also simulates waves with really long wave lengths relative to the water depth and an overall motion of the water with a high accuracy. On the other hand, it cannot simulate waves with intermediate wavelength with a very high accuracy.

This method is ideal for simulating flowing water when the main focus does not lie on simulating surface wave properly. However, for simulating an ocean where one often focuses on getting the correct speed for all wavelengths, the tall cells are not very well suited.

\subsection{Finite Volume Method on an octree grid}

The aim of this method, is the same as the aim of the restricted tall cell grid approach, to reduce the number of cells that are needed in the simulation. It does so by modeling the grid with an \octree which allows for easy \idxs{adaptive}{mesh refinement}. This method was probably first implemented in \citeyear{Popinet2003} \citep{Popinet2003}, but has since been implemented a number of times \citep[e.g][]{Losasso2004,openfoam}. The fluid in important regions of the simulation, such as that closest to a surface, is therefore modeled with a fine grid to capture the small scale visual detail, while fluid further away from such regions are modeled with increasingly larger cell sizes the further away they get.

Using this method to simulate surface waves propagating on a large body of water allows an arbitrarily high accuracy, depending on how quickly the cells grow in size when the get farther away from regions of high importance to the simulation, no matter how large the wavelength is. However, it may, just like any other method that discretizes a set of \PDEs represented in the spatial domain, have some problems with getting the correct wave speed for waves with wavelengths close to the \idxs{Nyquist}{frequency}.

It also guarantees that the number of cells that are used in the simulation is $O(N_{\text{s}})$, where $N_{\text{s}}$ is the number of cells visible on the surface, so it ensures the time complexity $O(N_{\text{s}})$ per time step.

\section{Hybrid two- and three-dimensional methods}

\idxse{two-dimensional}{method}{Two-dimensional methods} is usually much faster than \idxsp{three-dimensional}{method}{s}. On the other hand, three-dimensional methods often have a level of realism that you can't find among two-dimensional methods, and are able to simulate phenomena such as splashes or wave breaking, and the mathematical model describing \FSI often follows naturally. For that reason, \idxe{hybrid method}{hybrid two-dimensional and three-dimensional methods} have been developed, which aim to combine the strengths of two-dimensional and three-dimensional methods and overcome their weaknesses by simulating regions with more complex water motion using with three-dimensional method and regions with less complex water motion with a two-dimensional method, and then couple these simulations with each other. Regions with complex water motion as close to a moving structure such as a ship, and close to the \shoreline if \idxs{wave}{breaking} or \idxs{wave}{shoaling} is important.

In an implementation from \citeyear{Thurey2006} \citep{Thurey2006}, a method known as the \idxs{Lattice-Boltzmann}{method}, which is similar to the \FVM, was used to simulate water in a small box, and was coupled to an \SWS taking place outside of the box. It turned out that with a two-dimensional region covering an area 35 times the size of the area over the three-dimensional region, the three-dimensional region still required more than two thirds of the entire simulation time, and updating a three-dimensional cell took in average three times as long as updating a two-dimensional cell. Although the simulation didn't run in real-time, it was further concluded that given enough computational resources, and in combination with adaptive grids, parallelization and low grid-resolutions, this could be used for interactive, real-time simulations of large water surfaces.

\section{Miscellaneous other methods}

Except from the methods already covered in this chapter, there are a few other methods commonly used in \CFD which for selected reasons are not suitable for simulating oceans.

We have the \LBM which has connections to cellular automata. The \LBM solves the \idxs{discrete}{Boltzmann equation}\index{equation!discrete Boltzmann|see{discrete Boltzmann equation}} and is basically a less powerful \CFD method than the \FVM.

We have the \MAC method, which is a \FVM simulation in which many small, massless marker particles are initially homogeneously distributed in the fluid and then carried with the flow to mark the presence of fluid, but this is basically a less powerful method than the \idxs{volume of fluid}{method}.

We have the \BEM in which the equations of motion are converted to \idxsp{integral}{equation}{s} and solved solely from the \boundaries, and has some interesting potential for the simulation of surface waves \citep{advancesinnumericalsimulationofnonlinearwaterwaves}. However, it belongs to the category of computationally expensive methods and is for that reason not suitable for simulating the surface of a large body of water in real-time.

We also have the \FEM which is the Lagrangian correspondence to the \FVM, where (like in the \FVM) the simulated region is divided into many small elements, or cells, but where (unlike in the \FVM) the cells follow the flow instead of being stationary and letting the flow pass through the cell walls. The \FEM have some advantages over the \FVM, like using the Lagrangian versions of the equations of motion which in contrast to the Euclidean versions don't contain advection terms and hence are not subject to smearing. It also naturally provides a representation of fluid interfaces. However, the \FEM is more complicated to implement than the \FVM. Cells change shape and are stretched out, which makes re-meshing necessary, and the \FEM is still not guaranteed to give results better than, or even as good as, those of the \FVM. Besides, the \FEM, which is very often used in solid state mechanics simulations, is not that well established in the field of \CFD, while the \FVM is somewhat of an industry standard.