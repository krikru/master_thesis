% !TEX TS-program = pdflatex

\documentclass[a4paper,english,10pt,ifm]{liuthesis}
% The option ifm will automatically include information specific for
% IFM.  The option crop show crop marks, useful in drafts, but should
% be removed for the version to send to printing.

% ====================================================================
% Included packages --------------------------------------------------
% ====================================================================
%\usepackage[swedish]{babel}
\usepackage[Sonny]{fncychap}
%\usepackage[Lenny]{fncychap}
%\usepackage[Glenn]{fncychap}
%\usepackage[Conny]{fncychap}
%\usepackage[Rejne]{fncychap}
%\usepackage[Bjarne]{fncychap}

\usepackage{graphicx}% Include figure files
\usepackage{psfrag,wrapfig}
\usepackage{bm}
\usepackage{subfigure}
\usepackage{float}
\usepackage{booktabs}


\usepackage[rightcaption]{sidecap} % Makes it possible to have captions beside floats
\sidecaptionvpos{figure}{t} % Defines the alignment of the figure-caption [b,c,t]
\sidecaptionvpos{table}{t} % Defines the alignment of the table-caption [b,c,t]

%\usepackage[usenames,dvipsnames]{color}
\usepackage{color}
%\usepackage{doublespace}

% Use the following command to get clickable references in your
% pdf-files.  Nb, it has to be put after all \usepackage-commands in
% order to work properly!
\useHyperRef

%\makeatletter
%
%\newcommand*{\ps@refhead}{%
%  \renewcommand*{\@oddfoot}{}%
%  \renewcommand*{\@evenfoot}{}%
%  \renewcommand*{\@evenhead}{%
%    \parbox{\textwidth}{\bfseries \thepage\hfill%
%      \vspace*{2\p@}\hrule}}%
%  \renewcommand*{\@oddhead}{%
%    \parbox{\textwidth}{\thepage%
%      \vspace*{2\p@}\hrule}
%}}
%
%\makeatother

% ====================================================================
% Own definitions and commands ---------------------------------------
% ====================================================================
%\setcounter{tocdepth}{1}% Set the depth of table of contents (1<-> Inclusion of \section, 2<->inclusion of \subsection and so forth)
\def \etal{{\it{et~al.}}}
\def \eg{e.g., }
\def \ie{i.e., }
\def \eq{Eq.}
\def \eqs{Eqs.}

\newcommand{\ud}{\mathrm{d}}
\newcommand{\beq}{\begin{equation}}
\newcommand{\eeq}{\end{equation}}
\newcommand{\tm}[1]{\textrm{#1}}
\newtheorem{thm}{Theorem}


\definecolor{PAPERCOVER}{RGB}{191,191,191}% Same grey color as on the frontpage
%\definecolor{PAPERCOVER}{RGB}{125,125,125}% grey 49
%\definecolor{PAPERCOVER}{RGB}{36,36,36}% 
% ====================================================================
% Definitions required for the thesis --------------------------------
% ====================================================================

% Declare type of thesis.
\thesisCategory{PhD}

\author{Name}
\title{Svensk titel}{Eng  titel}


% Date of presentation. (month, date, year)
\thesisDate{12}{17}{2010}

% Series of the thesis.
\series{Series}

% Thesis ISBN number
\ISBN{ISBN}

%\URL{url}

% Thesis ISSN number
\ISSN{ISSN}

% This is the information about where the thesis have been printed.
% It will be typeset on the back of the title page.
\printshop{Printed by LiU-Tryck, Link{\"o}ping 2010}

% If you have a picture on the front page that you want to explain,
% you can write it here and it will be put on the back of the title
% page.
\frontpageImage{%The picture of the frontpage show...
}

% Division at which the thesis was carried out, e.g., Computational
% Physics.
\thesisDivision{Theoretical and Computational Physics}
\supervisor{Igor Abrikosov \AT Theoretical and Computational Physics}
%\examiner{Igor Abrikosov \AT Theoretical and Computational Physics}

% Subject of the thesis, e.g., Computational Physics.
\subject{Theoretical and Computational Physics}

% What keywords to be put on the library page
\keywords{Keywords}

% If there is anyone you want to dedicate the thesis to, e.g.,
% girlfriend/boyfriend, husband/wife, etc., or any other wise things
% you want to have on a page in the beginning is written here.  If you
% don't want anything, remove this statement completely, otherwise it
% won't work!!!
%\dedication{\Large To my wife {\&} our\\ lovely daughter}
\dedication{\large To Bj\"orn}

% Import file containing your abstract.
%\notocsection[0]{Abstract}{theabstractanchor}

\pdfbookmark[0]{\abstractname}{theabstractanchor}
\begin{abstract}
The problem of real-time simulation of ocean surface waves, ship movement and the coupling in between is tackled, and a number of different methods are covered and discussed, of which the finite volume method has been implemented in an attempt to solve the problem, along with the compressible Euler equations, an octree based staggered grid which allows for easy adaptive mesh refinement, the volume of fluid method and a variant of the Hyper-C advection scheme for compressible flows for advection of the phase fraction field.
    
The process of implementing the methods that were chosen proves to be tricky in many ways, and involves a large number of topics that are very advanced only by themselves, and the implementation that was implemented in \thismasterthesiswork suffered from numerous issues, including problems with keeping the interface intact and a harsh restriction on the time-step size due to the CFL condition. Improvements required to make the method sustainable are discussed, and a few suggestions on alternative approaches that are already in use for very similar purposes are also given and discussed.
    
Furthermore, a method for compensating for gain/loss of mass when solving the incompressible flow equations with an inaccurately solved pressure Poison equation is presented and discussed. A momentum conservative method for transporting the velocity field on staggered grids without introducing unnecessary smearing is also presented and implemented. A simple, physically based illumination model for sea surfaces is derived and compared to the Blinn--Phong shading model, although it is never implemented. Finally, a two-dimensional partial differential equation in the spatial domain for simulating water surface waves for mildly varying bottom topography is derived and discussed, although it is deemed to be too slow for real-time purposes and is therefore never implemented.
\end{abstract}

% Import file containing your acknowledgements.

%-*- mode: LaTeX; -*-

\acknowledgements{
Hej
}
% Local Variables:
% TeX-master: "report.ltx"
% End: 

% ====================================================================
% Begining of the document -------------------------------------------
% ====================================================================

\begin{document}
%%%%%%%%%%%%%%%%%%%%%%%%%%%%%%%%%%%%%%%%%%%
% Create the loose abstract to be supplied with all copies of the PhD
% thesis.
%%%%%%%%%%%%%%%%%%%%%%%%%%%%%%%%%%%%%%%%%%%
%\makeLooseAbstract{\noindent som f\"or avl\"aggande av teknologie
%  doktorsexamen vid Link\"opings universitet kommer att offentligt
%  f\"orsvaras i h\"orsal Planck, Fysikhuset, Link\"opings universitet,
%  fredagen den 17 december 2010, kl. 10.15. 
%  Opponent \"ar Prof.~Josef~Kudrnovsk\'y, 
%  Institute of Physics, Academy of Sciences of the Czech Republic.} 
%%%%%%%%%%%%%%%%%%%%%%%%%%%%%%%%%%%%%%%%%%%

%  Na Slovance 2, CZ-182 21 Prague 8, Czech Republic.}

% Create all interesting things you want in the beginning of your
% document;  Front page, title page, dedication (if there is one),
% abstract, preface, and Table of Contents.

\makeFrontMatter

% ====================================================================
% Main content -----------------------------------------------------
% ====================================================================

% Start writing your text here.  For clarity it's advisable that you
% write each chapter, included paper etc. into separate files and
% include them here.  Below the examples are written out here to
% minimize the number of files to include.


\chapter{Introduction}
%\label{cha:introduction}
%\part{Introduction}


%\fbox{\setlength\fboxsep{1pt}
%\fbox{\setlength\fboxsep{2pt}
%\fbox{\setlength\fboxsep{3pt}
%\fbox{\setlength\fboxsep{4pt}
%\fbox{\setlength\fboxsep{5pt}
%\fbox{\setlength\fboxsep{6pt}
%\fbox{\setlength\fboxsep{7pt}
%\fbox{\setlength\fboxsep{8pt}
%hej!
%}
%}
%}
%}
%}
%}
%}
%}

\subsection{Motivation}

\begin{frame}[<+(1)->]
\frametitle{Motivation}

\uncover<+(1)->{Helikoptersimulator.}
%\begin{itemize}[<+(1)->]
%\item Helikoptersimulator
%item Wave dispersion
%end{itemize}

\uncover<+(1)->{\href{http://www.youtube.com/watch?v=bC2XIGMI2kM}{Lynx Helicopter Operating Limit Development}}

\end{frame}

\chapter{Requirements and difficulties}
\label{chap:requirementsanddifficulties}

\section{Wave dispersion and non-linearity}

One of the greatest challenges when simulating an ocean is the fact that ocean waves are subject to \idxs{wave}{dispersion}, also known as \indexify{frequency dispersion}\index{frequency dispersion!see{dispersion}}, which means that waves with different \wavelengths travel with different speeds. This means that the wave equation,
%
\begin{equation} \label{eq:wave_equation}
\frac{\partial^2 \eta}{\partial t^2} \,=\, c^2\nabla^2\eta,
\end{equation}
%
--- a \PDE\xspace --- which otherwise both has a simple definition and is simple to solve numerically, cannot be used, since it assumes that waves of all wavelengths travel with a single speed, $c$. Here, $\eta$ is the free surface elevation as a function of the the horizontal location, $\vec{r}$, and the time, $t$, and $\nabla$ is the \idxs{del}{operator}\index{$\nabla$|see{del operator}} which is commonly used in \idxs{vector}{algebra}.

In Airy wave theory, which treats the propagation of surface waves, the dispersion relation
%
\begin{equation} \label{eq:dispersion}
\omega^2(k) \,=\, \left(g\,+\,\frac{\gamma}{\rho}\,k^2\right)\,k\,\tanh(k\,h),
\end{equation}
%
is derived for water with no mean velocity \citep{Phillip1977}. Here, $\omega$ is the \idxs{angular}{frequency} of one \idxs{wave}{component}, $k$ is the \idx{wavenumber} of the component, $g$ is the \idxs{gravitational}{acceleration}, $\gamma$ is the \idxs{surface}{tension}, $\rho$ is the water density and $h$ is the \idxs{water}{depth}.

While this equation describes the propagation of a wave composed of a single wavelength very well, it cannot tell how a free surface elevation, $\eta$, consisting of multiple wavelengths will evolve. Only if the wave amplitude is very small (typically such that $|\nabla\eta| \ll 1$ and $|\eta| \ll h$, where $h$ is the water depth) can the surface be approximated as linear, and waves with different wavelengths can be individually described by \eqref{eq:dispersion} and superposed on top of each other to form the free surface elevation, without introducing too much error. If the \idxs{wave}{amplitude} on the other hand is not that small, strong non-linear phenomena are likely to take place, including for example \idxs{wave}{breaking}, which won't be caught in the simulation if the surface is linearized.

What is maybe even worse is that there is no simple way of turning this equation into a \PDE represented in the spatial domain (like \eqref{eq:wave_equation}). This increases the difficulty to describe the evolution of the free surface elevation significantly, even for a surface that has already been linearized. This is typically solved by transforming the free surface elevation in some way before processing it.

\section{Fluid--Structure Interaction}

As noted previously, ships have to be affected by waves, and ships also have to give rise to waves. Hence, there has to be a \idxs{two-way}{interaction} between water and ships. For most of the two-dimensional wave models, which just treat the surface as a \idxs{height}{map}, there is no natural way to make water and ships interact with each other.

The ships can quite easily be made to roll in a realistic way by just approximating the \idxs{pressure}{field} felt by the \idxs{ship}{hull} by looking at the free surface elevation, even though this method is slightly incorrect since it doesn't take into account the deviations the ship itself causes the pressure field. But to make ships give rise to waves as they are traveling on the water is more challenging.

One possibility is to use a separate, static height map for the wake, which moves after the ship as it is traveling. This wake will always look the same no matter where the ship is traveling and will not be affected by obstacles in the water. However, if the ship suddenly changes speed or course, so does the wake, which is a highly unnatural behavior for a wake.

A better approach may be to use a \idxs{response}{map} that tells the water how to respond when a ship is traveling on it. In that case, the wake will not be stored as a separate height map, but be merged into the same height map that is used to simulate the waves that affect the ship. The response would of course also depend on the speed of the ship so that the faster the ship goes, the higher the generated waves will be, and for a non-moving ship, there will be no waves generated.
\begin{frame}[<+(1)->]
\frametitle{Related work}

Two-dimensional methods

\begin{itemize}
\item item 1
\item item 2
\item item 3
\end{itemize}

Three-dimensional methods

\begin{itemize}
\item item 4
\item item 5
\item item 6
\end{itemize}

\end{frame}
\clearpage

%\chapter{Density-functional theory}
%\label{cha:dft}
%\input{./chapters/dft.tex}
\clearpage

% ====================================================================
% Bibliography -------------------------------------------------------
% ====================================================================
%\thispagestyle{refhead}

%\bibliographystyle{unsrt}
%\bibliography{references}

\cleardoublepage

% ====================================================================
% List of publications -----------------------------------------------
% ====================================================================

% Since I haven't found a proper way of mastering BibTeX to generate
% this in a nice way, I do it the foul way.  It's not nice but it
% works.  Following is a short description on how to do it, if you
% don't understand, ask!  I'll try to fix this up asap, so that it
% works in a fashion similar to the bibliography, see above, but it
% might be a bit trickier than I thought.  Help and all suggestions
% are welcome.
%
% 1) Create a .bib-file containing only your papers (here called
%    own_papers.bib. 
%
% 2) Look in the file lop-gen.aux.  This one contains three lines:
%    \citation{*} makes all entries go into the list of publications
%    \bibdata{} tells the name of the .bib-file created in 1).
%    \bibstyle tells the wanted bibliography style
%
% 3) Run bibtex on lop-gen.aux.  This will generate the lop-gen.bbl
%    ouput file.
%
% 4) Open the lop-gen.bbl file, copy the content into a .tex-file to
%    be included here (here called lop.tex), but change the
%    environment from thebibliography to thelistofpublications.
%
% 5) You're all done and you automatically have your own BibTeX styled
%    list of publications.

% \include{papers/own_bib/lop}
 \cleardoublepage
% ====================================================================
% Papers -------------------------------------------------------------
% ====================================================================

% Here you start an environment for including papers.
%: SECTION:PAPERS
%\papers

% To get an automatically generated front page for each included paper
% use the command \paper, it takes as arguments the title, the
% authors, and the journal.  The thumb indices are generated so that
% ten will fit on a page no matter how many papers you have.  If you
% have more than ten, the thumb indices will wrap around and start at
% the top of the page.

%%%%%%%%%%%%%%%%%%%%%%%%%%%%%%%%%%%%%%%%%%%%%%%%%%%%%%%%%%%%%%%
% PAPER I -- PAPER V: LEE in BULK (CLS + Auger)
%%%%%%%%%%%%%%%%%%%%%%%%%%%%%%%%%%%%%%%%%%%%%%%%%%%%%%%%%%%%%%%
% 1
%\pagecolor{PAPERCOVER}

% ====================================================================
% Appendices ---------------------------------------------------------
% ====================================================================

%\appendix


% ====================================================================
% Library page -------------------------------------------------------
% ====================================================================

% The library page will be automatically filled out using submitted
% information.  On general request it is added at the end.

\cleardoublepage

% Kommentera detta om du inte vill ha ngt biblioteksblad i slutet
%\makeLibraryPage \thispagestyle{empty}

% ====================================================================
% All done -----------------------------------------------------------
% ====================================================================

\end{document}
